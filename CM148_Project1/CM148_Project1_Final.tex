\documentclass[11pt]{article}

    \usepackage[breakable]{tcolorbox}
    \usepackage{parskip} % Stop auto-indenting (to mimic markdown behaviour)
    

    % Basic figure setup, for now with no caption control since it's done
    % automatically by Pandoc (which extracts ![](path) syntax from Markdown).
    \usepackage{graphicx}
    % Maintain compatibility with old templates. Remove in nbconvert 6.0
    \let\Oldincludegraphics\includegraphics
    % Ensure that by default, figures have no caption (until we provide a
    % proper Figure object with a Caption API and a way to capture that
    % in the conversion process - todo).
    \usepackage{caption}
    \DeclareCaptionFormat{nocaption}{}
    \captionsetup{format=nocaption,aboveskip=0pt,belowskip=0pt}

    \usepackage{float}
    \floatplacement{figure}{H} % forces figures to be placed at the correct location
    \usepackage{xcolor} % Allow colors to be defined
    \usepackage{enumerate} % Needed for markdown enumerations to work
    \usepackage{geometry} % Used to adjust the document margins
    \usepackage{amsmath} % Equations
    \usepackage{amssymb} % Equations
    \usepackage{textcomp} % defines textquotesingle
    % Hack from http://tex.stackexchange.com/a/47451/13684:
    \AtBeginDocument{%
        \def\PYZsq{\textquotesingle}% Upright quotes in Pygmentized code
    }
    \usepackage{upquote} % Upright quotes for verbatim code
    \usepackage{eurosym} % defines \euro

    \usepackage{iftex}
    \ifPDFTeX
        \usepackage[T1]{fontenc}
        \IfFileExists{alphabeta.sty}{
              \usepackage{alphabeta}
          }{
              \usepackage[mathletters]{ucs}
              \usepackage[utf8x]{inputenc}
          }
    \else
        \usepackage{fontspec}
        \usepackage{unicode-math}
    \fi

    \usepackage{fancyvrb} % verbatim replacement that allows latex
    \usepackage{grffile} % extends the file name processing of package graphics
                         % to support a larger range
    \makeatletter % fix for old versions of grffile with XeLaTeX
    \@ifpackagelater{grffile}{2019/11/01}
    {
      % Do nothing on new versions
    }
    {
      \def\Gread@@xetex#1{%
        \IfFileExists{"\Gin@base".bb}%
        {\Gread@eps{\Gin@base.bb}}%
        {\Gread@@xetex@aux#1}%
      }
    }
    \makeatother
    \usepackage[Export]{adjustbox} % Used to constrain images to a maximum size
    \adjustboxset{max size={0.9\linewidth}{0.9\paperheight}}

    % The hyperref package gives us a pdf with properly built
    % internal navigation ('pdf bookmarks' for the table of contents,
    % internal cross-reference links, web links for URLs, etc.)
    \usepackage{hyperref}
    % The default LaTeX title has an obnoxious amount of whitespace. By default,
    % titling removes some of it. It also provides customization options.
    \usepackage{titling}
    \usepackage{longtable} % longtable support required by pandoc >1.10
    \usepackage{booktabs}  % table support for pandoc > 1.12.2
    \usepackage{array}     % table support for pandoc >= 2.11.3
    \usepackage{calc}      % table minipage width calculation for pandoc >= 2.11.1
    \usepackage[inline]{enumitem} % IRkernel/repr support (it uses the enumerate* environment)
    \usepackage[normalem]{ulem} % ulem is needed to support strikethroughs (\sout)
                                % normalem makes italics be italics, not underlines
    \usepackage{mathrsfs}
    

    
    % Colors for the hyperref package
    \definecolor{urlcolor}{rgb}{0,.145,.698}
    \definecolor{linkcolor}{rgb}{.71,0.21,0.01}
    \definecolor{citecolor}{rgb}{.12,.54,.11}

    % ANSI colors
    \definecolor{ansi-black}{HTML}{3E424D}
    \definecolor{ansi-black-intense}{HTML}{282C36}
    \definecolor{ansi-red}{HTML}{E75C58}
    \definecolor{ansi-red-intense}{HTML}{B22B31}
    \definecolor{ansi-green}{HTML}{00A250}
    \definecolor{ansi-green-intense}{HTML}{007427}
    \definecolor{ansi-yellow}{HTML}{DDB62B}
    \definecolor{ansi-yellow-intense}{HTML}{B27D12}
    \definecolor{ansi-blue}{HTML}{208FFB}
    \definecolor{ansi-blue-intense}{HTML}{0065CA}
    \definecolor{ansi-magenta}{HTML}{D160C4}
    \definecolor{ansi-magenta-intense}{HTML}{A03196}
    \definecolor{ansi-cyan}{HTML}{60C6C8}
    \definecolor{ansi-cyan-intense}{HTML}{258F8F}
    \definecolor{ansi-white}{HTML}{C5C1B4}
    \definecolor{ansi-white-intense}{HTML}{A1A6B2}
    \definecolor{ansi-default-inverse-fg}{HTML}{FFFFFF}
    \definecolor{ansi-default-inverse-bg}{HTML}{000000}

    % common color for the border for error outputs.
    \definecolor{outerrorbackground}{HTML}{FFDFDF}

    % commands and environments needed by pandoc snippets
    % extracted from the output of `pandoc -s`
    \providecommand{\tightlist}{%
      \setlength{\itemsep}{0pt}\setlength{\parskip}{0pt}}
    \DefineVerbatimEnvironment{Highlighting}{Verbatim}{commandchars=\\\{\}}
    % Add ',fontsize=\small' for more characters per line
    \newenvironment{Shaded}{}{}
    \newcommand{\KeywordTok}[1]{\textcolor[rgb]{0.00,0.44,0.13}{\textbf{{#1}}}}
    \newcommand{\DataTypeTok}[1]{\textcolor[rgb]{0.56,0.13,0.00}{{#1}}}
    \newcommand{\DecValTok}[1]{\textcolor[rgb]{0.25,0.63,0.44}{{#1}}}
    \newcommand{\BaseNTok}[1]{\textcolor[rgb]{0.25,0.63,0.44}{{#1}}}
    \newcommand{\FloatTok}[1]{\textcolor[rgb]{0.25,0.63,0.44}{{#1}}}
    \newcommand{\CharTok}[1]{\textcolor[rgb]{0.25,0.44,0.63}{{#1}}}
    \newcommand{\StringTok}[1]{\textcolor[rgb]{0.25,0.44,0.63}{{#1}}}
    \newcommand{\CommentTok}[1]{\textcolor[rgb]{0.38,0.63,0.69}{\textit{{#1}}}}
    \newcommand{\OtherTok}[1]{\textcolor[rgb]{0.00,0.44,0.13}{{#1}}}
    \newcommand{\AlertTok}[1]{\textcolor[rgb]{1.00,0.00,0.00}{\textbf{{#1}}}}
    \newcommand{\FunctionTok}[1]{\textcolor[rgb]{0.02,0.16,0.49}{{#1}}}
    \newcommand{\RegionMarkerTok}[1]{{#1}}
    \newcommand{\ErrorTok}[1]{\textcolor[rgb]{1.00,0.00,0.00}{\textbf{{#1}}}}
    \newcommand{\NormalTok}[1]{{#1}}

    % Additional commands for more recent versions of Pandoc
    \newcommand{\ConstantTok}[1]{\textcolor[rgb]{0.53,0.00,0.00}{{#1}}}
    \newcommand{\SpecialCharTok}[1]{\textcolor[rgb]{0.25,0.44,0.63}{{#1}}}
    \newcommand{\VerbatimStringTok}[1]{\textcolor[rgb]{0.25,0.44,0.63}{{#1}}}
    \newcommand{\SpecialStringTok}[1]{\textcolor[rgb]{0.73,0.40,0.53}{{#1}}}
    \newcommand{\ImportTok}[1]{{#1}}
    \newcommand{\DocumentationTok}[1]{\textcolor[rgb]{0.73,0.13,0.13}{\textit{{#1}}}}
    \newcommand{\AnnotationTok}[1]{\textcolor[rgb]{0.38,0.63,0.69}{\textbf{\textit{{#1}}}}}
    \newcommand{\CommentVarTok}[1]{\textcolor[rgb]{0.38,0.63,0.69}{\textbf{\textit{{#1}}}}}
    \newcommand{\VariableTok}[1]{\textcolor[rgb]{0.10,0.09,0.49}{{#1}}}
    \newcommand{\ControlFlowTok}[1]{\textcolor[rgb]{0.00,0.44,0.13}{\textbf{{#1}}}}
    \newcommand{\OperatorTok}[1]{\textcolor[rgb]{0.40,0.40,0.40}{{#1}}}
    \newcommand{\BuiltInTok}[1]{{#1}}
    \newcommand{\ExtensionTok}[1]{{#1}}
    \newcommand{\PreprocessorTok}[1]{\textcolor[rgb]{0.74,0.48,0.00}{{#1}}}
    \newcommand{\AttributeTok}[1]{\textcolor[rgb]{0.49,0.56,0.16}{{#1}}}
    \newcommand{\InformationTok}[1]{\textcolor[rgb]{0.38,0.63,0.69}{\textbf{\textit{{#1}}}}}
    \newcommand{\WarningTok}[1]{\textcolor[rgb]{0.38,0.63,0.69}{\textbf{\textit{{#1}}}}}


    % Define a nice break command that doesn't care if a line doesn't already
    % exist.
    \def\br{\hspace*{\fill} \\* }
    % Math Jax compatibility definitions
    \def\gt{>}
    \def\lt{<}
    \let\Oldtex\TeX
    \let\Oldlatex\LaTeX
    \renewcommand{\TeX}{\textrm{\Oldtex}}
    \renewcommand{\LaTeX}{\textrm{\Oldlatex}}
    % Document parameters
    % Document title
    \title{CM148\_Project1\_Final}
    
    
    
    
    
% Pygments definitions
\makeatletter
\def\PY@reset{\let\PY@it=\relax \let\PY@bf=\relax%
    \let\PY@ul=\relax \let\PY@tc=\relax%
    \let\PY@bc=\relax \let\PY@ff=\relax}
\def\PY@tok#1{\csname PY@tok@#1\endcsname}
\def\PY@toks#1+{\ifx\relax#1\empty\else%
    \PY@tok{#1}\expandafter\PY@toks\fi}
\def\PY@do#1{\PY@bc{\PY@tc{\PY@ul{%
    \PY@it{\PY@bf{\PY@ff{#1}}}}}}}
\def\PY#1#2{\PY@reset\PY@toks#1+\relax+\PY@do{#2}}

\@namedef{PY@tok@w}{\def\PY@tc##1{\textcolor[rgb]{0.73,0.73,0.73}{##1}}}
\@namedef{PY@tok@c}{\let\PY@it=\textit\def\PY@tc##1{\textcolor[rgb]{0.24,0.48,0.48}{##1}}}
\@namedef{PY@tok@cp}{\def\PY@tc##1{\textcolor[rgb]{0.61,0.40,0.00}{##1}}}
\@namedef{PY@tok@k}{\let\PY@bf=\textbf\def\PY@tc##1{\textcolor[rgb]{0.00,0.50,0.00}{##1}}}
\@namedef{PY@tok@kp}{\def\PY@tc##1{\textcolor[rgb]{0.00,0.50,0.00}{##1}}}
\@namedef{PY@tok@kt}{\def\PY@tc##1{\textcolor[rgb]{0.69,0.00,0.25}{##1}}}
\@namedef{PY@tok@o}{\def\PY@tc##1{\textcolor[rgb]{0.40,0.40,0.40}{##1}}}
\@namedef{PY@tok@ow}{\let\PY@bf=\textbf\def\PY@tc##1{\textcolor[rgb]{0.67,0.13,1.00}{##1}}}
\@namedef{PY@tok@nb}{\def\PY@tc##1{\textcolor[rgb]{0.00,0.50,0.00}{##1}}}
\@namedef{PY@tok@nf}{\def\PY@tc##1{\textcolor[rgb]{0.00,0.00,1.00}{##1}}}
\@namedef{PY@tok@nc}{\let\PY@bf=\textbf\def\PY@tc##1{\textcolor[rgb]{0.00,0.00,1.00}{##1}}}
\@namedef{PY@tok@nn}{\let\PY@bf=\textbf\def\PY@tc##1{\textcolor[rgb]{0.00,0.00,1.00}{##1}}}
\@namedef{PY@tok@ne}{\let\PY@bf=\textbf\def\PY@tc##1{\textcolor[rgb]{0.80,0.25,0.22}{##1}}}
\@namedef{PY@tok@nv}{\def\PY@tc##1{\textcolor[rgb]{0.10,0.09,0.49}{##1}}}
\@namedef{PY@tok@no}{\def\PY@tc##1{\textcolor[rgb]{0.53,0.00,0.00}{##1}}}
\@namedef{PY@tok@nl}{\def\PY@tc##1{\textcolor[rgb]{0.46,0.46,0.00}{##1}}}
\@namedef{PY@tok@ni}{\let\PY@bf=\textbf\def\PY@tc##1{\textcolor[rgb]{0.44,0.44,0.44}{##1}}}
\@namedef{PY@tok@na}{\def\PY@tc##1{\textcolor[rgb]{0.41,0.47,0.13}{##1}}}
\@namedef{PY@tok@nt}{\let\PY@bf=\textbf\def\PY@tc##1{\textcolor[rgb]{0.00,0.50,0.00}{##1}}}
\@namedef{PY@tok@nd}{\def\PY@tc##1{\textcolor[rgb]{0.67,0.13,1.00}{##1}}}
\@namedef{PY@tok@s}{\def\PY@tc##1{\textcolor[rgb]{0.73,0.13,0.13}{##1}}}
\@namedef{PY@tok@sd}{\let\PY@it=\textit\def\PY@tc##1{\textcolor[rgb]{0.73,0.13,0.13}{##1}}}
\@namedef{PY@tok@si}{\let\PY@bf=\textbf\def\PY@tc##1{\textcolor[rgb]{0.64,0.35,0.47}{##1}}}
\@namedef{PY@tok@se}{\let\PY@bf=\textbf\def\PY@tc##1{\textcolor[rgb]{0.67,0.36,0.12}{##1}}}
\@namedef{PY@tok@sr}{\def\PY@tc##1{\textcolor[rgb]{0.64,0.35,0.47}{##1}}}
\@namedef{PY@tok@ss}{\def\PY@tc##1{\textcolor[rgb]{0.10,0.09,0.49}{##1}}}
\@namedef{PY@tok@sx}{\def\PY@tc##1{\textcolor[rgb]{0.00,0.50,0.00}{##1}}}
\@namedef{PY@tok@m}{\def\PY@tc##1{\textcolor[rgb]{0.40,0.40,0.40}{##1}}}
\@namedef{PY@tok@gh}{\let\PY@bf=\textbf\def\PY@tc##1{\textcolor[rgb]{0.00,0.00,0.50}{##1}}}
\@namedef{PY@tok@gu}{\let\PY@bf=\textbf\def\PY@tc##1{\textcolor[rgb]{0.50,0.00,0.50}{##1}}}
\@namedef{PY@tok@gd}{\def\PY@tc##1{\textcolor[rgb]{0.63,0.00,0.00}{##1}}}
\@namedef{PY@tok@gi}{\def\PY@tc##1{\textcolor[rgb]{0.00,0.52,0.00}{##1}}}
\@namedef{PY@tok@gr}{\def\PY@tc##1{\textcolor[rgb]{0.89,0.00,0.00}{##1}}}
\@namedef{PY@tok@ge}{\let\PY@it=\textit}
\@namedef{PY@tok@gs}{\let\PY@bf=\textbf}
\@namedef{PY@tok@ges}{\let\PY@bf=\textbf\let\PY@it=\textit}
\@namedef{PY@tok@gp}{\let\PY@bf=\textbf\def\PY@tc##1{\textcolor[rgb]{0.00,0.00,0.50}{##1}}}
\@namedef{PY@tok@go}{\def\PY@tc##1{\textcolor[rgb]{0.44,0.44,0.44}{##1}}}
\@namedef{PY@tok@gt}{\def\PY@tc##1{\textcolor[rgb]{0.00,0.27,0.87}{##1}}}
\@namedef{PY@tok@err}{\def\PY@bc##1{{\setlength{\fboxsep}{\string -\fboxrule}\fcolorbox[rgb]{1.00,0.00,0.00}{1,1,1}{\strut ##1}}}}
\@namedef{PY@tok@kc}{\let\PY@bf=\textbf\def\PY@tc##1{\textcolor[rgb]{0.00,0.50,0.00}{##1}}}
\@namedef{PY@tok@kd}{\let\PY@bf=\textbf\def\PY@tc##1{\textcolor[rgb]{0.00,0.50,0.00}{##1}}}
\@namedef{PY@tok@kn}{\let\PY@bf=\textbf\def\PY@tc##1{\textcolor[rgb]{0.00,0.50,0.00}{##1}}}
\@namedef{PY@tok@kr}{\let\PY@bf=\textbf\def\PY@tc##1{\textcolor[rgb]{0.00,0.50,0.00}{##1}}}
\@namedef{PY@tok@bp}{\def\PY@tc##1{\textcolor[rgb]{0.00,0.50,0.00}{##1}}}
\@namedef{PY@tok@fm}{\def\PY@tc##1{\textcolor[rgb]{0.00,0.00,1.00}{##1}}}
\@namedef{PY@tok@vc}{\def\PY@tc##1{\textcolor[rgb]{0.10,0.09,0.49}{##1}}}
\@namedef{PY@tok@vg}{\def\PY@tc##1{\textcolor[rgb]{0.10,0.09,0.49}{##1}}}
\@namedef{PY@tok@vi}{\def\PY@tc##1{\textcolor[rgb]{0.10,0.09,0.49}{##1}}}
\@namedef{PY@tok@vm}{\def\PY@tc##1{\textcolor[rgb]{0.10,0.09,0.49}{##1}}}
\@namedef{PY@tok@sa}{\def\PY@tc##1{\textcolor[rgb]{0.73,0.13,0.13}{##1}}}
\@namedef{PY@tok@sb}{\def\PY@tc##1{\textcolor[rgb]{0.73,0.13,0.13}{##1}}}
\@namedef{PY@tok@sc}{\def\PY@tc##1{\textcolor[rgb]{0.73,0.13,0.13}{##1}}}
\@namedef{PY@tok@dl}{\def\PY@tc##1{\textcolor[rgb]{0.73,0.13,0.13}{##1}}}
\@namedef{PY@tok@s2}{\def\PY@tc##1{\textcolor[rgb]{0.73,0.13,0.13}{##1}}}
\@namedef{PY@tok@sh}{\def\PY@tc##1{\textcolor[rgb]{0.73,0.13,0.13}{##1}}}
\@namedef{PY@tok@s1}{\def\PY@tc##1{\textcolor[rgb]{0.73,0.13,0.13}{##1}}}
\@namedef{PY@tok@mb}{\def\PY@tc##1{\textcolor[rgb]{0.40,0.40,0.40}{##1}}}
\@namedef{PY@tok@mf}{\def\PY@tc##1{\textcolor[rgb]{0.40,0.40,0.40}{##1}}}
\@namedef{PY@tok@mh}{\def\PY@tc##1{\textcolor[rgb]{0.40,0.40,0.40}{##1}}}
\@namedef{PY@tok@mi}{\def\PY@tc##1{\textcolor[rgb]{0.40,0.40,0.40}{##1}}}
\@namedef{PY@tok@il}{\def\PY@tc##1{\textcolor[rgb]{0.40,0.40,0.40}{##1}}}
\@namedef{PY@tok@mo}{\def\PY@tc##1{\textcolor[rgb]{0.40,0.40,0.40}{##1}}}
\@namedef{PY@tok@ch}{\let\PY@it=\textit\def\PY@tc##1{\textcolor[rgb]{0.24,0.48,0.48}{##1}}}
\@namedef{PY@tok@cm}{\let\PY@it=\textit\def\PY@tc##1{\textcolor[rgb]{0.24,0.48,0.48}{##1}}}
\@namedef{PY@tok@cpf}{\let\PY@it=\textit\def\PY@tc##1{\textcolor[rgb]{0.24,0.48,0.48}{##1}}}
\@namedef{PY@tok@c1}{\let\PY@it=\textit\def\PY@tc##1{\textcolor[rgb]{0.24,0.48,0.48}{##1}}}
\@namedef{PY@tok@cs}{\let\PY@it=\textit\def\PY@tc##1{\textcolor[rgb]{0.24,0.48,0.48}{##1}}}

\def\PYZbs{\char`\\}
\def\PYZus{\char`\_}
\def\PYZob{\char`\{}
\def\PYZcb{\char`\}}
\def\PYZca{\char`\^}
\def\PYZam{\char`\&}
\def\PYZlt{\char`\<}
\def\PYZgt{\char`\>}
\def\PYZsh{\char`\#}
\def\PYZpc{\char`\%}
\def\PYZdl{\char`\$}
\def\PYZhy{\char`\-}
\def\PYZsq{\char`\'}
\def\PYZdq{\char`\"}
\def\PYZti{\char`\~}
% for compatibility with earlier versions
\def\PYZat{@}
\def\PYZlb{[}
\def\PYZrb{]}
\makeatother


    % For linebreaks inside Verbatim environment from package fancyvrb.
    \makeatletter
        \newbox\Wrappedcontinuationbox
        \newbox\Wrappedvisiblespacebox
        \newcommand*\Wrappedvisiblespace {\textcolor{red}{\textvisiblespace}}
        \newcommand*\Wrappedcontinuationsymbol {\textcolor{red}{\llap{\tiny$\m@th\hookrightarrow$}}}
        \newcommand*\Wrappedcontinuationindent {3ex }
        \newcommand*\Wrappedafterbreak {\kern\Wrappedcontinuationindent\copy\Wrappedcontinuationbox}
        % Take advantage of the already applied Pygments mark-up to insert
        % potential linebreaks for TeX processing.
        %        {, <, #, %, $, ' and ": go to next line.
        %        _, }, ^, &, >, - and ~: stay at end of broken line.
        % Use of \textquotesingle for straight quote.
        \newcommand*\Wrappedbreaksatspecials {%
            \def\PYGZus{\discretionary{\char`\_}{\Wrappedafterbreak}{\char`\_}}%
            \def\PYGZob{\discretionary{}{\Wrappedafterbreak\char`\{}{\char`\{}}%
            \def\PYGZcb{\discretionary{\char`\}}{\Wrappedafterbreak}{\char`\}}}%
            \def\PYGZca{\discretionary{\char`\^}{\Wrappedafterbreak}{\char`\^}}%
            \def\PYGZam{\discretionary{\char`\&}{\Wrappedafterbreak}{\char`\&}}%
            \def\PYGZlt{\discretionary{}{\Wrappedafterbreak\char`\<}{\char`\<}}%
            \def\PYGZgt{\discretionary{\char`\>}{\Wrappedafterbreak}{\char`\>}}%
            \def\PYGZsh{\discretionary{}{\Wrappedafterbreak\char`\#}{\char`\#}}%
            \def\PYGZpc{\discretionary{}{\Wrappedafterbreak\char`\%}{\char`\%}}%
            \def\PYGZdl{\discretionary{}{\Wrappedafterbreak\char`\$}{\char`\$}}%
            \def\PYGZhy{\discretionary{\char`\-}{\Wrappedafterbreak}{\char`\-}}%
            \def\PYGZsq{\discretionary{}{\Wrappedafterbreak\textquotesingle}{\textquotesingle}}%
            \def\PYGZdq{\discretionary{}{\Wrappedafterbreak\char`\"}{\char`\"}}%
            \def\PYGZti{\discretionary{\char`\~}{\Wrappedafterbreak}{\char`\~}}%
        }
        % Some characters . , ; ? ! / are not pygmentized.
        % This macro makes them "active" and they will insert potential linebreaks
        \newcommand*\Wrappedbreaksatpunct {%
            \lccode`\~`\.\lowercase{\def~}{\discretionary{\hbox{\char`\.}}{\Wrappedafterbreak}{\hbox{\char`\.}}}%
            \lccode`\~`\,\lowercase{\def~}{\discretionary{\hbox{\char`\,}}{\Wrappedafterbreak}{\hbox{\char`\,}}}%
            \lccode`\~`\;\lowercase{\def~}{\discretionary{\hbox{\char`\;}}{\Wrappedafterbreak}{\hbox{\char`\;}}}%
            \lccode`\~`\:\lowercase{\def~}{\discretionary{\hbox{\char`\:}}{\Wrappedafterbreak}{\hbox{\char`\:}}}%
            \lccode`\~`\?\lowercase{\def~}{\discretionary{\hbox{\char`\?}}{\Wrappedafterbreak}{\hbox{\char`\?}}}%
            \lccode`\~`\!\lowercase{\def~}{\discretionary{\hbox{\char`\!}}{\Wrappedafterbreak}{\hbox{\char`\!}}}%
            \lccode`\~`\/\lowercase{\def~}{\discretionary{\hbox{\char`\/}}{\Wrappedafterbreak}{\hbox{\char`\/}}}%
            \catcode`\.\active
            \catcode`\,\active
            \catcode`\;\active
            \catcode`\:\active
            \catcode`\?\active
            \catcode`\!\active
            \catcode`\/\active
            \lccode`\~`\~
        }
    \makeatother

    \let\OriginalVerbatim=\Verbatim
    \makeatletter
    \renewcommand{\Verbatim}[1][1]{%
        %\parskip\z@skip
        \sbox\Wrappedcontinuationbox {\Wrappedcontinuationsymbol}%
        \sbox\Wrappedvisiblespacebox {\FV@SetupFont\Wrappedvisiblespace}%
        \def\FancyVerbFormatLine ##1{\hsize\linewidth
            \vtop{\raggedright\hyphenpenalty\z@\exhyphenpenalty\z@
                \doublehyphendemerits\z@\finalhyphendemerits\z@
                \strut ##1\strut}%
        }%
        % If the linebreak is at a space, the latter will be displayed as visible
        % space at end of first line, and a continuation symbol starts next line.
        % Stretch/shrink are however usually zero for typewriter font.
        \def\FV@Space {%
            \nobreak\hskip\z@ plus\fontdimen3\font minus\fontdimen4\font
            \discretionary{\copy\Wrappedvisiblespacebox}{\Wrappedafterbreak}
            {\kern\fontdimen2\font}%
        }%

        % Allow breaks at special characters using \PYG... macros.
        \Wrappedbreaksatspecials
        % Breaks at punctuation characters . , ; ? ! and / need catcode=\active
        \OriginalVerbatim[#1,codes*=\Wrappedbreaksatpunct]%
    }
    \makeatother

    % Exact colors from NB
    \definecolor{incolor}{HTML}{303F9F}
    \definecolor{outcolor}{HTML}{D84315}
    \definecolor{cellborder}{HTML}{CFCFCF}
    \definecolor{cellbackground}{HTML}{F7F7F7}

    % prompt
    \makeatletter
    \newcommand{\boxspacing}{\kern\kvtcb@left@rule\kern\kvtcb@boxsep}
    \makeatother
    \newcommand{\prompt}[4]{
        {\ttfamily\llap{{\color{#2}[#3]:\hspace{3pt}#4}}\vspace{-\baselineskip}}
    }
    

    
    % Prevent overflowing lines due to hard-to-break entities
    \sloppy
    % Setup hyperref package
    \hypersetup{
      breaklinks=true,  % so long urls are correctly broken across lines
      colorlinks=true,
      urlcolor=urlcolor,
      linkcolor=linkcolor,
      citecolor=citecolor,
      }
    % Slightly bigger margins than the latex defaults
    
    \geometry{verbose,tmargin=1in,bmargin=1in,lmargin=1in,rmargin=1in}
    
    

\begin{document}
    
    \maketitle
    
    

    
    \begin{tcolorbox}[breakable, size=fbox, boxrule=1pt, pad at break*=1mm,colback=cellbackground, colframe=cellborder]
\prompt{In}{incolor}{ }{\boxspacing}
\begin{Verbatim}[commandchars=\\\{\}]
\PY{k+kn}{import} \PY{n+nn}{sys}
\PY{k+kn}{import} \PY{n+nn}{sklearn}
\PY{k+kn}{import} \PY{n+nn}{numpy} \PY{k}{as} \PY{n+nn}{np}
\PY{k+kn}{import} \PY{n+nn}{scipy} \PY{k}{as} \PY{n+nn}{scp}
\PY{k+kn}{import} \PY{n+nn}{pandas} \PY{k}{as} \PY{n+nn}{pd}
\PY{o}{\PYZpc{}}\PY{k}{matplotlib} inline
\PY{k+kn}{import} \PY{n+nn}{matplotlib}\PY{n+nn}{.}\PY{n+nn}{pyplot} \PY{k}{as} \PY{n+nn}{plt}
\end{Verbatim}
\end{tcolorbox}

    \begin{tcolorbox}[breakable, size=fbox, boxrule=1pt, pad at break*=1mm,colback=cellbackground, colframe=cellborder]
\prompt{In}{incolor}{ }{\boxspacing}
\begin{Verbatim}[commandchars=\\\{\}]
\PY{k+kn}{from} \PY{n+nn}{google}\PY{n+nn}{.}\PY{n+nn}{colab} \PY{k+kn}{import} \PY{n}{drive}
\PY{n}{drive}\PY{o}{.}\PY{n}{mount}\PY{p}{(}\PY{l+s+s1}{\PYZsq{}}\PY{l+s+s1}{/content/drive}\PY{l+s+s1}{\PYZsq{}}\PY{p}{)}
\end{Verbatim}
\end{tcolorbox}

    \begin{Verbatim}[commandchars=\\\{\}]
Mounted at /content/drive
    \end{Verbatim}

    \begin{tcolorbox}[breakable, size=fbox, boxrule=1pt, pad at break*=1mm,colback=cellbackground, colframe=cellborder]
\prompt{In}{incolor}{ }{\boxspacing}
\begin{Verbatim}[commandchars=\\\{\}]
\PY{k+kn}{import} \PY{n+nn}{sys}
\PY{n}{sys}\PY{o}{.}\PY{n}{path}\PY{o}{.}\PY{n}{append}\PY{p}{(}\PY{l+s+s1}{\PYZsq{}}\PY{l+s+s1}{/content/drive/MyDrive/CM148\PYZus{}Project1}\PY{l+s+s1}{\PYZsq{}}\PY{p}{)}

\PY{k+kn}{import} \PY{n+nn}{os}
\PY{k+kn}{import} \PY{n+nn}{tarfile}
\PY{k+kn}{import} \PY{n+nn}{urllib}

\PY{n}{DATASET\PYZus{}PATH} \PY{o}{=} \PY{n}{sys}\PY{o}{.}\PY{n}{path}\PY{p}{[}\PY{o}{\PYZhy{}}\PY{l+m+mi}{1}\PY{p}{]}
\end{Verbatim}
\end{tcolorbox}

    \hypertarget{todo-applying-the-end-end-ml-steps-to-a-different-dataset.}{%
\section{TODO: Applying the end-end ML steps to a different
dataset.}\label{todo-applying-the-end-end-ml-steps-to-a-different-dataset.}}

    We will apply what we've learnt to another dataset
(\href{https://www.kaggle.com/datasets/dgomonov/new-york-city-airbnb-open-data}{NYC
airbnb dataset from 2019}). We will predict airbnb price based on other
features.

Note: You do not have to use only one cell when programming your code
and can do it over multiple cells.

    \hypertarget{pts-visualizing-data}{%
\subsection{{[}50 pts{]} Visualizing Data}\label{pts-visualizing-data}}

    \hypertarget{pts-load-the-data-statistics}{%
\subsubsection{{[}10 pts{]} Load the data +
statistics}\label{pts-load-the-data-statistics}}

    \hypertarget{load-the-dataset-airbnbab_nyc_2019.csv-and-display-the-first-5-few-rows-of-the-data}{%
\paragraph{- Load the dataset: airbnb/AB\_NYC\_2019.csv and display the
first 5 few rows of the
data}\label{load-the-dataset-airbnbab_nyc_2019.csv-and-display-the-first-5-few-rows-of-the-data}}

    \begin{tcolorbox}[breakable, size=fbox, boxrule=1pt, pad at break*=1mm,colback=cellbackground, colframe=cellborder]
\prompt{In}{incolor}{ }{\boxspacing}
\begin{Verbatim}[commandchars=\\\{\}]
\PY{c+c1}{\PYZsh{}Defining a function to load dataset}
\PY{k}{def} \PY{n+nf}{load\PYZus{}airbnb\PYZus{}data}\PY{p}{(}\PY{n}{dataset\PYZus{}path}\PY{p}{)}\PY{p}{:}
 \PY{n}{csv\PYZus{}path} \PY{o}{=} \PY{n}{os}\PY{o}{.}\PY{n}{path}\PY{o}{.}\PY{n}{join}\PY{p}{(}\PY{n}{dataset\PYZus{}path}\PY{p}{,} \PY{l+s+s2}{\PYZdq{}}\PY{l+s+s2}{AB\PYZus{}NYC\PYZus{}2019.csv}\PY{l+s+s2}{\PYZdq{}}\PY{p}{)}
 \PY{k}{return} \PY{n}{pd}\PY{o}{.}\PY{n}{read\PYZus{}csv}\PY{p}{(}\PY{n}{csv\PYZus{}path}\PY{p}{)}

\PY{c+c1}{\PYZsh{}Running function to load the dataset}
\PY{n}{airbnb} \PY{o}{=} \PY{n}{load\PYZus{}airbnb\PYZus{}data}\PY{p}{(}\PY{n}{DATASET\PYZus{}PATH}\PY{p}{)}

\PY{c+c1}{\PYZsh{}Displaying first 5 rows of data}
\PY{n}{airbnb}\PY{o}{.}\PY{n}{head}\PY{p}{(}\PY{p}{)}
\end{Verbatim}
\end{tcolorbox}

            \begin{tcolorbox}[breakable, size=fbox, boxrule=.5pt, pad at break*=1mm, opacityfill=0]
\prompt{Out}{outcolor}{ }{\boxspacing}
\begin{Verbatim}[commandchars=\\\{\}]
     id                                              name  host\_id  \textbackslash{}
0  2539                Clean \& quiet apt home by the park     2787
1  2595                             Skylit Midtown Castle     2845
2  3647               THE VILLAGE OF HARLEM{\ldots}NEW YORK !     4632
3  3831                   Cozy Entire Floor of Brownstone     4869
4  5022  Entire Apt: Spacious Studio/Loft by central park     7192

     host\_name neighbourhood\_group neighbourhood  latitude  longitude  \textbackslash{}
0         John            Brooklyn    Kensington  40.64749  -73.97237
1     Jennifer           Manhattan       Midtown  40.75362  -73.98377
2    Elisabeth           Manhattan        Harlem  40.80902  -73.94190
3  LisaRoxanne            Brooklyn  Clinton Hill  40.68514  -73.95976
4        Laura           Manhattan   East Harlem  40.79851  -73.94399

         room\_type  price  minimum\_nights  number\_of\_reviews last\_review  \textbackslash{}
0     Private room    149               1                  9  2018-10-19
1  Entire home/apt    225               1                 45  2019-05-21
2     Private room    150               3                  0         NaN
3  Entire home/apt     89               1                270  2019-07-05
4  Entire home/apt     80              10                  9  2018-11-19

   reviews\_per\_month  calculated\_host\_listings\_count  availability\_365
0               0.21                               6               365
1               0.38                               2               355
2                NaN                               1               365
3               4.64                               1               194
4               0.10                               1                 0
\end{Verbatim}
\end{tcolorbox}
        
    \hypertarget{pull-up-info-on-the-data-type-for-each-of-the-data-fields.-will-any-of-these-be-problematic-feeding-into-your-model-you-may-need-to-do-a-little-research-on-this-discuss}{%
\paragraph{- Pull up info on the data type for each of the data fields.
Will any of these be problematic feeding into your model (you may need
to do a little research on this)?
Discuss:}\label{pull-up-info-on-the-data-type-for-each-of-the-data-fields.-will-any-of-these-be-problematic-feeding-into-your-model-you-may-need-to-do-a-little-research-on-this-discuss}}

    \begin{tcolorbox}[breakable, size=fbox, boxrule=1pt, pad at break*=1mm,colback=cellbackground, colframe=cellborder]
\prompt{In}{incolor}{ }{\boxspacing}
\begin{Verbatim}[commandchars=\\\{\}]
\PY{c+c1}{\PYZsh{}To display the datatype for each data field}
\PY{n}{airbnb}\PY{o}{.}\PY{n}{info}\PY{p}{(}\PY{p}{)}
\end{Verbatim}
\end{tcolorbox}

    \begin{Verbatim}[commandchars=\\\{\}]
<class 'pandas.core.frame.DataFrame'>
RangeIndex: 48895 entries, 0 to 48894
Data columns (total 16 columns):
 \#   Column                          Non-Null Count  Dtype
---  ------                          --------------  -----
 0   id                              48895 non-null  int64
 1   name                            48879 non-null  object
 2   host\_id                         48895 non-null  int64
 3   host\_name                       48874 non-null  object
 4   neighbourhood\_group             48895 non-null  object
 5   neighbourhood                   48895 non-null  object
 6   latitude                        48895 non-null  float64
 7   longitude                       48895 non-null  float64
 8   room\_type                       48895 non-null  object
 9   price                           48895 non-null  int64
 10  minimum\_nights                  48895 non-null  int64
 11  number\_of\_reviews               48895 non-null  int64
 12  last\_review                     38843 non-null  object
 13  reviews\_per\_month               38843 non-null  float64
 14  calculated\_host\_listings\_count  48895 non-null  int64
 15  availability\_365                48895 non-null  int64
dtypes: float64(3), int64(7), object(6)
memory usage: 6.0+ MB
    \end{Verbatim}

    Ans: Might be problematic to feed categorical features (with type
`object') into a linear regression model since it makes little logical
sense to try to plot host\_name vs price for a homeownership model. It
would also be difficult to try to plot neighbourhood vs price unless we
use one hot encoding.

    \hypertarget{drop-the-following-columns-name-id-host_id-host_name-last_review-and-reviews_per_month-and-display-first-5-rows}{%
\paragraph{- Drop the following columns: name, id, host\_id, host\_name,
last\_review, and reviews\_per\_month and display first 5
rows}\label{drop-the-following-columns-name-id-host_id-host_name-last_review-and-reviews_per_month-and-display-first-5-rows}}

    \begin{tcolorbox}[breakable, size=fbox, boxrule=1pt, pad at break*=1mm,colback=cellbackground, colframe=cellborder]
\prompt{In}{incolor}{ }{\boxspacing}
\begin{Verbatim}[commandchars=\\\{\}]
\PY{c+c1}{\PYZsh{}Dropping the specified columns}
\PY{n}{attributes} \PY{o}{=} \PY{p}{[}
    \PY{l+s+s2}{\PYZdq{}}\PY{l+s+s2}{name}\PY{l+s+s2}{\PYZdq{}}\PY{p}{,}
    \PY{l+s+s2}{\PYZdq{}}\PY{l+s+s2}{id}\PY{l+s+s2}{\PYZdq{}}\PY{p}{,}
    \PY{l+s+s2}{\PYZdq{}}\PY{l+s+s2}{host\PYZus{}id}\PY{l+s+s2}{\PYZdq{}}\PY{p}{,}
    \PY{l+s+s2}{\PYZdq{}}\PY{l+s+s2}{host\PYZus{}name}\PY{l+s+s2}{\PYZdq{}}\PY{p}{,}
    \PY{l+s+s2}{\PYZdq{}}\PY{l+s+s2}{last\PYZus{}review}\PY{l+s+s2}{\PYZdq{}}\PY{p}{,}
    \PY{l+s+s2}{\PYZdq{}}\PY{l+s+s2}{reviews\PYZus{}per\PYZus{}month}\PY{l+s+s2}{\PYZdq{}}\PY{p}{,}
\PY{p}{]}

\PY{n}{airbnb} \PY{o}{=} \PY{n}{airbnb}\PY{o}{.}\PY{n}{drop}\PY{p}{(}\PY{n}{attributes}\PY{p}{,} \PY{n}{axis}\PY{o}{=}\PY{l+m+mi}{1}\PY{p}{)}

\PY{c+c1}{\PYZsh{}Displaying the first 5 rows of dropped dataset}
\PY{n}{airbnb}\PY{o}{.}\PY{n}{head}\PY{p}{(}\PY{p}{)}
\end{Verbatim}
\end{tcolorbox}

            \begin{tcolorbox}[breakable, size=fbox, boxrule=.5pt, pad at break*=1mm, opacityfill=0]
\prompt{Out}{outcolor}{ }{\boxspacing}
\begin{Verbatim}[commandchars=\\\{\}]
  neighbourhood\_group neighbourhood  latitude  longitude        room\_type  \textbackslash{}
0            Brooklyn    Kensington  40.64749  -73.97237     Private room
1           Manhattan       Midtown  40.75362  -73.98377  Entire home/apt
2           Manhattan        Harlem  40.80902  -73.94190     Private room
3            Brooklyn  Clinton Hill  40.68514  -73.95976  Entire home/apt
4           Manhattan   East Harlem  40.79851  -73.94399  Entire home/apt

   price  minimum\_nights  number\_of\_reviews  calculated\_host\_listings\_count  \textbackslash{}
0    149               1                  9                               6
1    225               1                 45                               2
2    150               3                  0                               1
3     89               1                270                               1
4     80              10                  9                               1

   availability\_365
0               365
1               355
2               365
3               194
4                 0
\end{Verbatim}
\end{tcolorbox}
        
    \hypertarget{display-a-summary-of-the-statistics-of-the-loaded-data-using-.describe}{%
\paragraph{- Display a summary of the statistics of the loaded data
using
.describe}\label{display-a-summary-of-the-statistics-of-the-loaded-data-using-.describe}}

    \begin{tcolorbox}[breakable, size=fbox, boxrule=1pt, pad at break*=1mm,colback=cellbackground, colframe=cellborder]
\prompt{In}{incolor}{ }{\boxspacing}
\begin{Verbatim}[commandchars=\\\{\}]
\PY{c+c1}{\PYZsh{}Displaying summary statistics of original airbnb loaded dataset}
\PY{n}{airbnb}\PY{o}{.}\PY{n}{describe}\PY{p}{(}\PY{p}{)}
\end{Verbatim}
\end{tcolorbox}

            \begin{tcolorbox}[breakable, size=fbox, boxrule=.5pt, pad at break*=1mm, opacityfill=0]
\prompt{Out}{outcolor}{ }{\boxspacing}
\begin{Verbatim}[commandchars=\\\{\}]
           latitude     longitude         price  minimum\_nights  \textbackslash{}
count  48895.000000  48895.000000  48895.000000    48895.000000
mean      40.728949    -73.952170    152.720687        7.029962
std        0.054530      0.046157    240.154170       20.510550
min       40.499790    -74.244420      0.000000        1.000000
25\%       40.690100    -73.983070     69.000000        1.000000
50\%       40.723070    -73.955680    106.000000        3.000000
75\%       40.763115    -73.936275    175.000000        5.000000
max       40.913060    -73.712990  10000.000000     1250.000000

       number\_of\_reviews  calculated\_host\_listings\_count  availability\_365
count       48895.000000                    48895.000000      48895.000000
mean           23.274466                        7.143982        112.781327
std            44.550582                       32.952519        131.622289
min             0.000000                        1.000000          0.000000
25\%             1.000000                        1.000000          0.000000
50\%             5.000000                        1.000000         45.000000
75\%            24.000000                        2.000000        227.000000
max           629.000000                      327.000000        365.000000
\end{Verbatim}
\end{tcolorbox}
        
    \hypertarget{pts-plot-boxplots-for-the-following-3-features-availability_365-number_of_reviews-price}{%
\subsubsection{\texorpdfstring{{[}10 pts{]} Plot
\href{https://en.wikipedia.org/wiki/Box_plot}{boxplots} for the
following 3 features: availability\_365, number\_of\_reviews,
price}{{[}10 pts{]} Plot boxplots for the following 3 features: availability\_365, number\_of\_reviews, price}}\label{pts-plot-boxplots-for-the-following-3-features-availability_365-number_of_reviews-price}}

You may use either pandas or matplotlib to plot the boxplot

    \begin{tcolorbox}[breakable, size=fbox, boxrule=1pt, pad at break*=1mm,colback=cellbackground, colframe=cellborder]
\prompt{In}{incolor}{ }{\boxspacing}
\begin{Verbatim}[commandchars=\\\{\}]
\PY{n}{features} \PY{o}{=} \PY{p}{[}\PY{l+s+s2}{\PYZdq{}}\PY{l+s+s2}{availability\PYZus{}365}\PY{l+s+s2}{\PYZdq{}}\PY{p}{,} \PY{l+s+s2}{\PYZdq{}}\PY{l+s+s2}{number\PYZus{}of\PYZus{}reviews}\PY{l+s+s2}{\PYZdq{}}\PY{p}{,} \PY{l+s+s2}{\PYZdq{}}\PY{l+s+s2}{price}\PY{l+s+s2}{\PYZdq{}}\PY{p}{]}
\PY{k}{for} \PY{n}{f} \PY{o+ow}{in} \PY{n}{features}\PY{p}{:}
  \PY{n}{airbnb}\PY{p}{[}\PY{p}{[}\PY{n}{f}\PY{p}{]}\PY{p}{]}\PY{o}{.}\PY{n}{boxplot}\PY{p}{(}\PY{p}{)}
  \PY{n}{plt}\PY{o}{.}\PY{n}{title}\PY{p}{(}\PY{l+s+sa}{f}\PY{l+s+s2}{\PYZdq{}}\PY{l+s+s2}{Boxplot of }\PY{l+s+si}{\PYZob{}}\PY{n}{f}\PY{l+s+si}{\PYZcb{}}\PY{l+s+s2}{\PYZdq{}}\PY{p}{)}
  \PY{n}{plt}\PY{o}{.}\PY{n}{ylabel}\PY{p}{(}\PY{l+s+s2}{\PYZdq{}}\PY{l+s+s2}{Values}\PY{l+s+s2}{\PYZdq{}}\PY{p}{)}
  \PY{n}{plt}\PY{o}{.}\PY{n}{show}\PY{p}{(}\PY{p}{)}
  \PY{n+nb}{print}\PY{p}{(}\PY{l+s+s2}{\PYZdq{}}\PY{l+s+s2}{             }\PY{l+s+s2}{\PYZdq{}}\PY{p}{)}
\end{Verbatim}
\end{tcolorbox}

    \begin{center}
    \adjustimage{max size={0.9\linewidth}{0.9\paperheight}}{CM148_Project1_Final_files/CM148_Project1_Final_17_0.png}
    \end{center}
    { \hspace*{\fill} \\}
    
    \begin{Verbatim}[commandchars=\\\{\}]

    \end{Verbatim}

    \begin{center}
    \adjustimage{max size={0.9\linewidth}{0.9\paperheight}}{CM148_Project1_Final_files/CM148_Project1_Final_17_2.png}
    \end{center}
    { \hspace*{\fill} \\}
    
    \begin{Verbatim}[commandchars=\\\{\}]

    \end{Verbatim}

    \begin{center}
    \adjustimage{max size={0.9\linewidth}{0.9\paperheight}}{CM148_Project1_Final_files/CM148_Project1_Final_17_4.png}
    \end{center}
    { \hspace*{\fill} \\}
    
    \begin{Verbatim}[commandchars=\\\{\}]

    \end{Verbatim}

    \hypertarget{what-do-you-observe-from-the-boxplot-about-the-features-anything-suprising}{%
\paragraph{- What do you observe from the boxplot about the features?
Anything
suprising?}\label{what-do-you-observe-from-the-boxplot-about-the-features-anything-suprising}}

    Ans: I was pretty surprised that the number\_of\_reviews had so many
points classified as outliers, but that makes some sense since there are
no bounds on the number of reviews, and similarly no upper bounds on the
price range because in the real world, price can be 20k, 30k etc even
though the highest value in this dataset is 10k. But for the number of
days available, the values can range only between 0 and 365 so there
were no outliers for this box plot as expected.

    \hypertarget{pts-plot-median-price-of-a-listing-per-neighbourhood_group-using-a-bar-plot}{%
\subsubsection{{[}10 pts{]} Plot median price of a listing per
neighbourhood\_group using a bar
plot}\label{pts-plot-median-price-of-a-listing-per-neighbourhood_group-using-a-bar-plot}}

    \begin{tcolorbox}[breakable, size=fbox, boxrule=1pt, pad at break*=1mm,colback=cellbackground, colframe=cellborder]
\prompt{In}{incolor}{ }{\boxspacing}
\begin{Verbatim}[commandchars=\\\{\}]
\PY{c+c1}{\PYZsh{}Grouping data by neighbourhood\PYZus{}group}
\PY{n}{neighbourhood\PYZus{}grouping} \PY{o}{=} \PY{n}{airbnb}\PY{o}{.}\PY{n}{groupby}\PY{p}{(}\PY{l+s+s2}{\PYZdq{}}\PY{l+s+s2}{neighbourhood\PYZus{}group}\PY{l+s+s2}{\PYZdq{}}\PY{p}{)}

\PY{c+c1}{\PYZsh{}Computing median prices}
\PY{n}{median\PYZus{}prices} \PY{o}{=} \PY{n}{neighbourhood\PYZus{}grouping}\PY{p}{[}\PY{l+s+s1}{\PYZsq{}}\PY{l+s+s1}{price}\PY{l+s+s1}{\PYZsq{}}\PY{p}{]}\PY{o}{.}\PY{n}{median}\PY{p}{(}\PY{p}{)}

\PY{c+c1}{\PYZsh{}Plotting bar plot}
\PY{n}{dataframe} \PY{o}{=} \PY{n}{pd}\PY{o}{.}\PY{n}{Series}\PY{p}{(}\PY{n}{median\PYZus{}prices}\PY{p}{)}
\PY{n}{dataframe}\PY{o}{.}\PY{n}{plot}\PY{p}{(}\PY{n}{kind}\PY{o}{=}\PY{l+s+s1}{\PYZsq{}}\PY{l+s+s1}{bar}\PY{l+s+s1}{\PYZsq{}}\PY{p}{)}
\PY{n}{plt}\PY{o}{.}\PY{n}{title}\PY{p}{(}\PY{l+s+s2}{\PYZdq{}}\PY{l+s+s2}{Median prices per neighbourhood group}\PY{l+s+s2}{\PYZdq{}}\PY{p}{)}
\PY{n}{plt}\PY{o}{.}\PY{n}{xlabel}\PY{p}{(}\PY{l+s+s2}{\PYZdq{}}\PY{l+s+s2}{Neighbourhood Group}\PY{l+s+s2}{\PYZdq{}}\PY{p}{)}
\PY{n}{plt}\PY{o}{.}\PY{n}{ylabel}\PY{p}{(}\PY{l+s+s2}{\PYZdq{}}\PY{l+s+s2}{Median Price}\PY{l+s+s2}{\PYZdq{}}\PY{p}{)}
\end{Verbatim}
\end{tcolorbox}

            \begin{tcolorbox}[breakable, size=fbox, boxrule=.5pt, pad at break*=1mm, opacityfill=0]
\prompt{Out}{outcolor}{ }{\boxspacing}
\begin{Verbatim}[commandchars=\\\{\}]
Text(0, 0.5, 'Median Price')
\end{Verbatim}
\end{tcolorbox}
        
    \begin{center}
    \adjustimage{max size={0.9\linewidth}{0.9\paperheight}}{CM148_Project1_Final_files/CM148_Project1_Final_21_1.png}
    \end{center}
    { \hspace*{\fill} \\}
    
    \hypertarget{describe-what-you-expected-to-see-with-these-features-and-what-you-actually-observed}{%
\paragraph{- Describe what you expected to see with these features and
what you actually
observed}\label{describe-what-you-expected-to-see-with-these-features-and-what-you-actually-observed}}

    Ans: I expected some affluent neighbourhoods to have significantly
higher median prices than the less affluent ones, and this was supported
by the data. Manhattan is one of the most central and prominent places
to live, so it makes sense that the prices here would be significantly
higher than other neighbourhoods like Queens or Brooklyn.

    \hypertarget{so-we-can-see-different-neighborhoods-have-dramatically-different-pricepoints-but-how-does-the-price-breakdown-by-range.-to-see-lets-do-a-histogram-of-price-by-neighborhood-to-get-a-better-sense-of-the-distribution.}{%
\paragraph{- So we can see different neighborhoods have dramatically
different pricepoints, but how does the price breakdown by range. To see
let's do a histogram of price by neighborhood to get a better sense of
the
distribution.}\label{so-we-can-see-different-neighborhoods-have-dramatically-different-pricepoints-but-how-does-the-price-breakdown-by-range.-to-see-lets-do-a-histogram-of-price-by-neighborhood-to-get-a-better-sense-of-the-distribution.}}

To prevent outliers from affecting the histogram, use the input
\emph{range = {[}0,300{]}} in the histogram function which will
upperbound the max price to 300 and ignore the outliers.

    \begin{tcolorbox}[breakable, size=fbox, boxrule=1pt, pad at break*=1mm,colback=cellbackground, colframe=cellborder]
\prompt{In}{incolor}{ }{\boxspacing}
\begin{Verbatim}[commandchars=\\\{\}]
\PY{n}{airbnb}\PY{o}{.}\PY{n}{hist}\PY{p}{(}\PY{n}{column}\PY{o}{=}\PY{l+s+s1}{\PYZsq{}}\PY{l+s+s1}{price}\PY{l+s+s1}{\PYZsq{}}\PY{p}{,} \PY{n}{by}\PY{o}{=}\PY{l+s+s1}{\PYZsq{}}\PY{l+s+s1}{neighbourhood\PYZus{}group}\PY{l+s+s1}{\PYZsq{}}\PY{p}{,} \PY{n+nb}{range}\PY{o}{=}\PY{p}{[}\PY{l+m+mi}{0}\PY{p}{,}\PY{l+m+mi}{300}\PY{p}{]}\PY{p}{,}\PY{n}{legend}\PY{o}{=}\PY{k+kc}{True}\PY{p}{,} \PY{n}{figsize}\PY{o}{=}\PY{p}{(}\PY{l+m+mi}{8}\PY{p}{,}\PY{l+m+mi}{8}\PY{p}{)}\PY{p}{)}
\PY{n}{plt}\PY{o}{.}\PY{n}{title}\PY{p}{(}\PY{l+s+s2}{\PYZdq{}}\PY{l+s+s2}{Histogram: Price vs Neighbourhood}\PY{l+s+s2}{\PYZdq{}}\PY{p}{)}
\PY{n}{plt}\PY{o}{.}\PY{n}{xlabel}\PY{p}{(}\PY{l+s+s2}{\PYZdq{}}\PY{l+s+s2}{Price}\PY{l+s+s2}{\PYZdq{}}\PY{p}{)}
\PY{n}{plt}\PY{o}{.}\PY{n}{ylabel}\PY{p}{(}\PY{l+s+s2}{\PYZdq{}}\PY{l+s+s2}{Frequency}\PY{l+s+s2}{\PYZdq{}}\PY{p}{)}
\PY{n}{plt}\PY{o}{.}\PY{n}{show}\PY{p}{(}\PY{p}{)}
\end{Verbatim}
\end{tcolorbox}

    \begin{center}
    \adjustimage{max size={0.9\linewidth}{0.9\paperheight}}{CM148_Project1_Final_files/CM148_Project1_Final_25_0.png}
    \end{center}
    { \hspace*{\fill} \\}
    
    \hypertarget{pts-plot-a-map-of-airbnbs-throughout-new-york.-you-do-not-need-to-overlay-a-map.}{%
\subsubsection{{[}5 pts{]} Plot a map of airbnbs throughout New York.
You do not need to overlay a
map.}\label{pts-plot-a-map-of-airbnbs-throughout-new-york.-you-do-not-need-to-overlay-a-map.}}

    \begin{tcolorbox}[breakable, size=fbox, boxrule=1pt, pad at break*=1mm,colback=cellbackground, colframe=cellborder]
\prompt{In}{incolor}{ }{\boxspacing}
\begin{Verbatim}[commandchars=\\\{\}]
\PY{n}{airbnb}\PY{o}{.}\PY{n}{plot}\PY{p}{(}\PY{n}{kind}\PY{o}{=}\PY{l+s+s1}{\PYZsq{}}\PY{l+s+s1}{scatter}\PY{l+s+s1}{\PYZsq{}}\PY{p}{,}\PY{n}{x}\PY{o}{=}\PY{l+s+s1}{\PYZsq{}}\PY{l+s+s1}{latitude}\PY{l+s+s1}{\PYZsq{}}\PY{p}{,} \PY{n}{y}\PY{o}{=}\PY{l+s+s1}{\PYZsq{}}\PY{l+s+s1}{longitude}\PY{l+s+s1}{\PYZsq{}}\PY{p}{)}
\PY{n}{plt}\PY{o}{.}\PY{n}{title}\PY{p}{(}\PY{l+s+s2}{\PYZdq{}}\PY{l+s+s2}{Airbnbs throughout New York}\PY{l+s+s2}{\PYZdq{}}\PY{p}{)}
\end{Verbatim}
\end{tcolorbox}

            \begin{tcolorbox}[breakable, size=fbox, boxrule=.5pt, pad at break*=1mm, opacityfill=0]
\prompt{Out}{outcolor}{ }{\boxspacing}
\begin{Verbatim}[commandchars=\\\{\}]
Text(0.5, 1.0, 'Airbnbs throughout New York')
\end{Verbatim}
\end{tcolorbox}
        
    \begin{center}
    \adjustimage{max size={0.9\linewidth}{0.9\paperheight}}{CM148_Project1_Final_files/CM148_Project1_Final_27_1.png}
    \end{center}
    { \hspace*{\fill} \\}
    
    \hypertarget{pts-plot-median-price-of-room-types-who-have-availability-greater-than-180-days-and-neighbourhood_group-is-manhattan}{%
\subsubsection{{[}10 pts{]} Plot median price of room types who have
availability greater than 180 days and neighbourhood\_group is
Manhattan}\label{pts-plot-median-price-of-room-types-who-have-availability-greater-than-180-days-and-neighbourhood_group-is-manhattan}}

    \begin{tcolorbox}[breakable, size=fbox, boxrule=1pt, pad at break*=1mm,colback=cellbackground, colframe=cellborder]
\prompt{In}{incolor}{ }{\boxspacing}
\begin{Verbatim}[commandchars=\\\{\}]
\PY{n}{manhattan\PYZus{}houses} \PY{o}{=} \PY{n}{neighbourhood\PYZus{}grouping}\PY{o}{.}\PY{n}{get\PYZus{}group}\PY{p}{(}\PY{l+s+s2}{\PYZdq{}}\PY{l+s+s2}{Manhattan}\PY{l+s+s2}{\PYZdq{}}\PY{p}{)}
\PY{n}{to\PYZus{}plot} \PY{o}{=} \PY{n}{manhattan\PYZus{}houses}\PY{p}{[}\PY{p}{(}\PY{n}{manhattan\PYZus{}houses}\PY{p}{[}\PY{l+s+s1}{\PYZsq{}}\PY{l+s+s1}{availability\PYZus{}365}\PY{l+s+s1}{\PYZsq{}}\PY{p}{]}\PY{o}{\PYZgt{}}\PY{l+m+mi}{180}\PY{p}{)}\PY{p}{]}
\PY{n}{to\PYZus{}plot} \PY{o}{=} \PY{n}{to\PYZus{}plot}\PY{o}{.}\PY{n}{groupby}\PY{p}{(}\PY{l+s+s1}{\PYZsq{}}\PY{l+s+s1}{room\PYZus{}type}\PY{l+s+s1}{\PYZsq{}}\PY{p}{)}
\PY{n}{to\PYZus{}plot} \PY{o}{=} \PY{n}{to\PYZus{}plot}\PY{p}{[}\PY{l+s+s1}{\PYZsq{}}\PY{l+s+s1}{price}\PY{l+s+s1}{\PYZsq{}}\PY{p}{]}\PY{o}{.}\PY{n}{median}\PY{p}{(}\PY{p}{)}

\PY{n}{dataframe} \PY{o}{=} \PY{n}{pd}\PY{o}{.}\PY{n}{Series}\PY{p}{(}\PY{n}{to\PYZus{}plot}\PY{p}{)}
\PY{n}{dataframe}\PY{o}{.}\PY{n}{plot}\PY{p}{(}\PY{n}{kind}\PY{o}{=}\PY{l+s+s1}{\PYZsq{}}\PY{l+s+s1}{bar}\PY{l+s+s1}{\PYZsq{}}\PY{p}{)}
\PY{n}{plt}\PY{o}{.}\PY{n}{title}\PY{p}{(}\PY{l+s+s2}{\PYZdq{}}\PY{l+s+s2}{Median prices per room type for Manhattan apartments with availability \PYZgt{} 180 days}\PY{l+s+s2}{\PYZdq{}}\PY{p}{)}
\PY{n}{plt}\PY{o}{.}\PY{n}{xlabel}\PY{p}{(}\PY{l+s+s2}{\PYZdq{}}\PY{l+s+s2}{Room type}\PY{l+s+s2}{\PYZdq{}}\PY{p}{)}
\PY{n}{plt}\PY{o}{.}\PY{n}{ylabel}\PY{p}{(}\PY{l+s+s2}{\PYZdq{}}\PY{l+s+s2}{Median Price}\PY{l+s+s2}{\PYZdq{}}\PY{p}{)}
\end{Verbatim}
\end{tcolorbox}

            \begin{tcolorbox}[breakable, size=fbox, boxrule=.5pt, pad at break*=1mm, opacityfill=0]
\prompt{Out}{outcolor}{ }{\boxspacing}
\begin{Verbatim}[commandchars=\\\{\}]
Text(0, 0.5, 'Median Price')
\end{Verbatim}
\end{tcolorbox}
        
    \begin{center}
    \adjustimage{max size={0.9\linewidth}{0.9\paperheight}}{CM148_Project1_Final_files/CM148_Project1_Final_29_1.png}
    \end{center}
    { \hspace*{\fill} \\}
    
    \hypertarget{pts-find-features-that-correlate-with-price}{%
\subsubsection{{[}5 pts{]} Find features that correlate with
price}\label{pts-find-features-that-correlate-with-price}}

Using the correlation matrix: - which features have positive correlation
with the price? - which features have negative correlation with the
price?

    \begin{tcolorbox}[breakable, size=fbox, boxrule=1pt, pad at break*=1mm,colback=cellbackground, colframe=cellborder]
\prompt{In}{incolor}{ }{\boxspacing}
\begin{Verbatim}[commandchars=\\\{\}]
\PY{n}{corr\PYZus{}matrix} \PY{o}{=} \PY{n}{airbnb}\PY{o}{.}\PY{n}{corr}\PY{p}{(}\PY{n}{numeric\PYZus{}only}\PY{o}{=}\PY{k+kc}{True}\PY{p}{)}
\PY{n}{corr\PYZus{}matrix}\PY{p}{[}\PY{l+s+s2}{\PYZdq{}}\PY{l+s+s2}{price}\PY{l+s+s2}{\PYZdq{}}\PY{p}{]}\PY{o}{.}\PY{n}{sort\PYZus{}values}\PY{p}{(}\PY{n}{ascending}\PY{o}{=}\PY{k+kc}{False}\PY{p}{)}
\end{Verbatim}
\end{tcolorbox}

            \begin{tcolorbox}[breakable, size=fbox, boxrule=.5pt, pad at break*=1mm, opacityfill=0]
\prompt{Out}{outcolor}{ }{\boxspacing}
\begin{Verbatim}[commandchars=\\\{\}]
price                             1.000000
availability\_365                  0.081829
calculated\_host\_listings\_count    0.057472
minimum\_nights                    0.042799
latitude                          0.033939
number\_of\_reviews                -0.047954
longitude                        -0.150019
Name: price, dtype: float64
\end{Verbatim}
\end{tcolorbox}
        
    Positive correlation with price:
\texttt{availability\_365,\ calculated\_host\_listings\_count,\ minimum\_nights,\ latitude}

Negative correlation with price:
\texttt{number\_of\_reviews,\ longitude}

    \hypertarget{plot-the-full-scatter-matrix-to-see-the-correlation-between-prices-and-the-other-features}{%
\paragraph{- Plot the full Scatter Matrix to see the correlation between
prices and the other
features}\label{plot-the-full-scatter-matrix-to-see-the-correlation-between-prices-and-the-other-features}}

    \begin{tcolorbox}[breakable, size=fbox, boxrule=1pt, pad at break*=1mm,colback=cellbackground, colframe=cellborder]
\prompt{In}{incolor}{ }{\boxspacing}
\begin{Verbatim}[commandchars=\\\{\}]
\PY{k+kn}{from} \PY{n+nn}{pandas}\PY{n+nn}{.}\PY{n+nn}{plotting} \PY{k+kn}{import} \PY{n}{scatter\PYZus{}matrix}
\PY{n}{scatter} \PY{o}{=} \PY{n}{scatter\PYZus{}matrix}\PY{p}{(}\PY{n}{airbnb}\PY{p}{,} \PY{n}{figsize}\PY{o}{=}\PY{p}{(}\PY{l+m+mi}{12}\PY{p}{,}\PY{l+m+mi}{19}\PY{p}{)}\PY{p}{)}
\end{Verbatim}
\end{tcolorbox}

    \begin{center}
    \adjustimage{max size={0.9\linewidth}{0.9\paperheight}}{CM148_Project1_Final_files/CM148_Project1_Final_34_0.png}
    \end{center}
    { \hspace*{\fill} \\}
    
    \hypertarget{pts-prepare-the-data}{%
\subsection{{[}30 pts{]} Prepare the Data}\label{pts-prepare-the-data}}

    \hypertarget{pts-partition-the-data-into-the-features-and-the-target-data.-the-target-data-is-price.-then-partition-the-feature-data-into-categorical-and-numerical-features.}{%
\subsubsection{{[}5 pts{]} Partition the data into the features and the
target data. The target data is price. Then partition the feature data
into categorical and numerical
features.}\label{pts-partition-the-data-into-the-features-and-the-target-data.-the-target-data-is-price.-then-partition-the-feature-data-into-categorical-and-numerical-features.}}

    \begin{tcolorbox}[breakable, size=fbox, boxrule=1pt, pad at break*=1mm,colback=cellbackground, colframe=cellborder]
\prompt{In}{incolor}{ }{\boxspacing}
\begin{Verbatim}[commandchars=\\\{\}]
\PY{n}{price\PYZus{}target} \PY{o}{=} \PY{n}{airbnb}\PY{p}{[}\PY{l+s+s1}{\PYZsq{}}\PY{l+s+s1}{price}\PY{l+s+s1}{\PYZsq{}}\PY{p}{]} \PY{c+c1}{\PYZsh{}target}
\PY{n}{feature\PYZus{}data} \PY{o}{=} \PY{n}{airbnb}\PY{o}{.}\PY{n}{drop}\PY{p}{(}\PY{n}{columns}\PY{o}{=}\PY{l+s+s1}{\PYZsq{}}\PY{l+s+s1}{price}\PY{l+s+s1}{\PYZsq{}}\PY{p}{)} \PY{c+c1}{\PYZsh{}feature data}

\PY{n}{categorical\PYZus{}features} \PY{o}{=} \PY{n}{feature\PYZus{}data}\PY{o}{.}\PY{n}{select\PYZus{}dtypes}\PY{p}{(}\PY{n}{include}\PY{o}{=}\PY{p}{[}\PY{l+s+s1}{\PYZsq{}}\PY{l+s+s1}{object}\PY{l+s+s1}{\PYZsq{}}\PY{p}{]}\PY{p}{)}
\PY{n}{numerical\PYZus{}features} \PY{o}{=} \PY{n}{feature\PYZus{}data}\PY{o}{.}\PY{n}{select\PYZus{}dtypes}\PY{p}{(}\PY{n}{exclude}\PY{o}{=}\PY{p}{[}\PY{l+s+s1}{\PYZsq{}}\PY{l+s+s1}{object}\PY{l+s+s1}{\PYZsq{}}\PY{p}{]}\PY{p}{)}
\end{Verbatim}
\end{tcolorbox}

    \hypertarget{pts-create-a-scikit-learn-transformer-that-augments-the-numerical-data-with-the-following-two-features}{%
\subsubsection{{[}10 pts{]} Create a scikit learn Transformer that
augments the numerical data with the following two
features}\label{pts-create-a-scikit-learn-transformer-that-augments-the-numerical-data-with-the-following-two-features}}

\begin{itemize}
\item
  Max\_yearly\_bookings = availability\_365 / minimum\_nights
\item
  Distance from airbnb to the NYC JFK Airport

  \begin{itemize}
  \tightlist
  \item
    Latitude: 40.641766 , Longitude: -73.780968
  \end{itemize}
\end{itemize}

Make sure to append these new features in this order.

You may use the previously defined distance\_func for the distance
calculation.

Note that this Transformer will be applied after imputation so we do not
have to worry about Nulls in the data.

    \begin{tcolorbox}[breakable, size=fbox, boxrule=1pt, pad at break*=1mm,colback=cellbackground, colframe=cellborder]
\prompt{In}{incolor}{ }{\boxspacing}
\begin{Verbatim}[commandchars=\\\{\}]
\PY{k+kn}{from} \PY{n+nn}{sklearn}\PY{n+nn}{.}\PY{n+nn}{impute} \PY{k+kn}{import} \PY{n}{SimpleImputer}
\PY{k+kn}{from} \PY{n+nn}{sklearn}\PY{n+nn}{.}\PY{n+nn}{compose} \PY{k+kn}{import} \PY{n}{ColumnTransformer}
\PY{k+kn}{from} \PY{n+nn}{sklearn}\PY{n+nn}{.}\PY{n+nn}{pipeline} \PY{k+kn}{import} \PY{n}{Pipeline}
\PY{k+kn}{from} \PY{n+nn}{sklearn}\PY{n+nn}{.}\PY{n+nn}{preprocessing} \PY{k+kn}{import} \PY{n}{StandardScaler}
\PY{k+kn}{from} \PY{n+nn}{sklearn}\PY{n+nn}{.}\PY{n+nn}{preprocessing} \PY{k+kn}{import} \PY{n}{OneHotEncoder}
\PY{k+kn}{from} \PY{n+nn}{sklearn}\PY{n+nn}{.}\PY{n+nn}{base} \PY{k+kn}{import} \PY{n}{BaseEstimator}\PY{p}{,} \PY{n}{TransformerMixin}

\PY{k}{class} \PY{n+nc}{AugmentFeatures}\PY{p}{(}\PY{n}{BaseEstimator}\PY{p}{,} \PY{n}{TransformerMixin}\PY{p}{)}\PY{p}{:}
  \PY{k}{def} \PY{n+nf+fm}{\PYZus{}\PYZus{}init\PYZus{}\PYZus{}}\PY{p}{(}\PY{n+nb+bp}{self}\PY{p}{)}\PY{p}{:}
    \PY{k}{return} \PY{k+kc}{None}

  \PY{k}{def} \PY{n+nf}{fit}\PY{p}{(}\PY{n+nb+bp}{self}\PY{p}{,} \PY{n}{X}\PY{p}{)}\PY{p}{:}
    \PY{k}{return} \PY{n+nb+bp}{self}

  \PY{k}{def} \PY{n+nf}{transform}\PY{p}{(}\PY{n+nb+bp}{self}\PY{p}{,}\PY{n}{X}\PY{p}{)}\PY{p}{:}

    \PY{n}{availability\PYZus{}365} \PY{o}{=} \PY{l+m+mi}{5}
    \PY{n}{minimum\PYZus{}nights} \PY{o}{=} \PY{l+m+mi}{2}
    \PY{n}{latitude} \PY{o}{=} \PY{l+m+mi}{0}
    \PY{n}{longitude} \PY{o}{=} \PY{l+m+mi}{1}

    \PY{n}{max\PYZus{}yearly\PYZus{}bookings} \PY{o}{=} \PY{n}{X}\PY{p}{[}\PY{p}{:}\PY{p}{,}\PY{n}{availability\PYZus{}365}\PY{p}{]}\PY{o}{/}\PY{n}{X}\PY{p}{[}\PY{p}{:}\PY{p}{,}\PY{n}{minimum\PYZus{}nights}\PY{p}{]}
    \PY{n}{distance\PYZus{}to\PYZus{}JFK} \PY{o}{=} \PY{n+nb+bp}{self}\PY{o}{.}\PY{n}{distance\PYZus{}func}\PY{p}{(}\PY{n}{X}\PY{p}{[}\PY{p}{:}\PY{p}{,}\PY{n}{latitude}\PY{p}{]}\PY{p}{,} \PY{n}{X}\PY{p}{[}\PY{p}{:}\PY{p}{,}\PY{n}{longitude}\PY{p}{]}\PY{p}{)}
    \PY{k}{return} \PY{n}{np}\PY{o}{.}\PY{n}{c\PYZus{}}\PY{p}{[}\PY{n}{X}\PY{p}{,} \PY{n}{max\PYZus{}yearly\PYZus{}bookings}\PY{p}{,} \PY{n}{distance\PYZus{}to\PYZus{}JFK}\PY{p}{]}

  \PY{k}{def} \PY{n+nf}{distance\PYZus{}func}\PY{p}{(}\PY{n+nb+bp}{self}\PY{p}{,} \PY{n}{lat1}\PY{p}{,} \PY{n}{lon1}\PY{p}{)}\PY{p}{:}
\PY{+w}{    }\PY{l+s+sd}{\PYZdq{}\PYZdq{}\PYZdq{}}
\PY{l+s+sd}{    Calculates the haversine distance between coordinates}
\PY{l+s+sd}{    on the latitude and longitude grid.}
\PY{l+s+sd}{    Distance is in km.}
\PY{l+s+sd}{    \PYZdq{}\PYZdq{}\PYZdq{}}
    \PY{n}{lat2} \PY{o}{=} \PY{l+m+mf}{40.641766}
    \PY{n}{lon2} \PY{o}{=} \PY{o}{\PYZhy{}}\PY{l+m+mf}{73.780968}

    \PY{n}{r} \PY{o}{=} \PY{l+m+mi}{6371}
    \PY{n}{phi1} \PY{o}{=} \PY{n}{np}\PY{o}{.}\PY{n}{radians}\PY{p}{(}\PY{n}{lat1}\PY{p}{)}
    \PY{n}{phi2} \PY{o}{=} \PY{n}{np}\PY{o}{.}\PY{n}{radians}\PY{p}{(}\PY{n}{lat2}\PY{p}{)}
    \PY{n}{delta\PYZus{}phi} \PY{o}{=} \PY{n}{np}\PY{o}{.}\PY{n}{radians}\PY{p}{(}\PY{n}{lat2} \PY{o}{\PYZhy{}} \PY{n}{lat1}\PY{p}{)}
    \PY{n}{delta\PYZus{}lambda} \PY{o}{=} \PY{n}{np}\PY{o}{.}\PY{n}{radians}\PY{p}{(}\PY{n}{lon2} \PY{o}{\PYZhy{}} \PY{n}{lon1}\PY{p}{)}
    \PY{n}{a} \PY{o}{=} \PY{p}{(}\PY{n}{np}\PY{o}{.}\PY{n}{sin}\PY{p}{(}\PY{n}{delta\PYZus{}phi} \PY{o}{/} \PY{l+m+mi}{2}\PY{p}{)} \PY{o}{*}\PY{o}{*} \PY{l+m+mi}{2} \PY{o}{+} \PY{n}{np}\PY{o}{.}\PY{n}{cos}\PY{p}{(}\PY{n}{phi1}\PY{p}{)} \PY{o}{*} \PY{n}{np}\PY{o}{.}\PY{n}{cos}\PY{p}{(}\PY{n}{phi2}\PY{p}{)} \PY{o}{*} \PY{n}{np}\PY{o}{.}\PY{n}{sin}\PY{p}{(}\PY{n}{delta\PYZus{}lambda} \PY{o}{/} \PY{l+m+mi}{2}\PY{p}{)} \PY{o}{*}\PY{o}{*} \PY{l+m+mi}{2}\PY{p}{)}
    \PY{n}{res} \PY{o}{=} \PY{n}{r} \PY{o}{*} \PY{p}{(}\PY{l+m+mi}{2} \PY{o}{*} \PY{n}{np}\PY{o}{.}\PY{n}{arctan2}\PY{p}{(}\PY{n}{np}\PY{o}{.}\PY{n}{sqrt}\PY{p}{(}\PY{n}{a}\PY{p}{)}\PY{p}{,} \PY{n}{np}\PY{o}{.}\PY{n}{sqrt}\PY{p}{(}\PY{l+m+mi}{1} \PY{o}{\PYZhy{}} \PY{n}{a}\PY{p}{)}\PY{p}{)}\PY{p}{)}
    \PY{k}{return} \PY{n}{np}\PY{o}{.}\PY{n}{round}\PY{p}{(}\PY{n}{res}\PY{p}{,} \PY{l+m+mi}{2}\PY{p}{)}
\end{Verbatim}
\end{tcolorbox}

    \hypertarget{test-your-new-agumentation-class-by-applying-it-to-the-numerical-data-you-created.-print-out-the-first-3-rows-of-the-resultant-data.}{%
\paragraph{-Test your new agumentation class by applying it to the
numerical data you created. Print out the first 3 rows of the resultant
data.}\label{test-your-new-agumentation-class-by-applying-it-to-the-numerical-data-you-created.-print-out-the-first-3-rows-of-the-resultant-data.}}

Do not worry about missing data since none of the features we used
involved nulls.

    \begin{tcolorbox}[breakable, size=fbox, boxrule=1pt, pad at break*=1mm,colback=cellbackground, colframe=cellborder]
\prompt{In}{incolor}{ }{\boxspacing}
\begin{Verbatim}[commandchars=\\\{\}]
\PY{n}{attribute\PYZus{}adder} \PY{o}{=} \PY{n}{AugmentFeatures}\PY{p}{(}\PY{p}{)}
\PY{n}{numerical\PYZus{}data} \PY{o}{=} \PY{n}{numerical\PYZus{}features}\PY{o}{.}\PY{n}{values}
\PY{n}{augmentation\PYZus{}test} \PY{o}{=} \PY{n}{attribute\PYZus{}adder}\PY{o}{.}\PY{n}{transform}\PY{p}{(}\PY{n}{numerical\PYZus{}data}\PY{p}{)}
\PY{n+nb}{print}\PY{p}{(}\PY{n}{augmentation\PYZus{}test}\PY{p}{[}\PY{l+m+mi}{0}\PY{p}{:}\PY{l+m+mi}{3}\PY{p}{]}\PY{p}{)}
\end{Verbatim}
\end{tcolorbox}

    \begin{Verbatim}[commandchars=\\\{\}]
[[ 40.64749    -73.97237      1.           9.           6.
  365.         365.          16.16      ]
 [ 40.75362    -73.98377      1.          45.           2.
  355.         355.          21.14      ]
 [ 40.80902    -73.9419       3.           0.           1.
  365.         121.66666667  23.02      ]]
    \end{Verbatim}

    \hypertarget{pts-create-a-sklearn-pipeline-that-performs-the-following-operations-of-the-feature-data}{%
\subsubsection{{[}10 pts{]} Create a sklearn pipeline that performs the
following operations of the feature
data}\label{pts-create-a-sklearn-pipeline-that-performs-the-following-operations-of-the-feature-data}}

Now, we will create a full pipeline that processes the data before
creating the model.

For the numerical data, perfrom the following operations in order: - Use
a SimpleImputer that imputes using the median value - Use the custom
feature augmentation made in the previous part - Use StandardScaler to
standardize the mean and standard deviation

For categorical features, perform the following: - Perform one hot
encoding on all the remaining categorical features:
\{neighbourhood\_group, room\_type\}

\textbf{After making the pipeline, perform the transform operation on
the feature data and print out the first 3 rows.}

    \begin{tcolorbox}[breakable, size=fbox, boxrule=1pt, pad at break*=1mm,colback=cellbackground, colframe=cellborder]
\prompt{In}{incolor}{ }{\boxspacing}
\begin{Verbatim}[commandchars=\\\{\}]
\PY{n}{num\PYZus{}pipeline} \PY{o}{=} \PY{n}{Pipeline}\PY{p}{(}
    \PY{p}{[}
        \PY{p}{(}\PY{l+s+s2}{\PYZdq{}}\PY{l+s+s2}{imputer}\PY{l+s+s2}{\PYZdq{}}\PY{p}{,} \PY{n}{SimpleImputer}\PY{p}{(}\PY{n}{strategy}\PY{o}{=}\PY{l+s+s2}{\PYZdq{}}\PY{l+s+s2}{median}\PY{l+s+s2}{\PYZdq{}}\PY{p}{)}\PY{p}{)}\PY{p}{,}
        \PY{p}{(}\PY{l+s+s2}{\PYZdq{}}\PY{l+s+s2}{attribs\PYZus{}adder}\PY{l+s+s2}{\PYZdq{}}\PY{p}{,} \PY{n}{AugmentFeatures}\PY{p}{(}\PY{p}{)}\PY{p}{)}\PY{p}{,}
        \PY{p}{(}\PY{l+s+s2}{\PYZdq{}}\PY{l+s+s2}{std\PYZus{}scaler}\PY{l+s+s2}{\PYZdq{}}\PY{p}{,} \PY{n}{StandardScaler}\PY{p}{(}\PY{p}{)}\PY{p}{)}\PY{p}{,}
    \PY{p}{]}
\PY{p}{)}

\PY{n}{num\PYZus{}features} \PY{o}{=} \PY{p}{[}\PY{l+s+s2}{\PYZdq{}}\PY{l+s+s2}{availability\PYZus{}365}\PY{l+s+s2}{\PYZdq{}}\PY{p}{,} \PY{l+s+s2}{\PYZdq{}}\PY{l+s+s2}{calculated\PYZus{}host\PYZus{}listings\PYZus{}count}\PY{l+s+s2}{\PYZdq{}}\PY{p}{,}\PY{l+s+s2}{\PYZdq{}}\PY{l+s+s2}{minimum\PYZus{}nights}\PY{l+s+s2}{\PYZdq{}}\PY{p}{,} \PY{l+s+s2}{\PYZdq{}}\PY{l+s+s2}{latitude}\PY{l+s+s2}{\PYZdq{}}\PY{p}{,}
                \PY{l+s+s2}{\PYZdq{}}\PY{l+s+s2}{number\PYZus{}of\PYZus{}reviews}\PY{l+s+s2}{\PYZdq{}}\PY{p}{,} \PY{l+s+s2}{\PYZdq{}}\PY{l+s+s2}{longitude}\PY{l+s+s2}{\PYZdq{}}\PY{p}{]}
\PY{n}{cat\PYZus{}features} \PY{o}{=} \PY{p}{[}\PY{l+s+s2}{\PYZdq{}}\PY{l+s+s2}{neighbourhood\PYZus{}group}\PY{l+s+s2}{\PYZdq{}}\PY{p}{,} \PY{l+s+s2}{\PYZdq{}}\PY{l+s+s2}{room\PYZus{}type}\PY{l+s+s2}{\PYZdq{}}\PY{p}{]}

\PY{n}{full\PYZus{}pipeline} \PY{o}{=} \PY{n}{ColumnTransformer}\PY{p}{(}
    \PY{p}{[}
      \PY{p}{(}\PY{l+s+s2}{\PYZdq{}}\PY{l+s+s2}{num}\PY{l+s+s2}{\PYZdq{}}\PY{p}{,} \PY{n}{num\PYZus{}pipeline}\PY{p}{,} \PY{n}{num\PYZus{}features}\PY{p}{)}\PY{p}{,}
      \PY{p}{(}\PY{l+s+s2}{\PYZdq{}}\PY{l+s+s2}{cat}\PY{l+s+s2}{\PYZdq{}}\PY{p}{,} \PY{n}{OneHotEncoder}\PY{p}{(}\PY{p}{)}\PY{p}{,} \PY{n}{cat\PYZus{}features}\PY{p}{)}\PY{p}{,}
    \PY{p}{]}
\PY{p}{)}

\PY{n}{airbnb\PYZus{}processed} \PY{o}{=} \PY{n}{full\PYZus{}pipeline}\PY{o}{.}\PY{n}{fit\PYZus{}transform}\PY{p}{(}\PY{n}{airbnb}\PY{p}{)}
\PY{n+nb}{print}\PY{p}{(}\PY{n}{airbnb\PYZus{}processed}\PY{p}{[}\PY{l+m+mi}{0}\PY{p}{:}\PY{l+m+mi}{3}\PY{p}{]}\PY{p}{)}
\end{Verbatim}
\end{tcolorbox}

    \begin{Verbatim}[commandchars=\\\{\}]
[[ 1.91625031e+00 -3.47164266e-02 -2.93996210e-01 -1.49384920e+00
  -3.20413582e-01 -4.37652087e-01 -1.52357197e+00 -1.57294594e-01
   0.00000000e+00  1.00000000e+00  0.00000000e+00  0.00000000e+00
   0.00000000e+00  0.00000000e+00  1.00000000e+00  0.00000000e+00]
 [ 1.84027456e+00 -1.56104443e-01 -2.93996210e-01  4.52436023e-01
   4.87664928e-01 -6.84639148e-01 -1.52402246e+00  1.63722867e-03
   0.00000000e+00  0.00000000e+00  1.00000000e+00  0.00000000e+00
   0.00000000e+00  1.00000000e+00  0.00000000e+00  0.00000000e+00]
 [ 1.91625031e+00 -1.86451447e-01 -1.96484417e-01  1.46839948e+00
  -5.22433210e-01  2.22496662e-01  4.25594350e-01 -3.24371470e-01
   0.00000000e+00  0.00000000e+00  1.00000000e+00  0.00000000e+00
   0.00000000e+00  0.00000000e+00  1.00000000e+00  0.00000000e+00]]
    \end{Verbatim}

    \hypertarget{pts-set-aside-20-of-the-data-as-test-test-80-train-20-test.-apply-previously-created-pipeline-to-the-train-and-test-data-separately-as-shown-in-the-introduction-example.}{%
\subsubsection{{[}5 pts{]} Set aside 20\% of the data as test test (80\%
train, 20\% test). Apply previously created pipeline to the train and
test data separately as shown in the introduction
example.}\label{pts-set-aside-20-of-the-data-as-test-test-80-train-20-test.-apply-previously-created-pipeline-to-the-train-and-test-data-separately-as-shown-in-the-introduction-example.}}

    \begin{tcolorbox}[breakable, size=fbox, boxrule=1pt, pad at break*=1mm,colback=cellbackground, colframe=cellborder]
\prompt{In}{incolor}{ }{\boxspacing}
\begin{Verbatim}[commandchars=\\\{\}]
\PY{k+kn}{from} \PY{n+nn}{sklearn}\PY{n+nn}{.}\PY{n+nn}{model\PYZus{}selection} \PY{k+kn}{import} \PY{n}{train\PYZus{}test\PYZus{}split}
\PY{n}{data\PYZus{}target} \PY{o}{=} \PY{n}{airbnb}\PY{p}{[}\PY{l+s+s2}{\PYZdq{}}\PY{l+s+s2}{price}\PY{l+s+s2}{\PYZdq{}}\PY{p}{]}
\PY{n}{train}\PY{p}{,} \PY{n}{test}\PY{p}{,} \PY{n}{target}\PY{p}{,} \PY{n}{target\PYZus{}test} \PY{o}{=} \PY{n}{train\PYZus{}test\PYZus{}split}\PY{p}{(}\PY{n}{feature\PYZus{}data}\PY{p}{,} \PY{n}{price\PYZus{}target}\PY{p}{,}
                                                   \PY{n}{test\PYZus{}size} \PY{o}{=} \PY{l+m+mf}{0.2}\PY{p}{,} \PY{n}{random\PYZus{}state}\PY{o}{=}\PY{l+m+mi}{0}\PY{p}{)}

\PY{n}{train} \PY{o}{=} \PY{n}{full\PYZus{}pipeline}\PY{o}{.}\PY{n}{fit\PYZus{}transform}\PY{p}{(}\PY{n}{train}\PY{p}{)}
\PY{n}{test} \PY{o}{=} \PY{n}{full\PYZus{}pipeline}\PY{o}{.}\PY{n}{fit\PYZus{}transform}\PY{p}{(}\PY{n}{test}\PY{p}{)}
\end{Verbatim}
\end{tcolorbox}

    \hypertarget{pts-fit-a-linear-regression-model}{%
\subsection{{[}20 pts{]} Fit a Linear Regression
Model}\label{pts-fit-a-linear-regression-model}}

The task is to predict the price, you could refer to the housing example
on how to train and evaluate your model using the mean squared error
(MSE). Provide both test and train set MSE values.

    \begin{tcolorbox}[breakable, size=fbox, boxrule=1pt, pad at break*=1mm,colback=cellbackground, colframe=cellborder]
\prompt{In}{incolor}{ }{\boxspacing}
\begin{Verbatim}[commandchars=\\\{\}]
\PY{k+kn}{from} \PY{n+nn}{sklearn}\PY{n+nn}{.}\PY{n+nn}{linear\PYZus{}model} \PY{k+kn}{import} \PY{n}{LinearRegression}
\PY{k+kn}{from} \PY{n+nn}{sklearn}\PY{n+nn}{.}\PY{n+nn}{metrics} \PY{k+kn}{import} \PY{n}{mean\PYZus{}squared\PYZus{}error}

\PY{n}{lin\PYZus{}reg} \PY{o}{=} \PY{n}{LinearRegression}\PY{p}{(}\PY{p}{)}
\PY{n}{lin\PYZus{}reg}\PY{o}{.}\PY{n}{fit}\PY{p}{(}\PY{n}{train}\PY{p}{,} \PY{n}{target}\PY{p}{)}

\PY{n}{preds\PYZus{}train} \PY{o}{=} \PY{n}{lin\PYZus{}reg}\PY{o}{.}\PY{n}{predict}\PY{p}{(}\PY{n}{train}\PY{p}{)}
\PY{n}{mse\PYZus{}train} \PY{o}{=} \PY{n}{mean\PYZus{}squared\PYZus{}error}\PY{p}{(}\PY{n}{target}\PY{p}{,} \PY{n}{preds\PYZus{}train}\PY{p}{)}
\PY{n+nb}{print}\PY{p}{(}\PY{l+s+s2}{\PYZdq{}}\PY{l+s+s2}{MSE Train:}\PY{l+s+s2}{\PYZdq{}}\PY{p}{,} \PY{n}{mse\PYZus{}train}\PY{p}{)}
\PY{n+nb}{print}\PY{p}{(}\PY{l+s+s2}{\PYZdq{}}\PY{l+s+s2}{RMSE Train:}\PY{l+s+s2}{\PYZdq{}}\PY{p}{,} \PY{n}{np}\PY{o}{.}\PY{n}{sqrt}\PY{p}{(}\PY{n}{mse\PYZus{}train}\PY{p}{)}\PY{p}{)}


\PY{n}{preds\PYZus{}test} \PY{o}{=} \PY{n}{lin\PYZus{}reg}\PY{o}{.}\PY{n}{predict}\PY{p}{(}\PY{n}{test}\PY{p}{)}
\PY{n}{mse\PYZus{}test} \PY{o}{=} \PY{n}{mean\PYZus{}squared\PYZus{}error}\PY{p}{(}\PY{n}{target\PYZus{}test}\PY{p}{,} \PY{n}{preds\PYZus{}test}\PY{p}{)}
\PY{n+nb}{print}\PY{p}{(}\PY{l+s+s2}{\PYZdq{}}\PY{l+s+s2}{MSE Test:}\PY{l+s+s2}{\PYZdq{}}\PY{p}{,} \PY{n}{mse\PYZus{}test}\PY{p}{)}
\PY{n+nb}{print}\PY{p}{(}\PY{l+s+s2}{\PYZdq{}}\PY{l+s+s2}{RMSE Test:}\PY{l+s+s2}{\PYZdq{}}\PY{p}{,} \PY{n}{np}\PY{o}{.}\PY{n}{sqrt}\PY{p}{(}\PY{n}{mse\PYZus{}test}\PY{p}{)}\PY{p}{)}
\end{Verbatim}
\end{tcolorbox}

    \begin{Verbatim}[commandchars=\\\{\}]
MSE Train: 52575.35280103505
RMSE Train: 229.2931590803246
MSE Test: 48457.49423197792
RMSE Test: 220.1306299268185
    \end{Verbatim}

    \begin{tcolorbox}[breakable, size=fbox, boxrule=1pt, pad at break*=1mm,colback=cellbackground, colframe=cellborder]
\prompt{In}{incolor}{9}{\boxspacing}
\begin{Verbatim}[commandchars=\\\{\}]
\PY{o}{!}jupyter\PY{+w}{ }nbconvert\PY{+w}{ }\PYZhy{}\PYZhy{}to\PY{+w}{ }html\PY{+w}{ }CM148\PYZus{}Project1\PYZus{}Final
\end{Verbatim}
\end{tcolorbox}

    \begin{Verbatim}[commandchars=\\\{\}]
[NbConvertApp] WARNING | pattern 'CM148\_Project1\_Final' matched no files
This application is used to convert notebook files (*.ipynb)
        to various other formats.

        WARNING: THE COMMANDLINE INTERFACE MAY CHANGE IN FUTURE RELEASES.

Options
=======
The options below are convenience aliases to configurable class-options,
as listed in the "Equivalent to" description-line of the aliases.
To see all configurable class-options for some <cmd>, use:
    <cmd> --help-all

--debug
    set log level to logging.DEBUG (maximize logging output)
    Equivalent to: [--Application.log\_level=10]
--show-config
    Show the application's configuration (human-readable format)
    Equivalent to: [--Application.show\_config=True]
--show-config-json
    Show the application's configuration (json format)
    Equivalent to: [--Application.show\_config\_json=True]
--generate-config
    generate default config file
    Equivalent to: [--JupyterApp.generate\_config=True]
-y
    Answer yes to any questions instead of prompting.
    Equivalent to: [--JupyterApp.answer\_yes=True]
--execute
    Execute the notebook prior to export.
    Equivalent to: [--ExecutePreprocessor.enabled=True]
--allow-errors
    Continue notebook execution even if one of the cells throws an error and
include the error message in the cell output (the default behaviour is to abort
conversion). This flag is only relevant if '--execute' was specified, too.
    Equivalent to: [--ExecutePreprocessor.allow\_errors=True]
--stdin
    read a single notebook file from stdin. Write the resulting notebook with
default basename 'notebook.*'
    Equivalent to: [--NbConvertApp.from\_stdin=True]
--stdout
    Write notebook output to stdout instead of files.
    Equivalent to: [--NbConvertApp.writer\_class=StdoutWriter]
--inplace
    Run nbconvert in place, overwriting the existing notebook (only
            relevant when converting to notebook format)
    Equivalent to: [--NbConvertApp.use\_output\_suffix=False
--NbConvertApp.export\_format=notebook --FilesWriter.build\_directory=]
--clear-output
    Clear output of current file and save in place,
            overwriting the existing notebook.
    Equivalent to: [--NbConvertApp.use\_output\_suffix=False
--NbConvertApp.export\_format=notebook --FilesWriter.build\_directory=
--ClearOutputPreprocessor.enabled=True]
--no-prompt
    Exclude input and output prompts from converted document.
    Equivalent to: [--TemplateExporter.exclude\_input\_prompt=True
--TemplateExporter.exclude\_output\_prompt=True]
--no-input
    Exclude input cells and output prompts from converted document.
            This mode is ideal for generating code-free reports.
    Equivalent to: [--TemplateExporter.exclude\_output\_prompt=True
--TemplateExporter.exclude\_input=True
--TemplateExporter.exclude\_input\_prompt=True]
--allow-chromium-download
    Whether to allow downloading chromium if no suitable version is found on the
system.
    Equivalent to: [--WebPDFExporter.allow\_chromium\_download=True]
--disable-chromium-sandbox
    Disable chromium security sandbox when converting to PDF..
    Equivalent to: [--WebPDFExporter.disable\_sandbox=True]
--show-input
    Shows code input. This flag is only useful for dejavu users.
    Equivalent to: [--TemplateExporter.exclude\_input=False]
--embed-images
    Embed the images as base64 dataurls in the output. This flag is only useful
for the HTML/WebPDF/Slides exports.
    Equivalent to: [--HTMLExporter.embed\_images=True]
--sanitize-html
    Whether the HTML in Markdown cells and cell outputs should be sanitized..
    Equivalent to: [--HTMLExporter.sanitize\_html=True]
--log-level=<Enum>
    Set the log level by value or name.
    Choices: any of [0, 10, 20, 30, 40, 50, 'DEBUG', 'INFO', 'WARN', 'ERROR',
'CRITICAL']
    Default: 30
    Equivalent to: [--Application.log\_level]
--config=<Unicode>
    Full path of a config file.
    Default: ''
    Equivalent to: [--JupyterApp.config\_file]
--to=<Unicode>
    The export format to be used, either one of the built-in formats
            ['asciidoc', 'custom', 'html', 'latex', 'markdown', 'notebook',
'pdf', 'python', 'rst', 'script', 'slides', 'webpdf']
            or a dotted object name that represents the import path for an
            ``Exporter`` class
    Default: ''
    Equivalent to: [--NbConvertApp.export\_format]
--template=<Unicode>
    Name of the template to use
    Default: ''
    Equivalent to: [--TemplateExporter.template\_name]
--template-file=<Unicode>
    Name of the template file to use
    Default: None
    Equivalent to: [--TemplateExporter.template\_file]
--theme=<Unicode>
    Template specific theme(e.g. the name of a JupyterLab CSS theme distributed
    as prebuilt extension for the lab template)
    Default: 'light'
    Equivalent to: [--HTMLExporter.theme]
--sanitize\_html=<Bool>
    Whether the HTML in Markdown cells and cell outputs should be sanitized.This
    should be set to True by nbviewer or similar tools.
    Default: False
    Equivalent to: [--HTMLExporter.sanitize\_html]
--writer=<DottedObjectName>
    Writer class used to write the
                                        results of the conversion
    Default: 'FilesWriter'
    Equivalent to: [--NbConvertApp.writer\_class]
--post=<DottedOrNone>
    PostProcessor class used to write the
                                        results of the conversion
    Default: ''
    Equivalent to: [--NbConvertApp.postprocessor\_class]
--output=<Unicode>
    overwrite base name use for output files.
                can only be used when converting one notebook at a time.
    Default: ''
    Equivalent to: [--NbConvertApp.output\_base]
--output-dir=<Unicode>
    Directory to write output(s) to. Defaults
                                  to output to the directory of each notebook.
To recover
                                  previous default behaviour (outputting to the
current
                                  working directory) use . as the flag value.
    Default: ''
    Equivalent to: [--FilesWriter.build\_directory]
--reveal-prefix=<Unicode>
    The URL prefix for reveal.js (version 3.x).
            This defaults to the reveal CDN, but can be any url pointing to a
copy
            of reveal.js.
            For speaker notes to work, this must be a relative path to a local
            copy of reveal.js: e.g., "reveal.js".
            If a relative path is given, it must be a subdirectory of the
            current directory (from which the server is run).
            See the usage documentation
            (https://nbconvert.readthedocs.io/en/latest/usage.html\#reveal-js-
html-slideshow)
            for more details.
    Default: ''
    Equivalent to: [--SlidesExporter.reveal\_url\_prefix]
--nbformat=<Enum>
    The nbformat version to write.
            Use this to downgrade notebooks.
    Choices: any of [1, 2, 3, 4]
    Default: 4
    Equivalent to: [--NotebookExporter.nbformat\_version]

Examples
--------

    The simplest way to use nbconvert is

            > jupyter nbconvert mynotebook.ipynb --to html

            Options include ['asciidoc', 'custom', 'html', 'latex', 'markdown',
'notebook', 'pdf', 'python', 'rst', 'script', 'slides', 'webpdf'].

            > jupyter nbconvert --to latex mynotebook.ipynb

            Both HTML and LaTeX support multiple output templates. LaTeX
includes
            'base', 'article' and 'report'.  HTML includes 'basic', 'lab' and
            'classic'. You can specify the flavor of the format used.

            > jupyter nbconvert --to html --template lab mynotebook.ipynb

            You can also pipe the output to stdout, rather than a file

            > jupyter nbconvert mynotebook.ipynb --stdout

            PDF is generated via latex

            > jupyter nbconvert mynotebook.ipynb --to pdf

            You can get (and serve) a Reveal.js-powered slideshow

            > jupyter nbconvert myslides.ipynb --to slides --post serve

            Multiple notebooks can be given at the command line in a couple of
            different ways:

            > jupyter nbconvert notebook*.ipynb
            > jupyter nbconvert notebook1.ipynb notebook2.ipynb

            or you can specify the notebooks list in a config file, containing::

                c.NbConvertApp.notebooks = ["my\_notebook.ipynb"]

            > jupyter nbconvert --config mycfg.py

To see all available configurables, use `--help-all`.

    \end{Verbatim}


    % Add a bibliography block to the postdoc
    
    
    
\end{document}
