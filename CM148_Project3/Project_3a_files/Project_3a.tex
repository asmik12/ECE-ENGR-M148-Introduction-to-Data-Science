\documentclass[11pt]{article}

    \usepackage[breakable]{tcolorbox}
    \usepackage{parskip} % Stop auto-indenting (to mimic markdown behaviour)
    

    % Basic figure setup, for now with no caption control since it's done
    % automatically by Pandoc (which extracts ![](path) syntax from Markdown).
    \usepackage{graphicx}
    % Maintain compatibility with old templates. Remove in nbconvert 6.0
    \let\Oldincludegraphics\includegraphics
    % Ensure that by default, figures have no caption (until we provide a
    % proper Figure object with a Caption API and a way to capture that
    % in the conversion process - todo).
    \usepackage{caption}
    \DeclareCaptionFormat{nocaption}{}
    \captionsetup{format=nocaption,aboveskip=0pt,belowskip=0pt}

    \usepackage{float}
    \floatplacement{figure}{H} % forces figures to be placed at the correct location
    \usepackage{xcolor} % Allow colors to be defined
    \usepackage{enumerate} % Needed for markdown enumerations to work
    \usepackage{geometry} % Used to adjust the document margins
    \usepackage{amsmath} % Equations
    \usepackage{amssymb} % Equations
    \usepackage{textcomp} % defines textquotesingle
    % Hack from http://tex.stackexchange.com/a/47451/13684:
    \AtBeginDocument{%
        \def\PYZsq{\textquotesingle}% Upright quotes in Pygmentized code
    }
    \usepackage{upquote} % Upright quotes for verbatim code
    \usepackage{eurosym} % defines \euro

    \usepackage{iftex}
    \ifPDFTeX
        \usepackage[T1]{fontenc}
        \IfFileExists{alphabeta.sty}{
              \usepackage{alphabeta}
          }{
              \usepackage[mathletters]{ucs}
              \usepackage[utf8x]{inputenc}
          }
    \else
        \usepackage{fontspec}
        \usepackage{unicode-math}
    \fi

    \usepackage{fancyvrb} % verbatim replacement that allows latex
    \usepackage{grffile} % extends the file name processing of package graphics
                         % to support a larger range
    \makeatletter % fix for old versions of grffile with XeLaTeX
    \@ifpackagelater{grffile}{2019/11/01}
    {
      % Do nothing on new versions
    }
    {
      \def\Gread@@xetex#1{%
        \IfFileExists{"\Gin@base".bb}%
        {\Gread@eps{\Gin@base.bb}}%
        {\Gread@@xetex@aux#1}%
      }
    }
    \makeatother
    \usepackage[Export]{adjustbox} % Used to constrain images to a maximum size
    \adjustboxset{max size={0.9\linewidth}{0.9\paperheight}}

    % The hyperref package gives us a pdf with properly built
    % internal navigation ('pdf bookmarks' for the table of contents,
    % internal cross-reference links, web links for URLs, etc.)
    \usepackage{hyperref}
    % The default LaTeX title has an obnoxious amount of whitespace. By default,
    % titling removes some of it. It also provides customization options.
    \usepackage{titling}
    \usepackage{longtable} % longtable support required by pandoc >1.10
    \usepackage{booktabs}  % table support for pandoc > 1.12.2
    \usepackage{array}     % table support for pandoc >= 2.11.3
    \usepackage{calc}      % table minipage width calculation for pandoc >= 2.11.1
    \usepackage[inline]{enumitem} % IRkernel/repr support (it uses the enumerate* environment)
    \usepackage[normalem]{ulem} % ulem is needed to support strikethroughs (\sout)
                                % normalem makes italics be italics, not underlines
    \usepackage{mathrsfs}
    

    
    % Colors for the hyperref package
    \definecolor{urlcolor}{rgb}{0,.145,.698}
    \definecolor{linkcolor}{rgb}{.71,0.21,0.01}
    \definecolor{citecolor}{rgb}{.12,.54,.11}

    % ANSI colors
    \definecolor{ansi-black}{HTML}{3E424D}
    \definecolor{ansi-black-intense}{HTML}{282C36}
    \definecolor{ansi-red}{HTML}{E75C58}
    \definecolor{ansi-red-intense}{HTML}{B22B31}
    \definecolor{ansi-green}{HTML}{00A250}
    \definecolor{ansi-green-intense}{HTML}{007427}
    \definecolor{ansi-yellow}{HTML}{DDB62B}
    \definecolor{ansi-yellow-intense}{HTML}{B27D12}
    \definecolor{ansi-blue}{HTML}{208FFB}
    \definecolor{ansi-blue-intense}{HTML}{0065CA}
    \definecolor{ansi-magenta}{HTML}{D160C4}
    \definecolor{ansi-magenta-intense}{HTML}{A03196}
    \definecolor{ansi-cyan}{HTML}{60C6C8}
    \definecolor{ansi-cyan-intense}{HTML}{258F8F}
    \definecolor{ansi-white}{HTML}{C5C1B4}
    \definecolor{ansi-white-intense}{HTML}{A1A6B2}
    \definecolor{ansi-default-inverse-fg}{HTML}{FFFFFF}
    \definecolor{ansi-default-inverse-bg}{HTML}{000000}

    % common color for the border for error outputs.
    \definecolor{outerrorbackground}{HTML}{FFDFDF}

    % commands and environments needed by pandoc snippets
    % extracted from the output of `pandoc -s`
    \providecommand{\tightlist}{%
      \setlength{\itemsep}{0pt}\setlength{\parskip}{0pt}}
    \DefineVerbatimEnvironment{Highlighting}{Verbatim}{commandchars=\\\{\}}
    % Add ',fontsize=\small' for more characters per line
    \newenvironment{Shaded}{}{}
    \newcommand{\KeywordTok}[1]{\textcolor[rgb]{0.00,0.44,0.13}{\textbf{{#1}}}}
    \newcommand{\DataTypeTok}[1]{\textcolor[rgb]{0.56,0.13,0.00}{{#1}}}
    \newcommand{\DecValTok}[1]{\textcolor[rgb]{0.25,0.63,0.44}{{#1}}}
    \newcommand{\BaseNTok}[1]{\textcolor[rgb]{0.25,0.63,0.44}{{#1}}}
    \newcommand{\FloatTok}[1]{\textcolor[rgb]{0.25,0.63,0.44}{{#1}}}
    \newcommand{\CharTok}[1]{\textcolor[rgb]{0.25,0.44,0.63}{{#1}}}
    \newcommand{\StringTok}[1]{\textcolor[rgb]{0.25,0.44,0.63}{{#1}}}
    \newcommand{\CommentTok}[1]{\textcolor[rgb]{0.38,0.63,0.69}{\textit{{#1}}}}
    \newcommand{\OtherTok}[1]{\textcolor[rgb]{0.00,0.44,0.13}{{#1}}}
    \newcommand{\AlertTok}[1]{\textcolor[rgb]{1.00,0.00,0.00}{\textbf{{#1}}}}
    \newcommand{\FunctionTok}[1]{\textcolor[rgb]{0.02,0.16,0.49}{{#1}}}
    \newcommand{\RegionMarkerTok}[1]{{#1}}
    \newcommand{\ErrorTok}[1]{\textcolor[rgb]{1.00,0.00,0.00}{\textbf{{#1}}}}
    \newcommand{\NormalTok}[1]{{#1}}

    % Additional commands for more recent versions of Pandoc
    \newcommand{\ConstantTok}[1]{\textcolor[rgb]{0.53,0.00,0.00}{{#1}}}
    \newcommand{\SpecialCharTok}[1]{\textcolor[rgb]{0.25,0.44,0.63}{{#1}}}
    \newcommand{\VerbatimStringTok}[1]{\textcolor[rgb]{0.25,0.44,0.63}{{#1}}}
    \newcommand{\SpecialStringTok}[1]{\textcolor[rgb]{0.73,0.40,0.53}{{#1}}}
    \newcommand{\ImportTok}[1]{{#1}}
    \newcommand{\DocumentationTok}[1]{\textcolor[rgb]{0.73,0.13,0.13}{\textit{{#1}}}}
    \newcommand{\AnnotationTok}[1]{\textcolor[rgb]{0.38,0.63,0.69}{\textbf{\textit{{#1}}}}}
    \newcommand{\CommentVarTok}[1]{\textcolor[rgb]{0.38,0.63,0.69}{\textbf{\textit{{#1}}}}}
    \newcommand{\VariableTok}[1]{\textcolor[rgb]{0.10,0.09,0.49}{{#1}}}
    \newcommand{\ControlFlowTok}[1]{\textcolor[rgb]{0.00,0.44,0.13}{\textbf{{#1}}}}
    \newcommand{\OperatorTok}[1]{\textcolor[rgb]{0.40,0.40,0.40}{{#1}}}
    \newcommand{\BuiltInTok}[1]{{#1}}
    \newcommand{\ExtensionTok}[1]{{#1}}
    \newcommand{\PreprocessorTok}[1]{\textcolor[rgb]{0.74,0.48,0.00}{{#1}}}
    \newcommand{\AttributeTok}[1]{\textcolor[rgb]{0.49,0.56,0.16}{{#1}}}
    \newcommand{\InformationTok}[1]{\textcolor[rgb]{0.38,0.63,0.69}{\textbf{\textit{{#1}}}}}
    \newcommand{\WarningTok}[1]{\textcolor[rgb]{0.38,0.63,0.69}{\textbf{\textit{{#1}}}}}


    % Define a nice break command that doesn't care if a line doesn't already
    % exist.
    \def\br{\hspace*{\fill} \\* }
    % Math Jax compatibility definitions
    \def\gt{>}
    \def\lt{<}
    \let\Oldtex\TeX
    \let\Oldlatex\LaTeX
    \renewcommand{\TeX}{\textrm{\Oldtex}}
    \renewcommand{\LaTeX}{\textrm{\Oldlatex}}
    % Document parameters
    % Document title
    \title{Project\_3a}
    
    
    
    
    
% Pygments definitions
\makeatletter
\def\PY@reset{\let\PY@it=\relax \let\PY@bf=\relax%
    \let\PY@ul=\relax \let\PY@tc=\relax%
    \let\PY@bc=\relax \let\PY@ff=\relax}
\def\PY@tok#1{\csname PY@tok@#1\endcsname}
\def\PY@toks#1+{\ifx\relax#1\empty\else%
    \PY@tok{#1}\expandafter\PY@toks\fi}
\def\PY@do#1{\PY@bc{\PY@tc{\PY@ul{%
    \PY@it{\PY@bf{\PY@ff{#1}}}}}}}
\def\PY#1#2{\PY@reset\PY@toks#1+\relax+\PY@do{#2}}

\@namedef{PY@tok@w}{\def\PY@tc##1{\textcolor[rgb]{0.73,0.73,0.73}{##1}}}
\@namedef{PY@tok@c}{\let\PY@it=\textit\def\PY@tc##1{\textcolor[rgb]{0.24,0.48,0.48}{##1}}}
\@namedef{PY@tok@cp}{\def\PY@tc##1{\textcolor[rgb]{0.61,0.40,0.00}{##1}}}
\@namedef{PY@tok@k}{\let\PY@bf=\textbf\def\PY@tc##1{\textcolor[rgb]{0.00,0.50,0.00}{##1}}}
\@namedef{PY@tok@kp}{\def\PY@tc##1{\textcolor[rgb]{0.00,0.50,0.00}{##1}}}
\@namedef{PY@tok@kt}{\def\PY@tc##1{\textcolor[rgb]{0.69,0.00,0.25}{##1}}}
\@namedef{PY@tok@o}{\def\PY@tc##1{\textcolor[rgb]{0.40,0.40,0.40}{##1}}}
\@namedef{PY@tok@ow}{\let\PY@bf=\textbf\def\PY@tc##1{\textcolor[rgb]{0.67,0.13,1.00}{##1}}}
\@namedef{PY@tok@nb}{\def\PY@tc##1{\textcolor[rgb]{0.00,0.50,0.00}{##1}}}
\@namedef{PY@tok@nf}{\def\PY@tc##1{\textcolor[rgb]{0.00,0.00,1.00}{##1}}}
\@namedef{PY@tok@nc}{\let\PY@bf=\textbf\def\PY@tc##1{\textcolor[rgb]{0.00,0.00,1.00}{##1}}}
\@namedef{PY@tok@nn}{\let\PY@bf=\textbf\def\PY@tc##1{\textcolor[rgb]{0.00,0.00,1.00}{##1}}}
\@namedef{PY@tok@ne}{\let\PY@bf=\textbf\def\PY@tc##1{\textcolor[rgb]{0.80,0.25,0.22}{##1}}}
\@namedef{PY@tok@nv}{\def\PY@tc##1{\textcolor[rgb]{0.10,0.09,0.49}{##1}}}
\@namedef{PY@tok@no}{\def\PY@tc##1{\textcolor[rgb]{0.53,0.00,0.00}{##1}}}
\@namedef{PY@tok@nl}{\def\PY@tc##1{\textcolor[rgb]{0.46,0.46,0.00}{##1}}}
\@namedef{PY@tok@ni}{\let\PY@bf=\textbf\def\PY@tc##1{\textcolor[rgb]{0.44,0.44,0.44}{##1}}}
\@namedef{PY@tok@na}{\def\PY@tc##1{\textcolor[rgb]{0.41,0.47,0.13}{##1}}}
\@namedef{PY@tok@nt}{\let\PY@bf=\textbf\def\PY@tc##1{\textcolor[rgb]{0.00,0.50,0.00}{##1}}}
\@namedef{PY@tok@nd}{\def\PY@tc##1{\textcolor[rgb]{0.67,0.13,1.00}{##1}}}
\@namedef{PY@tok@s}{\def\PY@tc##1{\textcolor[rgb]{0.73,0.13,0.13}{##1}}}
\@namedef{PY@tok@sd}{\let\PY@it=\textit\def\PY@tc##1{\textcolor[rgb]{0.73,0.13,0.13}{##1}}}
\@namedef{PY@tok@si}{\let\PY@bf=\textbf\def\PY@tc##1{\textcolor[rgb]{0.64,0.35,0.47}{##1}}}
\@namedef{PY@tok@se}{\let\PY@bf=\textbf\def\PY@tc##1{\textcolor[rgb]{0.67,0.36,0.12}{##1}}}
\@namedef{PY@tok@sr}{\def\PY@tc##1{\textcolor[rgb]{0.64,0.35,0.47}{##1}}}
\@namedef{PY@tok@ss}{\def\PY@tc##1{\textcolor[rgb]{0.10,0.09,0.49}{##1}}}
\@namedef{PY@tok@sx}{\def\PY@tc##1{\textcolor[rgb]{0.00,0.50,0.00}{##1}}}
\@namedef{PY@tok@m}{\def\PY@tc##1{\textcolor[rgb]{0.40,0.40,0.40}{##1}}}
\@namedef{PY@tok@gh}{\let\PY@bf=\textbf\def\PY@tc##1{\textcolor[rgb]{0.00,0.00,0.50}{##1}}}
\@namedef{PY@tok@gu}{\let\PY@bf=\textbf\def\PY@tc##1{\textcolor[rgb]{0.50,0.00,0.50}{##1}}}
\@namedef{PY@tok@gd}{\def\PY@tc##1{\textcolor[rgb]{0.63,0.00,0.00}{##1}}}
\@namedef{PY@tok@gi}{\def\PY@tc##1{\textcolor[rgb]{0.00,0.52,0.00}{##1}}}
\@namedef{PY@tok@gr}{\def\PY@tc##1{\textcolor[rgb]{0.89,0.00,0.00}{##1}}}
\@namedef{PY@tok@ge}{\let\PY@it=\textit}
\@namedef{PY@tok@gs}{\let\PY@bf=\textbf}
\@namedef{PY@tok@ges}{\let\PY@bf=\textbf\let\PY@it=\textit}
\@namedef{PY@tok@gp}{\let\PY@bf=\textbf\def\PY@tc##1{\textcolor[rgb]{0.00,0.00,0.50}{##1}}}
\@namedef{PY@tok@go}{\def\PY@tc##1{\textcolor[rgb]{0.44,0.44,0.44}{##1}}}
\@namedef{PY@tok@gt}{\def\PY@tc##1{\textcolor[rgb]{0.00,0.27,0.87}{##1}}}
\@namedef{PY@tok@err}{\def\PY@bc##1{{\setlength{\fboxsep}{\string -\fboxrule}\fcolorbox[rgb]{1.00,0.00,0.00}{1,1,1}{\strut ##1}}}}
\@namedef{PY@tok@kc}{\let\PY@bf=\textbf\def\PY@tc##1{\textcolor[rgb]{0.00,0.50,0.00}{##1}}}
\@namedef{PY@tok@kd}{\let\PY@bf=\textbf\def\PY@tc##1{\textcolor[rgb]{0.00,0.50,0.00}{##1}}}
\@namedef{PY@tok@kn}{\let\PY@bf=\textbf\def\PY@tc##1{\textcolor[rgb]{0.00,0.50,0.00}{##1}}}
\@namedef{PY@tok@kr}{\let\PY@bf=\textbf\def\PY@tc##1{\textcolor[rgb]{0.00,0.50,0.00}{##1}}}
\@namedef{PY@tok@bp}{\def\PY@tc##1{\textcolor[rgb]{0.00,0.50,0.00}{##1}}}
\@namedef{PY@tok@fm}{\def\PY@tc##1{\textcolor[rgb]{0.00,0.00,1.00}{##1}}}
\@namedef{PY@tok@vc}{\def\PY@tc##1{\textcolor[rgb]{0.10,0.09,0.49}{##1}}}
\@namedef{PY@tok@vg}{\def\PY@tc##1{\textcolor[rgb]{0.10,0.09,0.49}{##1}}}
\@namedef{PY@tok@vi}{\def\PY@tc##1{\textcolor[rgb]{0.10,0.09,0.49}{##1}}}
\@namedef{PY@tok@vm}{\def\PY@tc##1{\textcolor[rgb]{0.10,0.09,0.49}{##1}}}
\@namedef{PY@tok@sa}{\def\PY@tc##1{\textcolor[rgb]{0.73,0.13,0.13}{##1}}}
\@namedef{PY@tok@sb}{\def\PY@tc##1{\textcolor[rgb]{0.73,0.13,0.13}{##1}}}
\@namedef{PY@tok@sc}{\def\PY@tc##1{\textcolor[rgb]{0.73,0.13,0.13}{##1}}}
\@namedef{PY@tok@dl}{\def\PY@tc##1{\textcolor[rgb]{0.73,0.13,0.13}{##1}}}
\@namedef{PY@tok@s2}{\def\PY@tc##1{\textcolor[rgb]{0.73,0.13,0.13}{##1}}}
\@namedef{PY@tok@sh}{\def\PY@tc##1{\textcolor[rgb]{0.73,0.13,0.13}{##1}}}
\@namedef{PY@tok@s1}{\def\PY@tc##1{\textcolor[rgb]{0.73,0.13,0.13}{##1}}}
\@namedef{PY@tok@mb}{\def\PY@tc##1{\textcolor[rgb]{0.40,0.40,0.40}{##1}}}
\@namedef{PY@tok@mf}{\def\PY@tc##1{\textcolor[rgb]{0.40,0.40,0.40}{##1}}}
\@namedef{PY@tok@mh}{\def\PY@tc##1{\textcolor[rgb]{0.40,0.40,0.40}{##1}}}
\@namedef{PY@tok@mi}{\def\PY@tc##1{\textcolor[rgb]{0.40,0.40,0.40}{##1}}}
\@namedef{PY@tok@il}{\def\PY@tc##1{\textcolor[rgb]{0.40,0.40,0.40}{##1}}}
\@namedef{PY@tok@mo}{\def\PY@tc##1{\textcolor[rgb]{0.40,0.40,0.40}{##1}}}
\@namedef{PY@tok@ch}{\let\PY@it=\textit\def\PY@tc##1{\textcolor[rgb]{0.24,0.48,0.48}{##1}}}
\@namedef{PY@tok@cm}{\let\PY@it=\textit\def\PY@tc##1{\textcolor[rgb]{0.24,0.48,0.48}{##1}}}
\@namedef{PY@tok@cpf}{\let\PY@it=\textit\def\PY@tc##1{\textcolor[rgb]{0.24,0.48,0.48}{##1}}}
\@namedef{PY@tok@c1}{\let\PY@it=\textit\def\PY@tc##1{\textcolor[rgb]{0.24,0.48,0.48}{##1}}}
\@namedef{PY@tok@cs}{\let\PY@it=\textit\def\PY@tc##1{\textcolor[rgb]{0.24,0.48,0.48}{##1}}}

\def\PYZbs{\char`\\}
\def\PYZus{\char`\_}
\def\PYZob{\char`\{}
\def\PYZcb{\char`\}}
\def\PYZca{\char`\^}
\def\PYZam{\char`\&}
\def\PYZlt{\char`\<}
\def\PYZgt{\char`\>}
\def\PYZsh{\char`\#}
\def\PYZpc{\char`\%}
\def\PYZdl{\char`\$}
\def\PYZhy{\char`\-}
\def\PYZsq{\char`\'}
\def\PYZdq{\char`\"}
\def\PYZti{\char`\~}
% for compatibility with earlier versions
\def\PYZat{@}
\def\PYZlb{[}
\def\PYZrb{]}
\makeatother


    % For linebreaks inside Verbatim environment from package fancyvrb.
    \makeatletter
        \newbox\Wrappedcontinuationbox
        \newbox\Wrappedvisiblespacebox
        \newcommand*\Wrappedvisiblespace {\textcolor{red}{\textvisiblespace}}
        \newcommand*\Wrappedcontinuationsymbol {\textcolor{red}{\llap{\tiny$\m@th\hookrightarrow$}}}
        \newcommand*\Wrappedcontinuationindent {3ex }
        \newcommand*\Wrappedafterbreak {\kern\Wrappedcontinuationindent\copy\Wrappedcontinuationbox}
        % Take advantage of the already applied Pygments mark-up to insert
        % potential linebreaks for TeX processing.
        %        {, <, #, %, $, ' and ": go to next line.
        %        _, }, ^, &, >, - and ~: stay at end of broken line.
        % Use of \textquotesingle for straight quote.
        \newcommand*\Wrappedbreaksatspecials {%
            \def\PYGZus{\discretionary{\char`\_}{\Wrappedafterbreak}{\char`\_}}%
            \def\PYGZob{\discretionary{}{\Wrappedafterbreak\char`\{}{\char`\{}}%
            \def\PYGZcb{\discretionary{\char`\}}{\Wrappedafterbreak}{\char`\}}}%
            \def\PYGZca{\discretionary{\char`\^}{\Wrappedafterbreak}{\char`\^}}%
            \def\PYGZam{\discretionary{\char`\&}{\Wrappedafterbreak}{\char`\&}}%
            \def\PYGZlt{\discretionary{}{\Wrappedafterbreak\char`\<}{\char`\<}}%
            \def\PYGZgt{\discretionary{\char`\>}{\Wrappedafterbreak}{\char`\>}}%
            \def\PYGZsh{\discretionary{}{\Wrappedafterbreak\char`\#}{\char`\#}}%
            \def\PYGZpc{\discretionary{}{\Wrappedafterbreak\char`\%}{\char`\%}}%
            \def\PYGZdl{\discretionary{}{\Wrappedafterbreak\char`\$}{\char`\$}}%
            \def\PYGZhy{\discretionary{\char`\-}{\Wrappedafterbreak}{\char`\-}}%
            \def\PYGZsq{\discretionary{}{\Wrappedafterbreak\textquotesingle}{\textquotesingle}}%
            \def\PYGZdq{\discretionary{}{\Wrappedafterbreak\char`\"}{\char`\"}}%
            \def\PYGZti{\discretionary{\char`\~}{\Wrappedafterbreak}{\char`\~}}%
        }
        % Some characters . , ; ? ! / are not pygmentized.
        % This macro makes them "active" and they will insert potential linebreaks
        \newcommand*\Wrappedbreaksatpunct {%
            \lccode`\~`\.\lowercase{\def~}{\discretionary{\hbox{\char`\.}}{\Wrappedafterbreak}{\hbox{\char`\.}}}%
            \lccode`\~`\,\lowercase{\def~}{\discretionary{\hbox{\char`\,}}{\Wrappedafterbreak}{\hbox{\char`\,}}}%
            \lccode`\~`\;\lowercase{\def~}{\discretionary{\hbox{\char`\;}}{\Wrappedafterbreak}{\hbox{\char`\;}}}%
            \lccode`\~`\:\lowercase{\def~}{\discretionary{\hbox{\char`\:}}{\Wrappedafterbreak}{\hbox{\char`\:}}}%
            \lccode`\~`\?\lowercase{\def~}{\discretionary{\hbox{\char`\?}}{\Wrappedafterbreak}{\hbox{\char`\?}}}%
            \lccode`\~`\!\lowercase{\def~}{\discretionary{\hbox{\char`\!}}{\Wrappedafterbreak}{\hbox{\char`\!}}}%
            \lccode`\~`\/\lowercase{\def~}{\discretionary{\hbox{\char`\/}}{\Wrappedafterbreak}{\hbox{\char`\/}}}%
            \catcode`\.\active
            \catcode`\,\active
            \catcode`\;\active
            \catcode`\:\active
            \catcode`\?\active
            \catcode`\!\active
            \catcode`\/\active
            \lccode`\~`\~
        }
    \makeatother

    \let\OriginalVerbatim=\Verbatim
    \makeatletter
    \renewcommand{\Verbatim}[1][1]{%
        %\parskip\z@skip
        \sbox\Wrappedcontinuationbox {\Wrappedcontinuationsymbol}%
        \sbox\Wrappedvisiblespacebox {\FV@SetupFont\Wrappedvisiblespace}%
        \def\FancyVerbFormatLine ##1{\hsize\linewidth
            \vtop{\raggedright\hyphenpenalty\z@\exhyphenpenalty\z@
                \doublehyphendemerits\z@\finalhyphendemerits\z@
                \strut ##1\strut}%
        }%
        % If the linebreak is at a space, the latter will be displayed as visible
        % space at end of first line, and a continuation symbol starts next line.
        % Stretch/shrink are however usually zero for typewriter font.
        \def\FV@Space {%
            \nobreak\hskip\z@ plus\fontdimen3\font minus\fontdimen4\font
            \discretionary{\copy\Wrappedvisiblespacebox}{\Wrappedafterbreak}
            {\kern\fontdimen2\font}%
        }%

        % Allow breaks at special characters using \PYG... macros.
        \Wrappedbreaksatspecials
        % Breaks at punctuation characters . , ; ? ! and / need catcode=\active
        \OriginalVerbatim[#1,codes*=\Wrappedbreaksatpunct]%
    }
    \makeatother

    % Exact colors from NB
    \definecolor{incolor}{HTML}{303F9F}
    \definecolor{outcolor}{HTML}{D84315}
    \definecolor{cellborder}{HTML}{CFCFCF}
    \definecolor{cellbackground}{HTML}{F7F7F7}

    % prompt
    \makeatletter
    \newcommand{\boxspacing}{\kern\kvtcb@left@rule\kern\kvtcb@boxsep}
    \makeatother
    \newcommand{\prompt}[4]{
        {\ttfamily\llap{{\color{#2}[#3]:\hspace{3pt}#4}}\vspace{-\baselineskip}}
    }
    

    
    % Prevent overflowing lines due to hard-to-break entities
    \sloppy
    % Setup hyperref package
    \hypersetup{
      breaklinks=true,  % so long urls are correctly broken across lines
      colorlinks=true,
      urlcolor=urlcolor,
      linkcolor=linkcolor,
      citecolor=citecolor,
      }
    % Slightly bigger margins than the latex defaults
    
    \geometry{verbose,tmargin=1in,bmargin=1in,lmargin=1in,rmargin=1in}
    
    

\begin{document}
    
    \maketitle
    
    

    
    \hypertarget{project-3a}{%
\section{Project 3a}\label{project-3a}}

The final part of the project will ask you to perform your own data
science project to classify a new dataset.

\hypertarget{submission-details}{%
\subsection{Submission Details}\label{submission-details}}

\textbf{Project is due June 14th at 11:59 pm (Friday Midnight). To
submit the project, please save the notebook as a pdf file and submit
the assignment via Gradescope. In addition, make sure that all figures
are legible and sufficiently large. For best pdf results, we recommend
printing the notebook using
\href{https://www.latex-project.org/}{\(\LaTeX\)}}

    \hypertarget{loading-essentials-and-helper-functions}{%
\subsection{Loading Essentials and Helper
Functions}\label{loading-essentials-and-helper-functions}}

    \begin{tcolorbox}[breakable, size=fbox, boxrule=1pt, pad at break*=1mm,colback=cellbackground, colframe=cellborder]
\prompt{In}{incolor}{1}{\boxspacing}
\begin{Verbatim}[commandchars=\\\{\}]
\PY{c+c1}{\PYZsh{} fix for windows memory leak with MKL}
\PY{k+kn}{import} \PY{n+nn}{os}
\PY{k+kn}{import} \PY{n+nn}{platform}

\PY{k}{if} \PY{n}{platform}\PY{o}{.}\PY{n}{system}\PY{p}{(}\PY{p}{)} \PY{o}{==} \PY{l+s+s2}{\PYZdq{}}\PY{l+s+s2}{Windows}\PY{l+s+s2}{\PYZdq{}}\PY{p}{:}
    \PY{n}{os}\PY{o}{.}\PY{n}{environ}\PY{p}{[}\PY{l+s+s2}{\PYZdq{}}\PY{l+s+s2}{OMP\PYZus{}NUM\PYZus{}THREADS}\PY{l+s+s2}{\PYZdq{}}\PY{p}{]} \PY{o}{=} \PY{l+s+s2}{\PYZdq{}}\PY{l+s+s2}{2}\PY{l+s+s2}{\PYZdq{}}
\end{Verbatim}
\end{tcolorbox}

    \begin{tcolorbox}[breakable, size=fbox, boxrule=1pt, pad at break*=1mm,colback=cellbackground, colframe=cellborder]
\prompt{In}{incolor}{2}{\boxspacing}
\begin{Verbatim}[commandchars=\\\{\}]
\PY{c+c1}{\PYZsh{}Connecting to Google Drive}
\PY{k+kn}{from} \PY{n+nn}{google}\PY{n+nn}{.}\PY{n+nn}{colab} \PY{k+kn}{import} \PY{n}{drive}
\PY{n}{drive}\PY{o}{.}\PY{n}{mount}\PY{p}{(}\PY{l+s+s1}{\PYZsq{}}\PY{l+s+s1}{/content/drive}\PY{l+s+s1}{\PYZsq{}}\PY{p}{)}
\end{Verbatim}
\end{tcolorbox}

    \begin{Verbatim}[commandchars=\\\{\}]
Mounted at /content/drive
    \end{Verbatim}

    \begin{tcolorbox}[breakable, size=fbox, boxrule=1pt, pad at break*=1mm,colback=cellbackground, colframe=cellborder]
\prompt{In}{incolor}{3}{\boxspacing}
\begin{Verbatim}[commandchars=\\\{\}]
\PY{o}{\PYZpc{}}\PY{k}{cd} \PYZsq{}/content/drive/MyDrive/CM148\PYZus{}Project3\PYZsq{}
\end{Verbatim}
\end{tcolorbox}

    \begin{Verbatim}[commandchars=\\\{\}]
/content/drive/MyDrive/CM148\_Project3
    \end{Verbatim}

    \begin{tcolorbox}[breakable, size=fbox, boxrule=1pt, pad at break*=1mm,colback=cellbackground, colframe=cellborder]
\prompt{In}{incolor}{4}{\boxspacing}
\begin{Verbatim}[commandchars=\\\{\}]
\PY{c+c1}{\PYZsh{} import libraries}
\PY{k+kn}{import} \PY{n+nn}{time}
\PY{k+kn}{import} \PY{n+nn}{random}
\PY{k+kn}{import} \PY{n+nn}{numpy} \PY{k}{as} \PY{n+nn}{np}  \PY{c+c1}{\PYZsh{} linear algebra}
\PY{k+kn}{import} \PY{n+nn}{pandas} \PY{k}{as} \PY{n+nn}{pd}  \PY{c+c1}{\PYZsh{} data processing, CSV file I/O (e.g. pd.read\PYZus{}csv)}
\PY{k+kn}{import} \PY{n+nn}{matplotlib}\PY{n+nn}{.}\PY{n+nn}{pyplot} \PY{k}{as} \PY{n+nn}{plt}  \PY{c+c1}{\PYZsh{} this is used for the plot the graph}

\PY{c+c1}{\PYZsh{} Sklearn classes}
\PY{k+kn}{from} \PY{n+nn}{sklearn}\PY{n+nn}{.}\PY{n+nn}{model\PYZus{}selection} \PY{k+kn}{import} \PY{p}{(}
    \PY{n}{train\PYZus{}test\PYZus{}split}\PY{p}{,}
    \PY{n}{cross\PYZus{}val\PYZus{}score}\PY{p}{,}
    \PY{n}{GridSearchCV}\PY{p}{,}
    \PY{n}{KFold}\PY{p}{,}
\PY{p}{)}
\PY{k+kn}{from} \PY{n+nn}{sklearn} \PY{k+kn}{import} \PY{n}{metrics}
\PY{k+kn}{from} \PY{n+nn}{sklearn}\PY{n+nn}{.}\PY{n+nn}{metrics} \PY{k+kn}{import} \PY{n}{confusion\PYZus{}matrix}\PY{p}{,} \PY{n}{silhouette\PYZus{}score}
\PY{k+kn}{import} \PY{n+nn}{sklearn}\PY{n+nn}{.}\PY{n+nn}{metrics}\PY{n+nn}{.}\PY{n+nn}{cluster} \PY{k}{as} \PY{n+nn}{smc}
\PY{k+kn}{from} \PY{n+nn}{sklearn}\PY{n+nn}{.}\PY{n+nn}{cluster} \PY{k+kn}{import} \PY{n}{KMeans}
\PY{k+kn}{from} \PY{n+nn}{sklearn}\PY{n+nn}{.}\PY{n+nn}{tree} \PY{k+kn}{import} \PY{n}{DecisionTreeClassifier}\PY{p}{,} \PY{n}{export\PYZus{}text}
\PY{k+kn}{from} \PY{n+nn}{sklearn}\PY{n+nn}{.}\PY{n+nn}{pipeline} \PY{k+kn}{import} \PY{n}{Pipeline}\PY{p}{,} \PY{n}{FeatureUnion}
\PY{k+kn}{from} \PY{n+nn}{sklearn}\PY{n+nn}{.}\PY{n+nn}{preprocessing} \PY{k+kn}{import} \PY{p}{(}
    \PY{n}{StandardScaler}\PY{p}{,}
    \PY{n}{OneHotEncoder}\PY{p}{,}
    \PY{n}{LabelEncoder}\PY{p}{,}
    \PY{n}{MinMaxScaler}\PY{p}{,}
\PY{p}{)}
\PY{k+kn}{from} \PY{n+nn}{sklearn}\PY{n+nn}{.}\PY{n+nn}{compose} \PY{k+kn}{import} \PY{n}{ColumnTransformer}\PY{p}{,} \PY{n}{make\PYZus{}column\PYZus{}transformer}
\PY{k+kn}{from} \PY{n+nn}{sklearn} \PY{k+kn}{import} \PY{n}{tree}
\PY{k+kn}{from} \PY{n+nn}{sklearn} \PY{k+kn}{import} \PY{n}{datasets}
\PY{k+kn}{from} \PY{n+nn}{sklearn}\PY{n+nn}{.}\PY{n+nn}{decomposition} \PY{k+kn}{import} \PY{n}{PCA}
\PY{k+kn}{from} \PY{n+nn}{sklearn}\PY{n+nn}{.}\PY{n+nn}{neural\PYZus{}network} \PY{k+kn}{import} \PY{n}{MLPClassifier}
\PY{k+kn}{from} \PY{n+nn}{sklearn}\PY{n+nn}{.}\PY{n+nn}{datasets} \PY{k+kn}{import} \PY{n}{make\PYZus{}blobs}

\PY{k+kn}{import} \PY{n+nn}{sys}
\PY{n}{sys}\PY{o}{.}\PY{n}{path}\PY{o}{.}\PY{n}{append}\PY{p}{(}\PY{l+s+s1}{\PYZsq{}}\PY{l+s+s1}{/content/drive/MyDrive/CM148\PYZus{}Project3}\PY{l+s+s1}{\PYZsq{}}\PY{p}{)}

\PY{k+kn}{import} \PY{n+nn}{helper} \PY{k}{as} \PY{n+nn}{hel}
\PY{k+kn}{from} \PY{n+nn}{helper} \PY{k+kn}{import} \PY{n}{draw\PYZus{}confusion\PYZus{}matrix}

\PY{o}{\PYZpc{}}\PY{k}{matplotlib} inline
\PY{o}{\PYZpc{}}\PY{k}{config} InlineBackend.figure\PYZus{}format = \PYZsq{}retina\PYZsq{}

\PY{c+c1}{\PYZsh{} Sets random seed for reproducibility}
\PY{n}{SEED} \PY{o}{=} \PY{l+m+mi}{42}
\PY{n}{random}\PY{o}{.}\PY{n}{seed}\PY{p}{(}\PY{n}{SEED}\PY{p}{)}
\end{Verbatim}
\end{tcolorbox}

    \hypertarget{background-dataset-information-recap}{%
\subsection{Background: Dataset Information
(Recap)}\label{background-dataset-information-recap}}

For this exercise we will be using a subset of the UCI Heart Disease
dataset, leveraging the fourteen most commonly used attributes. All
identifying information about the patient has been scrubbed. You will be
asked to classify whether a patient is suffering from heart disease
based on a host of potential medical factors.

The dataset includes 14 columns. The information provided by each column
is as follows:

age: Age in years

sex: (male/female)

cp: Chest pain type (0 = asymptomatic; 1 = atypical angina; 2 =
non-anginal pain; 3 = typical angina)

trestbps: Resting blood pressure (in mm Hg on admission to the hospital)

chol: cholesterol in mg/dl

fbs Fasting blood sugar \textgreater{} 120 mg/dl (1 = true; 0 = false)

restecg: Resting electrocardiographic results (0= showing probable or
definite left ventricular hypertrophy by Estes' criteria; 1 = normal; 2
= having ST-T wave abnormality (T wave inversions and/or ST elevation or
depression of \textgreater{} 0.05 mV))

thalach: Maximum heart rate achieved

exang: Exercise induced angina (1 = yes; 0 = no)

oldpeak: Depression induced by exercise relative to rest

slope: The slope of the peak exercise ST segment (0 = downsloping; 1 =
flat; 2 = upsloping)

ca: Number of major vessels (0-3) colored by flourosopy

thal: 1 = normal; 2 = fixed defect; 7 = reversable defect

sick: Indicates the presence of Heart disease (True = Disease; False =
No disease)

    \hypertarget{preprocess-data}{%
\subsection{Preprocess Data}\label{preprocess-data}}

This part is done for you since you would have already completed it in
project 2. Use the train, target, test, and target\_test for all future
parts. We also provide the column names for each transformed column for
future use.

    \begin{tcolorbox}[breakable, size=fbox, boxrule=1pt, pad at break*=1mm,colback=cellbackground, colframe=cellborder]
\prompt{In}{incolor}{5}{\boxspacing}
\begin{Verbatim}[commandchars=\\\{\}]
\PY{c+c1}{\PYZsh{} Preprocess Data}

\PY{c+c1}{\PYZsh{} Load Data}
\PY{n}{data} \PY{o}{=} \PY{n}{pd}\PY{o}{.}\PY{n}{read\PYZus{}csv}\PY{p}{(}\PY{l+s+s2}{\PYZdq{}}\PY{l+s+s2}{/content/drive/MyDrive/CM148\PYZus{}Project3/datasets/heartdisease.csv}\PY{l+s+s2}{\PYZdq{}}\PY{p}{)}

\PY{c+c1}{\PYZsh{} Transform target feature into numerical}
\PY{n}{le} \PY{o}{=} \PY{n}{LabelEncoder}\PY{p}{(}\PY{p}{)}
\PY{n}{data}\PY{p}{[}\PY{l+s+s2}{\PYZdq{}}\PY{l+s+s2}{target}\PY{l+s+s2}{\PYZdq{}}\PY{p}{]} \PY{o}{=} \PY{n}{le}\PY{o}{.}\PY{n}{fit\PYZus{}transform}\PY{p}{(}\PY{n}{data}\PY{p}{[}\PY{l+s+s2}{\PYZdq{}}\PY{l+s+s2}{sick}\PY{l+s+s2}{\PYZdq{}}\PY{p}{]}\PY{p}{)}
\PY{n}{data}\PY{p}{[}\PY{l+s+s2}{\PYZdq{}}\PY{l+s+s2}{sex}\PY{l+s+s2}{\PYZdq{}}\PY{p}{]} \PY{o}{=} \PY{n}{le}\PY{o}{.}\PY{n}{fit\PYZus{}transform}\PY{p}{(}\PY{n}{data}\PY{p}{[}\PY{l+s+s2}{\PYZdq{}}\PY{l+s+s2}{sex}\PY{l+s+s2}{\PYZdq{}}\PY{p}{]}\PY{p}{)}
\PY{n}{data} \PY{o}{=} \PY{n}{data}\PY{o}{.}\PY{n}{drop}\PY{p}{(}\PY{p}{[}\PY{l+s+s2}{\PYZdq{}}\PY{l+s+s2}{sick}\PY{l+s+s2}{\PYZdq{}}\PY{p}{]}\PY{p}{,} \PY{n}{axis}\PY{o}{=}\PY{l+m+mi}{1}\PY{p}{)}

\PY{c+c1}{\PYZsh{} Split target and data}
\PY{n}{y} \PY{o}{=} \PY{n}{data}\PY{p}{[}\PY{l+s+s2}{\PYZdq{}}\PY{l+s+s2}{target}\PY{l+s+s2}{\PYZdq{}}\PY{p}{]}
\PY{n}{x} \PY{o}{=} \PY{n}{data}\PY{o}{.}\PY{n}{drop}\PY{p}{(}\PY{p}{[}\PY{l+s+s2}{\PYZdq{}}\PY{l+s+s2}{target}\PY{l+s+s2}{\PYZdq{}}\PY{p}{]}\PY{p}{,} \PY{n}{axis}\PY{o}{=}\PY{l+m+mi}{1}\PY{p}{)}
\PY{n}{original\PYZus{}feature\PYZus{}names} \PY{o}{=} \PY{n}{x}\PY{o}{.}\PY{n}{columns}\PY{o}{.}\PY{n}{tolist}\PY{p}{(}\PY{p}{)}

\PY{c+c1}{\PYZsh{} Train test split}
\PY{c+c1}{\PYZsh{} 40\PYZpc{} in test data as was in project 2}
\PY{n}{train\PYZus{}raw}\PY{p}{,} \PY{n}{test\PYZus{}raw}\PY{p}{,} \PY{n}{target}\PY{p}{,} \PY{n}{target\PYZus{}test} \PY{o}{=} \PY{n}{train\PYZus{}test\PYZus{}split}\PY{p}{(}
    \PY{n}{x}\PY{p}{,} \PY{n}{y}\PY{p}{,} \PY{n}{test\PYZus{}size}\PY{o}{=}\PY{l+m+mf}{0.4}\PY{p}{,} \PY{n}{stratify}\PY{o}{=}\PY{n}{y}\PY{p}{,} \PY{n}{random\PYZus{}state}\PY{o}{=}\PY{l+m+mi}{0}
\PY{p}{)}

\PY{c+c1}{\PYZsh{} Feature Transformation}
\PY{c+c1}{\PYZsh{} This is the only change from project 2 since we replaced standard scaler to minmax}
\PY{c+c1}{\PYZsh{} This was done to ensure that the numerical features were still of the same scale}
\PY{c+c1}{\PYZsh{} as the one hot encoded features}
\PY{n}{num\PYZus{}pipeline} \PY{o}{=} \PY{n}{Pipeline}\PY{p}{(}\PY{p}{[}\PY{p}{(}\PY{l+s+s2}{\PYZdq{}}\PY{l+s+s2}{minmax}\PY{l+s+s2}{\PYZdq{}}\PY{p}{,} \PY{n}{MinMaxScaler}\PY{p}{(}\PY{p}{)}\PY{p}{)}\PY{p}{]}\PY{p}{)}

\PY{n}{heart\PYZus{}num} \PY{o}{=} \PY{n}{train\PYZus{}raw}\PY{o}{.}\PY{n}{drop}\PY{p}{(}
    \PY{p}{[}\PY{l+s+s2}{\PYZdq{}}\PY{l+s+s2}{sex}\PY{l+s+s2}{\PYZdq{}}\PY{p}{,} \PY{l+s+s2}{\PYZdq{}}\PY{l+s+s2}{cp}\PY{l+s+s2}{\PYZdq{}}\PY{p}{,} \PY{l+s+s2}{\PYZdq{}}\PY{l+s+s2}{fbs}\PY{l+s+s2}{\PYZdq{}}\PY{p}{,} \PY{l+s+s2}{\PYZdq{}}\PY{l+s+s2}{restecg}\PY{l+s+s2}{\PYZdq{}}\PY{p}{,} \PY{l+s+s2}{\PYZdq{}}\PY{l+s+s2}{exang}\PY{l+s+s2}{\PYZdq{}}\PY{p}{,} \PY{l+s+s2}{\PYZdq{}}\PY{l+s+s2}{slope}\PY{l+s+s2}{\PYZdq{}}\PY{p}{,} \PY{l+s+s2}{\PYZdq{}}\PY{l+s+s2}{ca}\PY{l+s+s2}{\PYZdq{}}\PY{p}{,} \PY{l+s+s2}{\PYZdq{}}\PY{l+s+s2}{thal}\PY{l+s+s2}{\PYZdq{}}\PY{p}{]}\PY{p}{,} \PY{n}{axis}\PY{o}{=}\PY{l+m+mi}{1}
\PY{p}{)}
\PY{n}{numerical\PYZus{}features} \PY{o}{=} \PY{n+nb}{list}\PY{p}{(}\PY{n}{heart\PYZus{}num}\PY{p}{)}
\PY{n}{categorical\PYZus{}features} \PY{o}{=} \PY{p}{[}\PY{l+s+s2}{\PYZdq{}}\PY{l+s+s2}{sex}\PY{l+s+s2}{\PYZdq{}}\PY{p}{,} \PY{l+s+s2}{\PYZdq{}}\PY{l+s+s2}{cp}\PY{l+s+s2}{\PYZdq{}}\PY{p}{,} \PY{l+s+s2}{\PYZdq{}}\PY{l+s+s2}{fbs}\PY{l+s+s2}{\PYZdq{}}\PY{p}{,} \PY{l+s+s2}{\PYZdq{}}\PY{l+s+s2}{restecg}\PY{l+s+s2}{\PYZdq{}}\PY{p}{,} \PY{l+s+s2}{\PYZdq{}}\PY{l+s+s2}{exang}\PY{l+s+s2}{\PYZdq{}}\PY{p}{,} \PY{l+s+s2}{\PYZdq{}}\PY{l+s+s2}{slope}\PY{l+s+s2}{\PYZdq{}}\PY{p}{,} \PY{l+s+s2}{\PYZdq{}}\PY{l+s+s2}{ca}\PY{l+s+s2}{\PYZdq{}}\PY{p}{,} \PY{l+s+s2}{\PYZdq{}}\PY{l+s+s2}{thal}\PY{l+s+s2}{\PYZdq{}}\PY{p}{]}

\PY{n}{full\PYZus{}pipeline} \PY{o}{=} \PY{n}{ColumnTransformer}\PY{p}{(}
    \PY{p}{[}
        \PY{p}{(}\PY{l+s+s2}{\PYZdq{}}\PY{l+s+s2}{num}\PY{l+s+s2}{\PYZdq{}}\PY{p}{,} \PY{n}{num\PYZus{}pipeline}\PY{p}{,} \PY{n}{numerical\PYZus{}features}\PY{p}{)}\PY{p}{,}
        \PY{p}{(}\PY{l+s+s2}{\PYZdq{}}\PY{l+s+s2}{cat}\PY{l+s+s2}{\PYZdq{}}\PY{p}{,} \PY{n}{OneHotEncoder}\PY{p}{(}\PY{n}{categories}\PY{o}{=}\PY{l+s+s2}{\PYZdq{}}\PY{l+s+s2}{auto}\PY{l+s+s2}{\PYZdq{}}\PY{p}{)}\PY{p}{,} \PY{n}{categorical\PYZus{}features}\PY{p}{)}\PY{p}{,}
    \PY{p}{]}
\PY{p}{)}

\PY{c+c1}{\PYZsh{} Transform raw data}
\PY{n}{train} \PY{o}{=} \PY{n}{full\PYZus{}pipeline}\PY{o}{.}\PY{n}{fit\PYZus{}transform}\PY{p}{(}\PY{n}{train\PYZus{}raw}\PY{p}{)}
\PY{n}{test} \PY{o}{=} \PY{n}{full\PYZus{}pipeline}\PY{o}{.}\PY{n}{transform}\PY{p}{(}\PY{n}{test\PYZus{}raw}\PY{p}{)}  \PY{c+c1}{\PYZsh{} Note that there is no fit calls}

\PY{c+c1}{\PYZsh{} Extracts features names for each transformed column}
\PY{n}{feature\PYZus{}names} \PY{o}{=} \PY{n}{full\PYZus{}pipeline}\PY{o}{.}\PY{n}{get\PYZus{}feature\PYZus{}names\PYZus{}out}\PY{p}{(}\PY{n+nb}{list}\PY{p}{(}\PY{n}{x}\PY{o}{.}\PY{n}{columns}\PY{p}{)}\PY{p}{)}
\end{Verbatim}
\end{tcolorbox}

    \begin{tcolorbox}[breakable, size=fbox, boxrule=1pt, pad at break*=1mm,colback=cellbackground, colframe=cellborder]
\prompt{In}{incolor}{6}{\boxspacing}
\begin{Verbatim}[commandchars=\\\{\}]
\PY{n+nb}{print}\PY{p}{(}\PY{l+s+s2}{\PYZdq{}}\PY{l+s+s2}{Column names after transformation by pipeline: }\PY{l+s+s2}{\PYZdq{}}\PY{p}{,} \PY{n}{feature\PYZus{}names}\PY{p}{)}
\end{Verbatim}
\end{tcolorbox}

    \begin{Verbatim}[commandchars=\\\{\}]
Column names after transformation by pipeline:  ['num\_\_age' 'num\_\_trestbps'
'num\_\_chol' 'num\_\_thalach' 'num\_\_oldpeak'
 'cat\_\_sex\_0' 'cat\_\_sex\_1' 'cat\_\_cp\_0' 'cat\_\_cp\_1' 'cat\_\_cp\_2' 'cat\_\_cp\_3'
 'cat\_\_fbs\_0' 'cat\_\_fbs\_1' 'cat\_\_restecg\_0' 'cat\_\_restecg\_1'
 'cat\_\_restecg\_2' 'cat\_\_exang\_0' 'cat\_\_exang\_1' 'cat\_\_slope\_0'
 'cat\_\_slope\_1' 'cat\_\_slope\_2' 'cat\_\_ca\_0' 'cat\_\_ca\_1' 'cat\_\_ca\_2'
 'cat\_\_ca\_3' 'cat\_\_ca\_4' 'cat\_\_thal\_0' 'cat\_\_thal\_1' 'cat\_\_thal\_2'
 'cat\_\_thal\_3']
    \end{Verbatim}

    The following shows the baseline accuracy of simply classifying every
sample as the majority class.

    \begin{tcolorbox}[breakable, size=fbox, boxrule=1pt, pad at break*=1mm,colback=cellbackground, colframe=cellborder]
\prompt{In}{incolor}{7}{\boxspacing}
\begin{Verbatim}[commandchars=\\\{\}]
\PY{c+c1}{\PYZsh{} Baseline accuracy of using the majority class}
\PY{n}{ct} \PY{o}{=} \PY{n}{target\PYZus{}test}\PY{o}{.}\PY{n}{value\PYZus{}counts}\PY{p}{(}\PY{p}{)}
\PY{n+nb}{print}\PY{p}{(}\PY{l+s+s2}{\PYZdq{}}\PY{l+s+s2}{Counts of each class in target\PYZus{}test: }\PY{l+s+s2}{\PYZdq{}}\PY{p}{)}
\PY{n+nb}{print}\PY{p}{(}\PY{n}{ct}\PY{p}{)}
\PY{n+nb}{print}\PY{p}{(}
    \PY{l+s+s2}{\PYZdq{}}\PY{l+s+s2}{==========================================}\PY{l+s+s2}{\PYZdq{}}\PY{p}{,}
    \PY{l+s+s2}{\PYZdq{}}\PY{l+s+se}{\PYZbs{}n}\PY{l+s+s2}{Baseline Accuraccy of using Majority Class:}\PY{l+s+s2}{\PYZdq{}}\PY{p}{,}
    \PY{n}{np}\PY{o}{.}\PY{n}{round}\PY{p}{(}\PY{n}{np}\PY{o}{.}\PY{n}{max}\PY{p}{(}\PY{n}{ct}\PY{p}{)} \PY{o}{/} \PY{n}{np}\PY{o}{.}\PY{n}{sum}\PY{p}{(}\PY{n}{ct}\PY{p}{)}\PY{p}{,} \PY{l+m+mi}{3}\PY{p}{)}\PY{p}{,}
\PY{p}{)}
\end{Verbatim}
\end{tcolorbox}

    \begin{Verbatim}[commandchars=\\\{\}]
Counts of each class in target\_test:
target
0    66
1    56
Name: count, dtype: int64
==========================================
Baseline Accuraccy of using Majority Class: 0.541
    \end{Verbatim}

    \hypertarget{pts-decision-trees}{%
\subsection{1. (25 pts) Decision Trees}\label{pts-decision-trees}}

    \hypertarget{pts-apply-decision-tree-on-train-data}{%
\subsubsection{1.1. {[}5 pts{]} Apply Decision Tree on Train
Data}\label{pts-apply-decision-tree-on-train-data}}

Apply the decision tree on the \textbf{train data} with default
parameters of the DecisionTreeClassifier. \textbf{Report the accuracy
and print the confusion matrix}. Make sure to use
\texttt{random\_state\ =\ SEED} so that your results match ours.

    \begin{tcolorbox}[breakable, size=fbox, boxrule=1pt, pad at break*=1mm,colback=cellbackground, colframe=cellborder]
\prompt{In}{incolor}{8}{\boxspacing}
\begin{Verbatim}[commandchars=\\\{\}]
\PY{c+c1}{\PYZsh{} TODO}
\PY{n}{decision\PYZus{}tree} \PY{o}{=} \PY{n}{DecisionTreeClassifier}\PY{p}{(}\PY{n}{random\PYZus{}state}\PY{o}{=}\PY{n}{SEED}\PY{p}{)}
\PY{n}{decision\PYZus{}tree}\PY{o}{.}\PY{n}{fit}\PY{p}{(}\PY{n}{train\PYZus{}raw}\PY{p}{,} \PY{n}{target}\PY{p}{)}
\PY{n}{prediction} \PY{o}{=} \PY{n}{decision\PYZus{}tree}\PY{o}{.}\PY{n}{predict}\PY{p}{(}\PY{n}{test\PYZus{}raw}\PY{p}{)}
\PY{n}{acc}\PY{o}{=}\PY{n}{metrics}\PY{o}{.}\PY{n}{accuracy\PYZus{}score}\PY{p}{(}\PY{n}{target\PYZus{}test}\PY{p}{,} \PY{n}{prediction}\PY{p}{)}
\PY{n}{acc}
\end{Verbatim}
\end{tcolorbox}

            \begin{tcolorbox}[breakable, size=fbox, boxrule=.5pt, pad at break*=1mm, opacityfill=0]
\prompt{Out}{outcolor}{8}{\boxspacing}
\begin{Verbatim}[commandchars=\\\{\}]
0.7131147540983607
\end{Verbatim}
\end{tcolorbox}
        
    \begin{tcolorbox}[breakable, size=fbox, boxrule=1pt, pad at break*=1mm,colback=cellbackground, colframe=cellborder]
\prompt{In}{incolor}{9}{\boxspacing}
\begin{Verbatim}[commandchars=\\\{\}]
\PY{c+c1}{\PYZsh{}Printing confusion matrix}
\PY{n}{hel}\PY{o}{.}\PY{n}{draw\PYZus{}confusion\PYZus{}matrix}\PY{p}{(}\PY{n}{target\PYZus{}test}\PY{p}{,} \PY{n}{prediction}\PY{p}{,} \PY{p}{[}\PY{l+s+s2}{\PYZdq{}}\PY{l+s+s2}{Not Sick (0)}\PY{l+s+s2}{\PYZdq{}}\PY{p}{,} \PY{l+s+s2}{\PYZdq{}}\PY{l+s+s2}{Sick(1)}\PY{l+s+s2}{\PYZdq{}}\PY{p}{]}\PY{p}{)}
\end{Verbatim}
\end{tcolorbox}

    \begin{center}
    \adjustimage{max size={0.9\linewidth}{0.9\paperheight}}{Project_3a_files/Project_3a_15_0.png}
    \end{center}
    { \hspace*{\fill} \\}
    
    \hypertarget{pts-visualize-the-decision-tree}{%
\subsubsection{1.2. {[}5 pts{]} Visualize the Decision
Tree}\label{pts-visualize-the-decision-tree}}

Visualize the first two layers of the decision tree that you trained.

    \begin{tcolorbox}[breakable, size=fbox, boxrule=1pt, pad at break*=1mm,colback=cellbackground, colframe=cellborder]
\prompt{In}{incolor}{10}{\boxspacing}
\begin{Verbatim}[commandchars=\\\{\}]
\PY{c+c1}{\PYZsh{} TODO}
\PY{n}{plt}\PY{o}{.}\PY{n}{figure}\PY{p}{(}\PY{n}{figsize}\PY{o}{=}\PY{p}{(}\PY{l+m+mi}{10}\PY{p}{,}\PY{l+m+mi}{8}\PY{p}{)}\PY{p}{)}
\PY{n}{tree}\PY{o}{.}\PY{n}{plot\PYZus{}tree}\PY{p}{(}\PY{n}{decision\PYZus{}tree}\PY{p}{,} \PY{n}{filled}\PY{o}{=}\PY{k+kc}{True}\PY{p}{,} \PY{n}{max\PYZus{}depth}\PY{o}{=}\PY{l+m+mi}{2}\PY{p}{,} \PY{n}{rounded}\PY{o}{=}\PY{k+kc}{True}\PY{p}{,} \PY{n}{feature\PYZus{}names}\PY{o}{=}\PY{n}{feature\PYZus{}names}\PY{p}{)}
\PY{n}{plt}\PY{o}{.}\PY{n}{show}\PY{p}{(}\PY{p}{)}
\end{Verbatim}
\end{tcolorbox}

    \begin{center}
    \adjustimage{max size={0.9\linewidth}{0.9\paperheight}}{Project_3a_files/Project_3a_17_0.png}
    \end{center}
    { \hspace*{\fill} \\}
    
    \textbf{What is the gini index improvement of the first split?}

    Response:

Gini index of parent = 0.496 Gini index of left child = 0.403 Gini index
of right child = 0.283

\[ \text{Gini Index Improvement} = 0.496 - \frac{93}{181}\cdot 0.403 - \frac{88}{181}\cdot 0.283 = 0.1513425414  \approx 0.151\]

    \hypertarget{pts-plot-the-importance-of-each-feature-for-the-decision-tree}{%
\subsubsection{1.3 {[}5 pts{]} Plot the importance of each feature for
the Decision
Tree}\label{pts-plot-the-importance-of-each-feature-for-the-decision-tree}}

    \begin{tcolorbox}[breakable, size=fbox, boxrule=1pt, pad at break*=1mm,colback=cellbackground, colframe=cellborder]
\prompt{In}{incolor}{11}{\boxspacing}
\begin{Verbatim}[commandchars=\\\{\}]
\PY{c+c1}{\PYZsh{} TODO}
\PY{n}{feature\PYZus{}importances} \PY{o}{=} \PY{n}{decision\PYZus{}tree}\PY{o}{.}\PY{n}{feature\PYZus{}importances\PYZus{}}
\PY{n}{plt}\PY{o}{.}\PY{n}{figure}\PY{p}{(}\PY{n}{figsize}\PY{o}{=}\PY{p}{(}\PY{l+m+mi}{12}\PY{p}{,}\PY{l+m+mi}{4}\PY{p}{)}\PY{p}{)}
\PY{n}{plt}\PY{o}{.}\PY{n}{bar}\PY{p}{(}\PY{n}{original\PYZus{}feature\PYZus{}names}\PY{p}{,} \PY{n}{feature\PYZus{}importances}\PY{p}{)}
\PY{n}{plt}\PY{o}{.}\PY{n}{xlabel}\PY{p}{(}\PY{l+s+s1}{\PYZsq{}}\PY{l+s+s1}{Feature Name}\PY{l+s+s1}{\PYZsq{}}\PY{p}{)}
\PY{n}{plt}\PY{o}{.}\PY{n}{ylabel}\PY{p}{(}\PY{l+s+s1}{\PYZsq{}}\PY{l+s+s1}{Importance}\PY{l+s+s1}{\PYZsq{}}\PY{p}{)}
\PY{n}{plt}\PY{o}{.}\PY{n}{title}\PY{p}{(}\PY{l+s+s2}{\PYZdq{}}\PY{l+s+s2}{Importance of Features in Decision Tree}\PY{l+s+s2}{\PYZdq{}}\PY{p}{)}
\PY{n}{plt}\PY{o}{.}\PY{n}{show}\PY{p}{(}\PY{p}{)}
\end{Verbatim}
\end{tcolorbox}

    \begin{center}
    \adjustimage{max size={0.9\linewidth}{0.9\paperheight}}{Project_3a_files/Project_3a_21_0.png}
    \end{center}
    { \hspace*{\fill} \\}
    
    \textbf{How many features have non-zero importance for the Decision
Tree? If we remove the features with zero importance, will it change the
decision tree for the same sampled dataset?}

    Response: All except 1 have non-zero importance for the decision tree.
If we remove features with zero importance, it is unlikely to change the
decision tree for the same sampled dataset since these features are not
very significant in the decision tree formation.

    \hypertarget{pts-optimize-decision-tree}{%
\subsubsection{1.4 {[}10 pts{]} Optimize Decision
Tree}\label{pts-optimize-decision-tree}}

While the default Decision Tree performs fairly well on the data, lets
see if we can improve performance by optimizing the parameters.

Run a \texttt{GridSearchCV} with 5-Fold Cross Validation for the
Decision Tree. Find the best model parameters for accuracy amongst the
following:

\begin{itemize}
\tightlist
\item
  \texttt{max\_depth} = {[}2, 4, 8, 16, 32{]}
\item
  \texttt{min\_samples\_split} = {[}2, 4, 8, 16{]}
\item
  \texttt{criterion} = {[}\texttt{gini}, \texttt{entropy}{]}
\end{itemize}

After using \texttt{GridSearchCV}, Print the \textbf{best 5 models} with
the following parameters: \texttt{rank\_test\_score},
\texttt{param\_max\_depth}, \texttt{param\_min\_samples\_split},
\texttt{param\_criterion}, \texttt{mean\_test\_score},
\texttt{std\_test\_score}.

    \begin{tcolorbox}[breakable, size=fbox, boxrule=1pt, pad at break*=1mm,colback=cellbackground, colframe=cellborder]
\prompt{In}{incolor}{12}{\boxspacing}
\begin{Verbatim}[commandchars=\\\{\}]
\PY{c+c1}{\PYZsh{} TODO}
\PY{c+c1}{\PYZsh{}DOUBT: Do we set random state to seed here?}
\PY{n}{parameters} \PY{o}{=} \PY{p}{[}
    \PY{p}{\PYZob{}}
        \PY{l+s+s2}{\PYZdq{}}\PY{l+s+s2}{max\PYZus{}depth}\PY{l+s+s2}{\PYZdq{}}\PY{p}{:} \PY{p}{[}\PY{l+m+mi}{2}\PY{p}{,}\PY{l+m+mi}{4}\PY{p}{,}\PY{l+m+mi}{8}\PY{p}{,}\PY{l+m+mi}{16}\PY{p}{,}\PY{l+m+mi}{32}\PY{p}{]}\PY{p}{,}
        \PY{l+s+s2}{\PYZdq{}}\PY{l+s+s2}{min\PYZus{}samples\PYZus{}split}\PY{l+s+s2}{\PYZdq{}}\PY{p}{:} \PY{p}{[}\PY{l+m+mi}{2}\PY{p}{,}\PY{l+m+mi}{4}\PY{p}{,}\PY{l+m+mi}{8}\PY{p}{,}\PY{l+m+mi}{16}\PY{p}{]}\PY{p}{,}
        \PY{l+s+s2}{\PYZdq{}}\PY{l+s+s2}{criterion}\PY{l+s+s2}{\PYZdq{}}\PY{p}{:} \PY{p}{[}\PY{l+s+s2}{\PYZdq{}}\PY{l+s+s2}{gini}\PY{l+s+s2}{\PYZdq{}}\PY{p}{,} \PY{l+s+s2}{\PYZdq{}}\PY{l+s+s2}{entropy}\PY{l+s+s2}{\PYZdq{}}\PY{p}{]}
    \PY{p}{\PYZcb{}}
\PY{p}{]}

\PY{n}{k} \PY{o}{=} \PY{l+m+mi}{5}
\PY{n}{kf} \PY{o}{=} \PY{n}{KFold}\PY{p}{(}\PY{n}{n\PYZus{}splits}\PY{o}{=}\PY{n}{k}\PY{p}{,} \PY{n}{random\PYZus{}state}\PY{o}{=}\PY{k+kc}{None}\PY{p}{)}

\PY{n}{decision\PYZus{}tree} \PY{o}{=} \PY{n}{DecisionTreeClassifier}\PY{p}{(}\PY{p}{)}
\PY{n}{grid} \PY{o}{=} \PY{n}{GridSearchCV}\PY{p}{(}\PY{n}{decision\PYZus{}tree}\PY{p}{,} \PY{n}{parameters}\PY{p}{,} \PY{n}{cv}\PY{o}{=}\PY{n}{kf}\PY{p}{,} \PY{n}{scoring}\PY{o}{=}\PY{l+s+s2}{\PYZdq{}}\PY{l+s+s2}{accuracy}\PY{l+s+s2}{\PYZdq{}}\PY{p}{)}
\PY{n}{grid}\PY{o}{.}\PY{n}{fit}\PY{p}{(}\PY{n}{train\PYZus{}raw}\PY{p}{,} \PY{n}{target}\PY{p}{)}
\PY{n}{res} \PY{o}{=} \PY{n}{pd}\PY{o}{.}\PY{n}{DataFrame}\PY{p}{(}\PY{n}{grid}\PY{o}{.}\PY{n}{cv\PYZus{}results\PYZus{}}\PY{p}{)}
\end{Verbatim}
\end{tcolorbox}

    \begin{tcolorbox}[breakable, size=fbox, boxrule=1pt, pad at break*=1mm,colback=cellbackground, colframe=cellborder]
\prompt{In}{incolor}{13}{\boxspacing}
\begin{Verbatim}[commandchars=\\\{\}]
\PY{n}{res\PYZus{}sorted} \PY{o}{=} \PY{n}{res}\PY{o}{.}\PY{n}{sort\PYZus{}values}\PY{p}{(}\PY{n}{by}\PY{o}{=}\PY{l+s+s1}{\PYZsq{}}\PY{l+s+s1}{mean\PYZus{}test\PYZus{}score}\PY{l+s+s1}{\PYZsq{}}\PY{p}{,} \PY{n}{ascending}\PY{o}{=}\PY{k+kc}{False}\PY{p}{)}
\PY{n}{wanted\PYZus{}params} \PY{o}{=} \PY{p}{[}\PY{l+s+s2}{\PYZdq{}}\PY{l+s+s2}{rank\PYZus{}test\PYZus{}score}\PY{l+s+s2}{\PYZdq{}}\PY{p}{,} \PY{l+s+s2}{\PYZdq{}}\PY{l+s+s2}{param\PYZus{}max\PYZus{}depth}\PY{l+s+s2}{\PYZdq{}}\PY{p}{,} \PY{l+s+s2}{\PYZdq{}}\PY{l+s+s2}{param\PYZus{}min\PYZus{}samples\PYZus{}split}\PY{l+s+s2}{\PYZdq{}}\PY{p}{,} \PY{l+s+s2}{\PYZdq{}}\PY{l+s+s2}{param\PYZus{}criterion}\PY{l+s+s2}{\PYZdq{}}\PY{p}{,} \PY{l+s+s2}{\PYZdq{}}\PY{l+s+s2}{mean\PYZus{}test\PYZus{}score}\PY{l+s+s2}{\PYZdq{}}\PY{p}{,} \PY{l+s+s2}{\PYZdq{}}\PY{l+s+s2}{std\PYZus{}test\PYZus{}score}\PY{l+s+s2}{\PYZdq{}}\PY{p}{]}
\PY{k}{for} \PY{n}{i} \PY{o+ow}{in} \PY{n+nb}{range}\PY{p}{(}\PY{l+m+mi}{4}\PY{p}{)}\PY{p}{:}
  \PY{n+nb}{print}\PY{p}{(}\PY{l+s+s2}{\PYZdq{}}\PY{l+s+s2}{Rank:}\PY{l+s+s2}{\PYZdq{}}\PY{p}{,} \PY{n}{i}\PY{o}{+}\PY{l+m+mi}{1}\PY{p}{)}
  \PY{n+nb}{print}\PY{p}{(}\PY{l+s+s2}{\PYZdq{}}\PY{l+s+s2}{Score:}\PY{l+s+s2}{\PYZdq{}}\PY{p}{,} \PY{n}{res\PYZus{}sorted}\PY{o}{.}\PY{n}{iloc}\PY{p}{[}\PY{n}{i}\PY{p}{]}\PY{p}{[}\PY{l+s+s1}{\PYZsq{}}\PY{l+s+s1}{mean\PYZus{}test\PYZus{}score}\PY{l+s+s1}{\PYZsq{}}\PY{p}{]}\PY{p}{)}
  \PY{n+nb}{print}\PY{p}{(}\PY{l+s+s2}{\PYZdq{}}\PY{l+s+s2}{Parameters:}\PY{l+s+s2}{\PYZdq{}}\PY{p}{,} \PY{n}{res\PYZus{}sorted}\PY{o}{.}\PY{n}{iloc}\PY{p}{[}\PY{l+m+mi}{0}\PY{p}{]}\PY{p}{[}\PY{l+s+s1}{\PYZsq{}}\PY{l+s+s1}{params}\PY{l+s+s1}{\PYZsq{}}\PY{p}{]}\PY{p}{)}
  \PY{k}{for} \PY{n}{p} \PY{o+ow}{in} \PY{n}{wanted\PYZus{}params}\PY{p}{:}
    \PY{n}{value} \PY{o}{=} \PY{n}{res\PYZus{}sorted}\PY{o}{.}\PY{n}{iloc}\PY{p}{[}\PY{n}{i}\PY{p}{]}\PY{p}{[}\PY{n}{p}\PY{p}{]}
    \PY{n+nb}{print}\PY{p}{(}\PY{l+s+sa}{f}\PY{l+s+s2}{\PYZdq{}}\PY{l+s+si}{\PYZob{}}\PY{n}{p}\PY{l+s+si}{\PYZcb{}}\PY{l+s+s2}{:}\PY{l+s+si}{\PYZob{}}\PY{n}{value}\PY{l+s+si}{\PYZcb{}}\PY{l+s+s2}{\PYZdq{}}\PY{p}{)}
  \PY{n+nb}{print}\PY{p}{(}\PY{l+s+s2}{\PYZdq{}}\PY{l+s+s2}{   }\PY{l+s+s2}{\PYZdq{}}\PY{p}{)}
\end{Verbatim}
\end{tcolorbox}

    \begin{Verbatim}[commandchars=\\\{\}]
Rank: 1
Score: 0.7794294294294294
Parameters: \{'criterion': 'gini', 'max\_depth': 32, 'min\_samples\_split': 2\}
rank\_test\_score:1
param\_max\_depth:32
param\_min\_samples\_split:2
param\_criterion:gini
mean\_test\_score:0.7794294294294294
std\_test\_score:0.07274058078537654

Rank: 2
Score: 0.7789789789789789
Parameters: \{'criterion': 'gini', 'max\_depth': 32, 'min\_samples\_split': 2\}
rank\_test\_score:2
param\_max\_depth:16
param\_min\_samples\_split:2
param\_criterion:gini
mean\_test\_score:0.7789789789789789
std\_test\_score:0.08054348526593912

Rank: 3
Score: 0.7680180180180181
Parameters: \{'criterion': 'gini', 'max\_depth': 32, 'min\_samples\_split': 2\}
rank\_test\_score:3
param\_max\_depth:32
param\_min\_samples\_split:4
param\_criterion:entropy
mean\_test\_score:0.7680180180180181
std\_test\_score:0.08141821800752469

Rank: 4
Score: 0.7678678678678679
Parameters: \{'criterion': 'gini', 'max\_depth': 32, 'min\_samples\_split': 2\}
rank\_test\_score:4
param\_max\_depth:4
param\_min\_samples\_split:4
param\_criterion:gini
mean\_test\_score:0.7678678678678679
std\_test\_score:0.07392268706399856

    \end{Verbatim}

    \textbf{Using the best model you have, report the test accuracy and
print out the confusion matrix}

    \begin{tcolorbox}[breakable, size=fbox, boxrule=1pt, pad at break*=1mm,colback=cellbackground, colframe=cellborder]
\prompt{In}{incolor}{14}{\boxspacing}
\begin{Verbatim}[commandchars=\\\{\}]
\PY{c+c1}{\PYZsh{} TODO}
\PY{n}{best\PYZus{}model} \PY{o}{=} \PY{n}{DecisionTreeClassifier}\PY{p}{(}\PY{n}{criterion}\PY{o}{=}\PY{l+s+s1}{\PYZsq{}}\PY{l+s+s1}{gini}\PY{l+s+s1}{\PYZsq{}}\PY{p}{,} \PY{n}{max\PYZus{}depth}\PY{o}{=}\PY{l+m+mi}{32}\PY{p}{,} \PY{n}{min\PYZus{}samples\PYZus{}split}\PY{o}{=}\PY{l+m+mi}{4}\PY{p}{)}
\PY{n}{best\PYZus{}model}\PY{o}{.}\PY{n}{fit}\PY{p}{(}\PY{n}{train\PYZus{}raw}\PY{p}{,} \PY{n}{target}\PY{p}{)}
\PY{n}{predicted} \PY{o}{=} \PY{n}{best\PYZus{}model}\PY{o}{.}\PY{n}{predict}\PY{p}{(}\PY{n}{test\PYZus{}raw}\PY{p}{)}
\PY{n+nb}{print}\PY{p}{(}\PY{l+s+s2}{\PYZdq{}}\PY{l+s+si}{\PYZpc{}\PYZhy{}12s}\PY{l+s+s2}{ }\PY{l+s+si}{\PYZpc{}f}\PY{l+s+s2}{\PYZdq{}} \PY{o}{\PYZpc{}} \PY{p}{(}\PY{l+s+s1}{\PYZsq{}}\PY{l+s+s1}{Accuracy:}\PY{l+s+s1}{\PYZsq{}}\PY{p}{,} \PY{n}{metrics}\PY{o}{.}\PY{n}{accuracy\PYZus{}score}\PY{p}{(}\PY{n}{target\PYZus{}test}\PY{p}{,}\PY{n}{predicted}\PY{p}{)}\PY{p}{)}\PY{p}{)}
\end{Verbatim}
\end{tcolorbox}

    \begin{Verbatim}[commandchars=\\\{\}]
Accuracy:    0.704918
    \end{Verbatim}

    \begin{tcolorbox}[breakable, size=fbox, boxrule=1pt, pad at break*=1mm,colback=cellbackground, colframe=cellborder]
\prompt{In}{incolor}{15}{\boxspacing}
\begin{Verbatim}[commandchars=\\\{\}]
\PY{c+c1}{\PYZsh{}Printing out the confusion matrix}
\PY{n}{hel}\PY{o}{.}\PY{n}{draw\PYZus{}confusion\PYZus{}matrix}\PY{p}{(}\PY{n}{target\PYZus{}test}\PY{p}{,} \PY{n}{predicted}\PY{p}{,} \PY{p}{[}\PY{l+s+s2}{\PYZdq{}}\PY{l+s+s2}{Not Sick (0)}\PY{l+s+s2}{\PYZdq{}}\PY{p}{,} \PY{l+s+s2}{\PYZdq{}}\PY{l+s+s2}{Sick(1)}\PY{l+s+s2}{\PYZdq{}}\PY{p}{]}\PY{p}{)}
\end{Verbatim}
\end{tcolorbox}

    \begin{center}
    \adjustimage{max size={0.9\linewidth}{0.9\paperheight}}{Project_3a_files/Project_3a_29_0.png}
    \end{center}
    { \hspace*{\fill} \\}
    
    \hypertarget{pts-multi-layer-perceptron}{%
\subsection{2. (20 pts) Multi-Layer
Perceptron}\label{pts-multi-layer-perceptron}}

    \hypertarget{pts-applying-a-multi-layer-perceptron}{%
\subsubsection{2.1 {[}5 pts{]} Applying a Multi-Layer
Perceptron}\label{pts-applying-a-multi-layer-perceptron}}

Apply the MLP on the \textbf{train data} with
\texttt{hidden\_layer\_sizes=(50,\ 50)} and \texttt{max\_iter\ =\ 1000}.
\textbf{Report the accuracy and print the confusion matrix}. Make sure
to set \texttt{random\_state=SEED}.

    \begin{tcolorbox}[breakable, size=fbox, boxrule=1pt, pad at break*=1mm,colback=cellbackground, colframe=cellborder]
\prompt{In}{incolor}{16}{\boxspacing}
\begin{Verbatim}[commandchars=\\\{\}]
\PY{c+c1}{\PYZsh{} TODO}
\PY{n}{mlp} \PY{o}{=} \PY{n}{MLPClassifier}\PY{p}{(}\PY{n}{hidden\PYZus{}layer\PYZus{}sizes}\PY{o}{=}\PY{p}{(}\PY{l+m+mi}{50}\PY{p}{,}\PY{l+m+mi}{50}\PY{p}{)}\PY{p}{,} \PY{n}{max\PYZus{}iter}\PY{o}{=}\PY{l+m+mi}{1000}\PY{p}{,} \PY{n}{random\PYZus{}state}\PY{o}{=}\PY{n}{SEED}\PY{p}{)}
\PY{n}{mlp}\PY{o}{.}\PY{n}{fit}\PY{p}{(}\PY{n}{train\PYZus{}raw}\PY{p}{,} \PY{n}{target}\PY{p}{)}
\PY{n}{prediction} \PY{o}{=} \PY{n}{mlp}\PY{o}{.}\PY{n}{predict}\PY{p}{(}\PY{n}{test\PYZus{}raw}\PY{p}{)}
\PY{n+nb}{print}\PY{p}{(}\PY{l+s+s2}{\PYZdq{}}\PY{l+s+si}{\PYZpc{}\PYZhy{}12s}\PY{l+s+s2}{ }\PY{l+s+si}{\PYZpc{}f}\PY{l+s+s2}{\PYZdq{}} \PY{o}{\PYZpc{}} \PY{p}{(}\PY{l+s+s1}{\PYZsq{}}\PY{l+s+s1}{Accuracy:}\PY{l+s+s1}{\PYZsq{}}\PY{p}{,} \PY{n}{metrics}\PY{o}{.}\PY{n}{accuracy\PYZus{}score}\PY{p}{(}\PY{n}{target\PYZus{}test}\PY{p}{,}\PY{n}{prediction}\PY{p}{)}\PY{p}{)}\PY{p}{)}
\end{Verbatim}
\end{tcolorbox}

    \begin{Verbatim}[commandchars=\\\{\}]
Accuracy:    0.803279
    \end{Verbatim}

    \begin{Verbatim}[commandchars=\\\{\}]
/usr/local/lib/python3.10/dist-
packages/sklearn/neural\_network/\_multilayer\_perceptron.py:686:
ConvergenceWarning: Stochastic Optimizer: Maximum iterations (1000) reached and
the optimization hasn't converged yet.
  warnings.warn(
    \end{Verbatim}

    \begin{tcolorbox}[breakable, size=fbox, boxrule=1pt, pad at break*=1mm,colback=cellbackground, colframe=cellborder]
\prompt{In}{incolor}{17}{\boxspacing}
\begin{Verbatim}[commandchars=\\\{\}]
\PY{c+c1}{\PYZsh{}Printing out the confusion matrix}
\PY{n}{hel}\PY{o}{.}\PY{n}{draw\PYZus{}confusion\PYZus{}matrix}\PY{p}{(}\PY{n}{target\PYZus{}test}\PY{p}{,} \PY{n}{prediction}\PY{p}{,} \PY{p}{[}\PY{l+s+s2}{\PYZdq{}}\PY{l+s+s2}{Not Sick (0)}\PY{l+s+s2}{\PYZdq{}}\PY{p}{,} \PY{l+s+s2}{\PYZdq{}}\PY{l+s+s2}{Sick(1)}\PY{l+s+s2}{\PYZdq{}}\PY{p}{]}\PY{p}{)}
\end{Verbatim}
\end{tcolorbox}

    \begin{center}
    \adjustimage{max size={0.9\linewidth}{0.9\paperheight}}{Project_3a_files/Project_3a_33_0.png}
    \end{center}
    { \hspace*{\fill} \\}
    
    \hypertarget{pts-speedtest-between-decision-tree-and-mlp}{%
\subsubsection{2.2 {[}10 pts{]} Speedtest between Decision Tree and
MLP}\label{pts-speedtest-between-decision-tree-and-mlp}}

Let us compare the training times and prediction times of a Decision
Tree and an MLP. \textbf{Time how long it takes for a Decision Tree and
an MLP to perform a .fit operation (i.e.~training the model). Then, time
how long it takes for a Decision Tree and an MLP to perform a .predict
operation (i.e.~predicting the testing data). Print out the timings and
specify which model was quicker for each operation.} We recommend using
the \href{https://docs.python.org/3/library/time.html}{time} python
module to time your code. An example of the time module was shown in
project 2. Use the default Decision Tree Classifier and the MLP with the
previously mentioned parameters.

    \begin{tcolorbox}[breakable, size=fbox, boxrule=1pt, pad at break*=1mm,colback=cellbackground, colframe=cellborder]
\prompt{In}{incolor}{18}{\boxspacing}
\begin{Verbatim}[commandchars=\\\{\}]
\PY{c+c1}{\PYZsh{} TODO}
\PY{c+c1}{\PYZsh{}DOUBT: Do we use random\PYZus{}state=SEED throughout the project or only where it is specified?}
\PY{c+c1}{\PYZsh{}DOUBT: Is it normal to get different results each time we rerun this cell?}
\PY{k+kn}{import} \PY{n+nn}{time}
\PY{n+nb}{print}\PY{p}{(}\PY{l+s+s2}{\PYZdq{}}\PY{l+s+s2}{Decision Tree Classifier}\PY{l+s+s2}{\PYZdq{}}\PY{p}{)}
\PY{n}{dt} \PY{o}{=} \PY{n}{DecisionTreeClassifier}\PY{p}{(}\PY{n}{random\PYZus{}state}\PY{o}{=}\PY{n}{SEED}\PY{p}{)}
\PY{n}{start} \PY{o}{=} \PY{n}{time}\PY{o}{.}\PY{n}{time}\PY{p}{(}\PY{p}{)}
\PY{n}{dt}\PY{o}{.}\PY{n}{fit}\PY{p}{(}\PY{n}{train\PYZus{}raw}\PY{p}{,} \PY{n}{target}\PY{p}{)}
\PY{n}{end} \PY{o}{=} \PY{n}{time}\PY{o}{.}\PY{n}{time}\PY{p}{(}\PY{p}{)}
\PY{n}{time\PYZus{}taken} \PY{o}{=} \PY{p}{(}\PY{n}{end}\PY{o}{\PYZhy{}}\PY{n}{start}\PY{p}{)}\PY{o}{*}\PY{l+m+mi}{1000}
\PY{n+nb}{print}\PY{p}{(}\PY{l+s+s2}{\PYZdq{}}\PY{l+s+s2}{Time taken to fit:}\PY{l+s+s2}{\PYZdq{}}\PY{p}{,} \PY{n+nb}{round}\PY{p}{(}\PY{n}{time\PYZus{}taken}\PY{p}{,}\PY{l+m+mi}{5}\PY{p}{)}\PY{p}{,} \PY{l+s+s2}{\PYZdq{}}\PY{l+s+s2}{ milliseconds}\PY{l+s+s2}{\PYZdq{}}\PY{p}{)}
\PY{n}{start} \PY{o}{=} \PY{n}{time}\PY{o}{.}\PY{n}{time}\PY{p}{(}\PY{p}{)}
\PY{n}{dt}\PY{o}{.}\PY{n}{predict}\PY{p}{(}\PY{n}{test\PYZus{}raw}\PY{p}{)}
\PY{n}{end}\PY{o}{=}\PY{n}{time}\PY{o}{.}\PY{n}{time}\PY{p}{(}\PY{p}{)}
\PY{n}{time\PYZus{}taken} \PY{o}{=} \PY{p}{(}\PY{n}{end}\PY{o}{\PYZhy{}}\PY{n}{start}\PY{p}{)}\PY{o}{*}\PY{l+m+mi}{1000}
\PY{n+nb}{print}\PY{p}{(}\PY{l+s+s2}{\PYZdq{}}\PY{l+s+s2}{Time taken to predict:}\PY{l+s+s2}{\PYZdq{}}\PY{p}{,} \PY{n+nb}{round}\PY{p}{(}\PY{n}{time\PYZus{}taken}\PY{p}{,}\PY{l+m+mi}{5}\PY{p}{)}\PY{p}{,} \PY{l+s+s2}{\PYZdq{}}\PY{l+s+s2}{ milliseconds}\PY{l+s+s2}{\PYZdq{}}\PY{p}{)}

\PY{n+nb}{print}\PY{p}{(}\PY{l+s+s2}{\PYZdq{}}\PY{l+s+s2}{  }\PY{l+s+s2}{\PYZdq{}}\PY{p}{)}

\PY{n+nb}{print}\PY{p}{(}\PY{l+s+s2}{\PYZdq{}}\PY{l+s+s2}{Multi\PYZhy{}Layer Perceptron}\PY{l+s+s2}{\PYZdq{}}\PY{p}{)}
\PY{n}{mlp} \PY{o}{=} \PY{n}{MLPClassifier}\PY{p}{(}\PY{n}{hidden\PYZus{}layer\PYZus{}sizes}\PY{o}{=}\PY{p}{(}\PY{l+m+mi}{50}\PY{p}{,}\PY{l+m+mi}{50}\PY{p}{)}\PY{p}{,} \PY{n}{max\PYZus{}iter}\PY{o}{=}\PY{l+m+mi}{1000}\PY{p}{,} \PY{n}{random\PYZus{}state}\PY{o}{=}\PY{n}{SEED}\PY{p}{)}
\PY{n}{start} \PY{o}{=} \PY{n}{time}\PY{o}{.}\PY{n}{time}\PY{p}{(}\PY{p}{)}
\PY{n}{mlp}\PY{o}{.}\PY{n}{fit}\PY{p}{(}\PY{n}{train\PYZus{}raw}\PY{p}{,} \PY{n}{target}\PY{p}{)}
\PY{n}{end} \PY{o}{=} \PY{n}{time}\PY{o}{.}\PY{n}{time}\PY{p}{(}\PY{p}{)}
\PY{n}{time\PYZus{}taken} \PY{o}{=} \PY{p}{(}\PY{n}{end}\PY{o}{\PYZhy{}}\PY{n}{start}\PY{p}{)}\PY{o}{*}\PY{l+m+mi}{1000}
\PY{n+nb}{print}\PY{p}{(}\PY{l+s+s2}{\PYZdq{}}\PY{l+s+s2}{Time taken to fit:}\PY{l+s+s2}{\PYZdq{}}\PY{p}{,} \PY{n+nb}{round}\PY{p}{(}\PY{n}{time\PYZus{}taken}\PY{p}{,}\PY{l+m+mi}{5}\PY{p}{)}\PY{p}{,} \PY{l+s+s2}{\PYZdq{}}\PY{l+s+s2}{ milliseconds}\PY{l+s+s2}{\PYZdq{}}\PY{p}{)}
\PY{n}{start} \PY{o}{=} \PY{n}{time}\PY{o}{.}\PY{n}{time}\PY{p}{(}\PY{p}{)}
\PY{n}{mlp}\PY{o}{.}\PY{n}{predict}\PY{p}{(}\PY{n}{test\PYZus{}raw}\PY{p}{)}
\PY{n}{end}\PY{o}{=}\PY{n}{time}\PY{o}{.}\PY{n}{time}\PY{p}{(}\PY{p}{)}
\PY{n}{time\PYZus{}taken} \PY{o}{=} \PY{p}{(}\PY{n}{end}\PY{o}{\PYZhy{}}\PY{n}{start}\PY{p}{)}\PY{o}{*}\PY{l+m+mi}{1000}
\PY{n+nb}{print}\PY{p}{(}\PY{l+s+s2}{\PYZdq{}}\PY{l+s+s2}{Time taken to predict:}\PY{l+s+s2}{\PYZdq{}}\PY{p}{,} \PY{n+nb}{round}\PY{p}{(}\PY{n}{time\PYZus{}taken}\PY{p}{,}\PY{l+m+mi}{5}\PY{p}{)}\PY{p}{,} \PY{l+s+s2}{\PYZdq{}}\PY{l+s+s2}{ milliseconds}\PY{l+s+s2}{\PYZdq{}}\PY{p}{)}
\end{Verbatim}
\end{tcolorbox}

    \begin{Verbatim}[commandchars=\\\{\}]
Decision Tree Classifier
Time taken to fit: 5.14221  milliseconds
Time taken to predict: 2.60186  milliseconds

Multi-Layer Perceptron
Time taken to fit: 1304.97146  milliseconds
Time taken to predict: 4.7667  milliseconds
    \end{Verbatim}

    \begin{Verbatim}[commandchars=\\\{\}]
/usr/local/lib/python3.10/dist-
packages/sklearn/neural\_network/\_multilayer\_perceptron.py:686:
ConvergenceWarning: Stochastic Optimizer: Maximum iterations (1000) reached and
the optimization hasn't converged yet.
  warnings.warn(
    \end{Verbatim}

    Decision Trees were much quicker than the MLP.

    \hypertarget{pts-compare-and-contrast-decision-trees-and-mlps.}{%
\subsubsection{2.3 {[}5 pts{]} Compare and contrast Decision Trees and
MLPs.}\label{pts-compare-and-contrast-decision-trees-and-mlps.}}

    \textbf{Describe at least one advantage and disadvantage of using an MLP
over a Decision Tree.}

    \textbf{Advantage}: MLP can model highly complex decision boundaries,
making them better suited to capturing the intricacies of the
relationships between features and labels. This makes MLPs particularly
powerful in tasks involving image recognition, natural language
processing, and other applications involving the representation of
complex features.

\textbf{Disadvantage}: MLPs typically require more computational
resources during training than decision trees. Additionally, MLPs often
require careful tuning of hyperparameters such as learning rate, number
of layers, and number of neurons per layer, which can be a complex and
time-consuming process. In contrast, decision trees are generally faster
to train and require less computational power.

    \hypertarget{pts-pca}{%
\subsection{3 (35 pts) PCA}\label{pts-pca}}

    \hypertarget{pts-transform-the-train-data-using-pca}{%
\subsubsection{3.1 {[}5 pts{]} Transform the train data using
PCA}\label{pts-transform-the-train-data-using-pca}}

Train a PCA model to project the train data on the top 10 components.
\textbf{Print out the 10 principal components}. Look at the
documentation of
\href{https://scikit-learn.org/stable/modules/generated/sklearn.decomposition.PCA.html}{PCA}
for reference.

    \begin{tcolorbox}[breakable, size=fbox, boxrule=1pt, pad at break*=1mm,colback=cellbackground, colframe=cellborder]
\prompt{In}{incolor}{19}{\boxspacing}
\begin{Verbatim}[commandchars=\\\{\}]
\PY{c+c1}{\PYZsh{} TODO}
\PY{n}{pca} \PY{o}{=} \PY{n}{PCA}\PY{p}{(}\PY{n}{n\PYZus{}components}\PY{o}{=}\PY{l+m+mi}{10}\PY{p}{)}
\PY{n}{pca}\PY{o}{.}\PY{n}{fit}\PY{p}{(}\PY{n}{train\PYZus{}raw}\PY{p}{)}
\PY{n}{pca}\PY{o}{.}\PY{n}{components\PYZus{}}
\end{Verbatim}
\end{tcolorbox}

            \begin{tcolorbox}[breakable, size=fbox, boxrule=.5pt, pad at break*=1mm, opacityfill=0]
\prompt{Out}{outcolor}{19}{\boxspacing}
\begin{Verbatim}[commandchars=\\\{\}]
array([[ 2.86302825e-02, -1.57112401e-03, -1.07081012e-03,
         3.06295216e-02,  9.99068335e-01, -3.62199146e-04,
        -1.83163373e-03,  9.09393485e-03,  7.20851679e-04,
         2.75262314e-03, -9.24029293e-04,  1.01649988e-03,
         2.19465285e-03],
       [ 1.78551783e-01,  1.10060700e-03, -1.56111507e-02,
         2.34612134e-01, -3.72197428e-03,  9.43064773e-04,
        -1.87604619e-03, -9.55041306e-01,  8.48205273e-03,
         2.00625169e-02, -1.21725721e-02,  8.92892650e-03,
         2.64264742e-03],
       [ 6.69982412e-02, -9.11728712e-04,  7.59615929e-03,
         9.65177653e-01, -3.38334570e-02,  4.54618399e-03,
        -5.03130343e-03,  2.49984452e-01, -1.64177537e-03,
         1.27102571e-02, -2.19480478e-03,  5.13810654e-03,
         2.53505829e-03],
       [ 9.80938550e-01, -2.60163853e-03,  7.40660758e-03,
        -1.09795373e-01, -2.61875433e-02,  3.56487678e-03,
        -7.67599865e-03,  1.56529148e-01, -7.27865899e-03,
         2.88350450e-03,  2.82965735e-03,  1.73161717e-02,
        -3.86897078e-03],
       [-1.16784453e-02,  6.16030585e-02, -3.04193632e-01,
        -1.64359797e-02, -2.89866589e-03, -2.23540199e-02,
        -2.93627325e-02,  2.36482487e-02,  1.16711470e-01,
         7.85440577e-01, -2.01402495e-01,  4.55223046e-01,
         1.51676636e-01],
       [ 8.24638411e-03, -1.82787034e-02,  4.29528729e-01,
        -7.11941735e-03, -4.46297665e-04, -9.21935088e-03,
         6.55168925e-02, -3.28287023e-04,  1.41419987e-02,
         5.20623808e-01, -1.97091442e-01, -7.06761187e-01,
         3.26947397e-02],
       [-1.43100398e-02,  6.40370756e-02,  8.33626595e-01,
        -5.34039506e-03,  1.23033906e-03,  5.03328294e-02,
         7.28500917e-02, -1.33170277e-02, -1.08158091e-01,
         3.50142901e-02,  3.82096500e-02,  5.25884703e-01,
         4.21236033e-02],
       [-7.70163536e-03, -2.21489498e-01,  8.92742021e-03,
        -8.03465235e-04,  1.25877129e-03,  6.76332824e-02,
        -1.24739445e-01,  1.85228561e-04, -7.46207864e-02,
         1.26711727e-01, -1.35430258e-01,  8.47570755e-02,
        -9.39956219e-01],
       [ 5.17534919e-03, -1.35616795e-01, -1.11479320e-01,
         5.00671059e-03,  1.28056873e-03, -1.37716664e-01,
         8.87268478e-01, -1.47059840e-03, -2.31340793e-01,
         1.17009275e-01,  2.91220118e-01,  1.53897905e-02,
        -1.03236743e-01],
       [ 6.75824353e-03,  7.06999842e-01, -1.84830242e-02,
         2.11659108e-03,  1.70706764e-03,  1.23071264e-01,
         3.52770735e-01,  6.64527341e-03,  2.83850906e-01,
        -1.72640455e-01, -4.64830353e-01,  1.27546922e-03,
        -1.83503948e-01]])
\end{Verbatim}
\end{tcolorbox}
        
    \hypertarget{pts-percentage-of-variance-explained-by-top-10-principal-components}{%
\subsubsection{3.2 {[}5 pts{]} Percentage of variance explained by top
10 principal
components}\label{pts-percentage-of-variance-explained-by-top-10-principal-components}}

Using PCA's ``explained\_variance\_ratio\_'', print the percentage of
variance explained by the top 10 principal components.

    \begin{tcolorbox}[breakable, size=fbox, boxrule=1pt, pad at break*=1mm,colback=cellbackground, colframe=cellborder]
\prompt{In}{incolor}{20}{\boxspacing}
\begin{Verbatim}[commandchars=\\\{\}]
\PY{c+c1}{\PYZsh{} TODO}
\PY{n}{proportions} \PY{o}{=} \PY{n}{pca}\PY{o}{.}\PY{n}{explained\PYZus{}variance\PYZus{}ratio\PYZus{}}
\PY{n+nb}{print}\PY{p}{(}\PY{l+s+s2}{\PYZdq{}}\PY{l+s+s2}{Percentage of variance explained by top 10 components:}\PY{l+s+s2}{\PYZdq{}}\PY{p}{,}\PY{n+nb}{round}\PY{p}{(}\PY{n}{np}\PY{o}{.}\PY{n}{sum}\PY{p}{(}\PY{n}{proportions}\PY{p}{)}\PY{o}{*}\PY{l+m+mi}{100}\PY{p}{,} \PY{l+m+mi}{2}\PY{p}{)}\PY{p}{,} \PY{l+s+s2}{\PYZdq{}}\PY{l+s+s2}{\PYZpc{}}\PY{l+s+s2}{\PYZdq{}}\PY{p}{)}
\end{Verbatim}
\end{tcolorbox}

    \begin{Verbatim}[commandchars=\\\{\}]
Percentage of variance explained by top 10 components: 99.99 \%
    \end{Verbatim}

    \hypertarget{pts-transform-the-train-and-test-data-into-train_pca-and-test_pca-using-pca}{%
\subsubsection{3.3 {[}5 pts{]} Transform the train and test data into
train\_pca and test\_pca using
PCA}\label{pts-transform-the-train-and-test-data-into-train_pca-and-test_pca-using-pca}}

Note: Use fit\_transform for train and transform for test

    \begin{tcolorbox}[breakable, size=fbox, boxrule=1pt, pad at break*=1mm,colback=cellbackground, colframe=cellborder]
\prompt{In}{incolor}{21}{\boxspacing}
\begin{Verbatim}[commandchars=\\\{\}]
\PY{c+c1}{\PYZsh{} TODO}
\PY{n}{train\PYZus{}pca} \PY{o}{=} \PY{n}{pca}\PY{o}{.}\PY{n}{fit\PYZus{}transform}\PY{p}{(}\PY{n}{train\PYZus{}raw}\PY{p}{)}
\PY{n}{test\PYZus{}pca} \PY{o}{=} \PY{n}{pca}\PY{o}{.}\PY{n}{transform}\PY{p}{(}\PY{n}{test\PYZus{}raw}\PY{p}{)}
\end{Verbatim}
\end{tcolorbox}

    \hypertarget{pts-pcadecision-tree}{%
\subsubsection{3.4 {[}5 pts{]} PCA+Decision
Tree}\label{pts-pcadecision-tree}}

Train the default Decision Tree Classifier using train\_pca.
\textbf{Report the accuracy using test\_pca and print the confusion
matrix}.

    \begin{tcolorbox}[breakable, size=fbox, boxrule=1pt, pad at break*=1mm,colback=cellbackground, colframe=cellborder]
\prompt{In}{incolor}{22}{\boxspacing}
\begin{Verbatim}[commandchars=\\\{\}]
\PY{c+c1}{\PYZsh{} TODO}
\PY{n}{decision\PYZus{}tree} \PY{o}{=} \PY{n}{DecisionTreeClassifier}\PY{p}{(}\PY{p}{)}
\PY{n}{decision\PYZus{}tree}\PY{o}{.}\PY{n}{fit}\PY{p}{(}\PY{n}{train\PYZus{}pca}\PY{p}{,} \PY{n}{target}\PY{p}{)}
\PY{n}{prediction} \PY{o}{=} \PY{n}{decision\PYZus{}tree}\PY{o}{.}\PY{n}{predict}\PY{p}{(}\PY{n}{test\PYZus{}pca}\PY{p}{)}
\PY{n+nb}{print}\PY{p}{(}\PY{l+s+s2}{\PYZdq{}}\PY{l+s+si}{\PYZpc{}\PYZhy{}12s}\PY{l+s+s2}{ }\PY{l+s+si}{\PYZpc{}f}\PY{l+s+s2}{\PYZdq{}} \PY{o}{\PYZpc{}} \PY{p}{(}\PY{l+s+s1}{\PYZsq{}}\PY{l+s+s1}{Accuracy:}\PY{l+s+s1}{\PYZsq{}}\PY{p}{,} \PY{n}{metrics}\PY{o}{.}\PY{n}{accuracy\PYZus{}score}\PY{p}{(}\PY{n}{target\PYZus{}test}\PY{p}{,}\PY{n}{prediction}\PY{p}{)}\PY{p}{)}\PY{p}{)}
\end{Verbatim}
\end{tcolorbox}

    \begin{Verbatim}[commandchars=\\\{\}]
Accuracy:    0.680328
    \end{Verbatim}

    \begin{tcolorbox}[breakable, size=fbox, boxrule=1pt, pad at break*=1mm,colback=cellbackground, colframe=cellborder]
\prompt{In}{incolor}{23}{\boxspacing}
\begin{Verbatim}[commandchars=\\\{\}]
\PY{c+c1}{\PYZsh{}Printing out the confusion matrix}
\PY{n}{hel}\PY{o}{.}\PY{n}{draw\PYZus{}confusion\PYZus{}matrix}\PY{p}{(}\PY{n}{target\PYZus{}test}\PY{p}{,} \PY{n}{prediction}\PY{p}{,} \PY{p}{[}\PY{l+s+s2}{\PYZdq{}}\PY{l+s+s2}{Not Sick (0)}\PY{l+s+s2}{\PYZdq{}}\PY{p}{,} \PY{l+s+s2}{\PYZdq{}}\PY{l+s+s2}{Sick(1)}\PY{l+s+s2}{\PYZdq{}}\PY{p}{]}\PY{p}{)}
\end{Verbatim}
\end{tcolorbox}

    \begin{center}
    \adjustimage{max size={0.9\linewidth}{0.9\paperheight}}{Project_3a_files/Project_3a_49_0.png}
    \end{center}
    { \hspace*{\fill} \\}
    
    \textbf{Does the model perform better with or without PCA?}

    Response: This model performs better without PCA since the accuracy for
DT model is 0.70, but for the PCA model the accuracy is lower (0.68)

    \hypertarget{pts-pcamlp}{%
\subsubsection{3.5 {[}5 pts{]} PCA+MLP}\label{pts-pcamlp}}

Train the MLP classifier with the same parameters as before using
train\_pca. \textbf{Report the accuracy using test\_pca and print the
confusion matrix}.

    \begin{tcolorbox}[breakable, size=fbox, boxrule=1pt, pad at break*=1mm,colback=cellbackground, colframe=cellborder]
\prompt{In}{incolor}{24}{\boxspacing}
\begin{Verbatim}[commandchars=\\\{\}]
\PY{c+c1}{\PYZsh{} TODO}
\PY{n}{mlp} \PY{o}{=} \PY{n}{MLPClassifier}\PY{p}{(}\PY{n}{hidden\PYZus{}layer\PYZus{}sizes}\PY{o}{=}\PY{p}{(}\PY{l+m+mi}{50}\PY{p}{,}\PY{l+m+mi}{50}\PY{p}{)}\PY{p}{,} \PY{n}{max\PYZus{}iter}\PY{o}{=}\PY{l+m+mi}{1000}\PY{p}{,} \PY{n}{random\PYZus{}state}\PY{o}{=}\PY{n}{SEED}\PY{p}{)}
\PY{n}{mlp}\PY{o}{.}\PY{n}{fit}\PY{p}{(}\PY{n}{train\PYZus{}pca}\PY{p}{,} \PY{n}{target}\PY{p}{)}
\PY{n}{prediction} \PY{o}{=} \PY{n}{mlp}\PY{o}{.}\PY{n}{predict}\PY{p}{(}\PY{n}{test\PYZus{}pca}\PY{p}{)}
\PY{n+nb}{print}\PY{p}{(}\PY{l+s+s2}{\PYZdq{}}\PY{l+s+si}{\PYZpc{}\PYZhy{}12s}\PY{l+s+s2}{ }\PY{l+s+si}{\PYZpc{}f}\PY{l+s+s2}{\PYZdq{}} \PY{o}{\PYZpc{}} \PY{p}{(}\PY{l+s+s1}{\PYZsq{}}\PY{l+s+s1}{Accuracy:}\PY{l+s+s1}{\PYZsq{}}\PY{p}{,} \PY{n}{metrics}\PY{o}{.}\PY{n}{accuracy\PYZus{}score}\PY{p}{(}\PY{n}{target\PYZus{}test}\PY{p}{,}\PY{n}{prediction}\PY{p}{)}\PY{p}{)}\PY{p}{)}
\end{Verbatim}
\end{tcolorbox}

    \begin{Verbatim}[commandchars=\\\{\}]
Accuracy:    0.696721
    \end{Verbatim}

    \begin{tcolorbox}[breakable, size=fbox, boxrule=1pt, pad at break*=1mm,colback=cellbackground, colframe=cellborder]
\prompt{In}{incolor}{25}{\boxspacing}
\begin{Verbatim}[commandchars=\\\{\}]
\PY{c+c1}{\PYZsh{}Printing out the confusion matrix}
\PY{n}{hel}\PY{o}{.}\PY{n}{draw\PYZus{}confusion\PYZus{}matrix}\PY{p}{(}\PY{n}{target\PYZus{}test}\PY{p}{,} \PY{n}{prediction}\PY{p}{,} \PY{p}{[}\PY{l+s+s2}{\PYZdq{}}\PY{l+s+s2}{Not Sick (0)}\PY{l+s+s2}{\PYZdq{}}\PY{p}{,} \PY{l+s+s2}{\PYZdq{}}\PY{l+s+s2}{Sick(1)}\PY{l+s+s2}{\PYZdq{}}\PY{p}{]}\PY{p}{)}
\end{Verbatim}
\end{tcolorbox}

    \begin{center}
    \adjustimage{max size={0.9\linewidth}{0.9\paperheight}}{Project_3a_files/Project_3a_54_0.png}
    \end{center}
    { \hspace*{\fill} \\}
    
    \textbf{Does the model perform better with or without PCA?}

    Response: The model performs significantly better without PCA since the
accuracy for the MLP classifier without PCA is 0.80, but with PCA it is
0.69.

    \hypertarget{pts-pros-and-cons-of-pca}{%
\subsubsection{3.6 {[}10 pts{]} Pros and Cons of
PCA}\label{pts-pros-and-cons-of-pca}}

\textbf{In your own words, provide at least two pros and at least two
cons for using PCA}

    Pros:

\begin{itemize}
\tightlist
\item
  Dimensionality Reduction: Reduces the dimensionality of the feature
  space while retaining most of the data variance, simplifying the
  dataset and making it easier to visualize and interpret.
\item
  Noise Reduction: By focusing only on the principal components of the
  data, PCA can help filter out noise from the data, improving the
  performance of the model.
\item
  Feature Extraction: PCA can be used to create new, uncorrelated
  features that can enhance the predictive power of models.
\end{itemize}

Cons:

\begin{itemize}
\item
  Loss of Interpretability: Principal Components are linear combinations
  of original features and may not be easily interpretable.
\item
  Assumes Linearity: PCA assumes that the relationship between the
  features is linear. It may not be able to effectively capture complex,
  non-linear relationships between features.
\item
  Variance Centric: PCA focuses on components that explain the most
  variance, but these components may not always be the most important
  for the specific predictive task. Important but less varying features
  might be ignored.
\item
  Sensitivity to Scaling: PCA is sensitive to the relative scaling of
  the original features. If features are on different scales, PCA
  results can be misleading unless data is properly normalized or
  standardized.
\end{itemize}

    \hypertarget{pts-k-means-clustering}{%
\subsection{4. (20 pts) K-Means
Clustering}\label{pts-k-means-clustering}}

    \hypertarget{pts-apply-k-means-to-the-train-data-and-print-out-the-inertia-score}{%
\subsubsection{4.1 {[}5 pts{]} Apply K-means to the train data and print
out the Inertia
score}\label{pts-apply-k-means-to-the-train-data-and-print-out-the-inertia-score}}

Use n\_cluster = 5 and \texttt{random\_state\ =\ SEED}.

    \begin{tcolorbox}[breakable, size=fbox, boxrule=1pt, pad at break*=1mm,colback=cellbackground, colframe=cellborder]
\prompt{In}{incolor}{26}{\boxspacing}
\begin{Verbatim}[commandchars=\\\{\}]
\PY{c+c1}{\PYZsh{} TODO}
\PY{n}{kmeans\PYZus{}clustering} \PY{o}{=} \PY{n}{KMeans}\PY{p}{(}\PY{n}{random\PYZus{}state}\PY{o}{=}\PY{n}{SEED}\PY{p}{,} \PY{n}{n\PYZus{}clusters}\PY{o}{=}\PY{l+m+mi}{5}\PY{p}{,} \PY{n}{n\PYZus{}init}\PY{o}{=}\PY{l+s+s1}{\PYZsq{}}\PY{l+s+s1}{auto}\PY{l+s+s1}{\PYZsq{}}\PY{p}{)}
\PY{n}{kmeans\PYZus{}clustering}\PY{o}{.}\PY{n}{fit}\PY{p}{(}\PY{n}{train\PYZus{}raw}\PY{p}{)}
\PY{n+nb}{print}\PY{p}{(}\PY{l+s+s2}{\PYZdq{}}\PY{l+s+s2}{Inertia score:}\PY{l+s+s2}{\PYZdq{}}\PY{p}{,} \PY{n}{kmeans\PYZus{}clustering}\PY{o}{.}\PY{n}{inertia\PYZus{}}\PY{p}{)}
\end{Verbatim}
\end{tcolorbox}

    \begin{Verbatim}[commandchars=\\\{\}]
Inertia score: 208540.66305208334
    \end{Verbatim}

    \hypertarget{pts-find-the-optimal-cluster-size-using-the-elbow-method.}{%
\subsubsection{4.2 {[}10 pts{]} Find the optimal cluster size using the
elbow
method.}\label{pts-find-the-optimal-cluster-size-using-the-elbow-method.}}

Use the elbow method to find the best cluster size or range of best
cluster sizes for the train data. Check the cluster sizes from 2 to 25.
Make sure to plot the Inertia and state where you think the elbow
starts. Make sure to use \texttt{random\_state\ =\ SEED}.

    \begin{tcolorbox}[breakable, size=fbox, boxrule=1pt, pad at break*=1mm,colback=cellbackground, colframe=cellborder]
\prompt{In}{incolor}{27}{\boxspacing}
\begin{Verbatim}[commandchars=\\\{\}]
\PY{c+c1}{\PYZsh{} TODO}
\PY{n}{inertia} \PY{o}{=} \PY{p}{[}\PY{p}{]}
\PY{n}{n\PYZus{}cluster} \PY{o}{=} \PY{p}{[}\PY{p}{]}
\PY{k}{for} \PY{n}{i} \PY{o+ow}{in} \PY{n+nb}{range}\PY{p}{(}\PY{l+m+mi}{2}\PY{p}{,}\PY{l+m+mi}{26}\PY{p}{)}\PY{p}{:}
  \PY{n}{kmeans\PYZus{}clustering} \PY{o}{=} \PY{n}{KMeans}\PY{p}{(}\PY{n}{random\PYZus{}state}\PY{o}{=}\PY{n}{SEED}\PY{p}{,} \PY{n}{n\PYZus{}clusters}\PY{o}{=}\PY{n}{i}\PY{p}{,} \PY{n}{n\PYZus{}init}\PY{o}{=}\PY{l+s+s1}{\PYZsq{}}\PY{l+s+s1}{auto}\PY{l+s+s1}{\PYZsq{}}\PY{p}{)}
  \PY{n}{kmeans\PYZus{}clustering}\PY{o}{.}\PY{n}{fit}\PY{p}{(}\PY{n}{train\PYZus{}raw}\PY{p}{)}
  \PY{n}{n\PYZus{}cluster}\PY{o}{.}\PY{n}{append}\PY{p}{(}\PY{n}{i}\PY{p}{)}
  \PY{n}{inertia}\PY{o}{.}\PY{n}{append}\PY{p}{(}\PY{n}{kmeans\PYZus{}clustering}\PY{o}{.}\PY{n}{inertia\PYZus{}}\PY{p}{)}


\PY{n}{plt}\PY{o}{.}\PY{n}{plot}\PY{p}{(}\PY{n}{n\PYZus{}cluster}\PY{p}{,} \PY{n}{inertia}\PY{p}{)}
\PY{n}{plt}\PY{o}{.}\PY{n}{xlabel}\PY{p}{(}\PY{l+s+s2}{\PYZdq{}}\PY{l+s+s2}{Number of Clusters}\PY{l+s+s2}{\PYZdq{}}\PY{p}{)}
\PY{n}{plt}\PY{o}{.}\PY{n}{ylabel}\PY{p}{(}\PY{l+s+s2}{\PYZdq{}}\PY{l+s+s2}{Inertia}\PY{l+s+s2}{\PYZdq{}}\PY{p}{)}
\PY{n}{plt}\PY{o}{.}\PY{n}{show}\PY{p}{(}\PY{p}{)}
\end{Verbatim}
\end{tcolorbox}

    \begin{center}
    \adjustimage{max size={0.9\linewidth}{0.9\paperheight}}{Project_3a_files/Project_3a_63_0.png}
    \end{center}
    { \hspace*{\fill} \\}
    
    From the plot, we can guess that the best cluster size is somewhere
between 5 and 10.

    \hypertarget{pts-find-the-optimal-cluster-size-for-the-train_pca-data}{%
\subsubsection{4.3 {[}5 pts{]} Find the optimal cluster size for the
train\_pca
data}\label{pts-find-the-optimal-cluster-size-for-the-train_pca-data}}

Repeat the same experiment but use train\_pca instead of train.

    \begin{tcolorbox}[breakable, size=fbox, boxrule=1pt, pad at break*=1mm,colback=cellbackground, colframe=cellborder]
\prompt{In}{incolor}{28}{\boxspacing}
\begin{Verbatim}[commandchars=\\\{\}]
\PY{c+c1}{\PYZsh{} TODO}
\PY{n}{inertia} \PY{o}{=} \PY{p}{[}\PY{p}{]}
\PY{n}{n\PYZus{}cluster} \PY{o}{=} \PY{p}{[}\PY{p}{]}
\PY{k}{for} \PY{n}{i} \PY{o+ow}{in} \PY{n+nb}{range}\PY{p}{(}\PY{l+m+mi}{2}\PY{p}{,}\PY{l+m+mi}{26}\PY{p}{)}\PY{p}{:}
  \PY{n}{kmeans\PYZus{}clustering} \PY{o}{=} \PY{n}{KMeans}\PY{p}{(}\PY{n}{random\PYZus{}state}\PY{o}{=}\PY{n}{SEED}\PY{p}{,} \PY{n}{n\PYZus{}clusters}\PY{o}{=}\PY{n}{i}\PY{p}{,} \PY{n}{n\PYZus{}init}\PY{o}{=}\PY{l+s+s1}{\PYZsq{}}\PY{l+s+s1}{auto}\PY{l+s+s1}{\PYZsq{}}\PY{p}{)}
  \PY{n}{kmeans\PYZus{}clustering}\PY{o}{.}\PY{n}{fit}\PY{p}{(}\PY{n}{train\PYZus{}pca}\PY{p}{)}
  \PY{n}{n\PYZus{}cluster}\PY{o}{.}\PY{n}{append}\PY{p}{(}\PY{n}{i}\PY{p}{)}
  \PY{n}{inertia}\PY{o}{.}\PY{n}{append}\PY{p}{(}\PY{n}{kmeans\PYZus{}clustering}\PY{o}{.}\PY{n}{inertia\PYZus{}}\PY{p}{)}

\PY{n}{plt}\PY{o}{.}\PY{n}{plot}\PY{p}{(}\PY{n}{n\PYZus{}cluster}\PY{p}{,} \PY{n}{inertia}\PY{p}{)}
\PY{n}{plt}\PY{o}{.}\PY{n}{xlabel}\PY{p}{(}\PY{l+s+s2}{\PYZdq{}}\PY{l+s+s2}{Number of Clusters}\PY{l+s+s2}{\PYZdq{}}\PY{p}{)}
\PY{n}{plt}\PY{o}{.}\PY{n}{ylabel}\PY{p}{(}\PY{l+s+s2}{\PYZdq{}}\PY{l+s+s2}{Inertia}\PY{l+s+s2}{\PYZdq{}}\PY{p}{)}
\PY{n}{plt}\PY{o}{.}\PY{n}{show}\PY{p}{(}\PY{p}{)}
\end{Verbatim}
\end{tcolorbox}

    \begin{center}
    \adjustimage{max size={0.9\linewidth}{0.9\paperheight}}{Project_3a_files/Project_3a_66_0.png}
    \end{center}
    { \hspace*{\fill} \\}
    
    Similar to the previous experiment, we can guess that the best cluster
size is somewhere between 5 and 10. Additionally, we see that the
inertia is much smaller for every cluster size when using PCA features.


    % Add a bibliography block to the postdoc
    
    
    
\end{document}
